\part{Verschlüsselung Allgemein}
\section{Sinn und Zweck der Verschlüsselung}


\section{Geschichte der Verschlüsselung}
\subsection{Klassische Kryptografie}
Die Kryptographie bezeichnet die Entwicklung von Methoden zur Verheimlichung von Nachrichten.
Solche Methoden wurden schon 3000 Jahre vor Christus bei den Ägyptern verwendet(Hyrogliphen). \\
Die Hebräer haben 600 vor Christus das Atbash entwickelt. Dabei wurde das Alphabet umgekehrt und entsprechend verschlüsselt.
Als kleines Beispiel

\begin{table}[ht]
\caption{Atbash Verschlüsselung}
\begin{center}
\begin{tabular}{|l|l|l|l|l|l|l|}
  a & b & c & d & e & f\\
  z & y & x & w & v & u\\
\end{tabular}
\end{center}
\end{table}


Die Caesar-Verschlüsselung wurde nach Julius Caesar benannt und zur militärischen Korrespondenz im Römischen Reich gebraucht.\\
Dafür wurde damals das Alphabet um 3 Stellen nach hinten verschoben. Aus einem D wird somit ein A. Dieses Verfahren wurde im 15.-Jahrhundert mit einer
Chiffrierscheibe verbessert und ist bis heute noch als Caesar Verschlüsselung bekannt. Die Verschiebung um 13 Stellen wird auch als Rot13 bezeichnet, 
da das Alphabet aus 26 Zeichen besteht, ist die Verschlüsselung und Entschlüsselung bei Rot13 die selbe.

\begin{table}[ht]
\caption{Caesar Verschlüsselung um 3 Zeichen}
\begin{center}
\begin{tabular}{|l|l|l|l|l|l|l|}
  a & b & c & d & e & f\\
  x & y & z & a & b & c\\
\end{tabular}
\end{center}
\end{table}

All diese Verfahren können ziemlich eifach geknackt werden. Da bestimmte Buchstaben in einer Sprache öfter vorkommen, kann durch Textanalyse die Verschiebung herausgefunden werden. Ebenfalls wird der Grundsatz verletzt, dass die Sicherheit nicht von der Geheimhaltung
des Algorythmus abhängen 

\subsection{Kryptographie im zweiten Weltkrieg}
Im zweiten Weltkrieg nutzten die Deutsche Wehrmacht die ENIGMA zur Verschlüsselung ihres Funkverkehrs. Die ENIGMA ist wie eine Schreibmaschine zu bedienen. Die ENIGMA besteht aus der Tastatur, einem Walzensatz und einer Anzeige über Lampen. 

\subsubsection{Funktion der ENIGMA Maschine}
Die Walzen können sich drehen und sind mit Elektrischen Kontakten miteinander verbunden. Wird eine Taste gedrückt, fliesst Strom durch den Walzenssatz und es leuchtet die Lampe des verschlüsselten Buchstabens auf. Die Walze dreht sich bei jedem Tastendruck weiter, so das der gleiche Buchstaben jeweils anders verschlüsselt wird. Z. B. AAA wird zu DEF. Nach 26 Umdrehungen der ersten Walze wird die nächste Walze gedreht. \\
Dabei gehen jeweils elektrische Impulse durch die Walze. Nach jeder Drehung legt der Strom eine andere Strecke durch die Walzen zurück. 
Über das Steckbrett konnte der Verschlüsselte Buchstaben nochmals ersetzt werden. z. B. wurde E mit J verbunden. Falls nach dem verschlüsseln ein E herauskam, wurde dieses mit J ersetzt. 

\subsubsection{Anwendung}
Die Deutsche Wehrmacht nutzte die ENIGMA um ihre Funksprüche zu verschlüsseln. Dazu wurden jeweils um Mitternacht die Walzen und die Steckverbindungen ausgetauscht bzw. umgesteckt. Ein neuer Schlüsselsatz sah beispielsweise so aus:

Quelle Wikipedia: Schlüsseltafel der Wehrmacht: Täglich um 
Tag UKW  Walzenlage  Ringstellung  ---- Steckerverbindungen ----
 31  B   I   IV III    16 26 08    AD CN ET FL GI JV KZ PU QY WX 

Die eigentliche Verschlüsselung war relativ simpel zu bewerkstelligen. Es musste nur die entsprechende Funknachricht eingegeben werden und die ENIGMA erledigte die Verschlüsselung. Der Austausch der aktuellen Walzenstellung und das Morsen der Nachricht waren jedoch wesentlich komplizierter.

Vor der Eigentlichen Nachricht wurde ein Nachrichtenkopf gemorst. In diesem wurde die länge des Textes bekannt gegeben, die Walzenstellung und die Nachrichtenart. 
Der eigentliche Text wurde in Gruppen aus 5 Buchstaben gemorst. Bei den ersten 5 Buchstaben wurde bestimmt, an wenn die Nachricht geht. Dabei wurden die ersten zwei Buchstaben zufällig ausgewählt und die letzten 3 durchmischt. Nach der Alphabetischen Ordnung, konnte über eine Tabelle festgestellt werden, ob die Nachricht einem was angeht, bevor man sie entschlüsselt.
Die Walzenstellung wurde ebenfalls nicht einfach übertragen. Wenn im Kopf z. B. QWE EWG angegeben wurde, hiess das man soll die Walze auf die Buchstaben QWE einstellen und EWG eingeben. Was dabei verschlüsselt heraus kam, war die Anfangsstellung der Walzen. Jetzt konnte mit der Decodierung der Nachricht begonnen werden.

\subsubsection{Schwachstelle der ENIGMA}
Die Verschlüsselung und Entschlüsselung waren jeweils gleich. Das bedeutet, es gab halb so viele Möglichkeiten, da bei gleicher Walzenstellung und Stecker aus einem B ein H wird und aus einem H ein B. 
Schlüsselraum
Es standen 5 Walzen und 2 Umkehrwalzen zur Verfügung. Davon wurden jeweils 3 Walzen und 1 Umkehrwalze eingesetzt. 
Von 5 möglichen Walzen werden 3 ausgewählt. Zusätzlich aus 2 Umehrwalzen eine. 
Daraus können die Anzahl Möglichkeiten wie folgt gerechnet werden: 
(5*4*3)*2 = 120




\subsection{Probleme der früheren Verschlüsselungen}
Die Verfahren bis zum zweiten Weltkrieg konnten relativ einfach geknackt werden. Wenn die Art der Verschlüsselung bekannt war, konnte der Schlüssel einfach ermittelt werden. 



\section{Moderne Kryptographie}[hb]

\subsubsection{Symmetrische Verschlüsselung}
\subsubsection{Asymmetrische Verschlüsselung}
Vor den 1970er Jahre, als es die asymmetrische Verschlüsselung noch nicht gab, wurden Schlüssel für eine symmetrische Verschlüsselung streng bewacht transportiert.
Vorallem bei Botschaften, wenn man einen neuen Schlüssel brauchte, waren dass die Personen, die den Koffer an ihren Arm mit einer Handschelle befestigten und noch zusätzlich von 5 Bodyguards beschützt in eine Panzerfesten Limusinen von einer Botschaft zur anderen fuhren.\\
Dieses Verfahren war sehr umständlich. Es kostete viel und dauerte zu lang. Wie aber sollte man einen Schlüssel sicher von einem Ort zum andern transportieren und dabei sicher sein, dass niemand einen Blick auf den Schlüssel werfen kann.\\
Anfang 1970 stellten W. Diffe und M.Hellman ein konzept vor, das die Grundidee von einer Asymmetrischen Verschlüsselung zeigte. Sie kannten das Verfahren aber nicht.
Später entwickelten James Ellis, Clifford Cockes und Malcom Willamson, alle vom Britischen Geheimdienst, an einer Asymmetrischen Verschlüsselung. Sie durften aber ihre Ergebnisse nicht veröffentlichen geschweige ein Patent anmelden.\\
Der grosse Durchbruch gelang Ronald L. \textbf{R}ivest, Adi \textbf{S}hamir und Leonard M. \textbf{A}dleman im Jahre 1977. Sie entwickelten den RSA-Algorithmus der bis heute als eines der sichersten Verfahren giltet.\\[2ex]
%
\paragraph{Verschlüsseln}
Bei einem assymetrischen Verfahren kann jede Sender einem Empfänger mit hilfe einer Funktion f eine verschlüsselte Nachricht c senden, die nur der Empfänger mit seinem privaten Schlüssel entschlüsseln kann. Das Funktioniert dank einem öffentlichen Schlüssel \textbf{Public-Key} e und eine privaten Schlüssel \textbf{Private-Key} d. \\
Deshalb spricht man auch von einer  \textbf{Public Key-Kryptographie}. Es ist praktisch unmöglich vom öffentlichen Schlüssel auf den privaten Schlüssel zu kommen. Dass macht die Public Key-Kryptographie extrem sicher. 
% Mehr dazu im Kapitel Sicherheit %
Der Verschlüsselungsalgorithmus verschlüsselt den klartext in ein verschlüsselten Text. So ergibt sich die Formel.
\begin{center}
$ c = f_e (m) $
\end{center}
\paragraph{Entschlüsseln}
Beim Entschlüsseln ist es gearde umgekehrt. Hierfür wird mit der Entschlüsselungsfunktion f und dem privaten Schlüssel d aus dem verschlüsseltem Text c den klartext umgewandelt.
\begin{center}
$ m' = f_d (c) $
\end{center}
Damit der richtige Text heraus kommt bei der Entschlüsselung muss 
\begin{center}
$ m' = f_d (c) = f_d(f_e(m))  = m$
\end{center}
sein.


Der Empfänger erstellt zwei Schlüssel. Einen öffentlichen den er verteilen kann und einen geheimen privaten Schlüssel denn er keinem zeigt. 

