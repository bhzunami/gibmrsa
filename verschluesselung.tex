\part{Verschlüsselung Allgemein}
\section{Sinn und Zweck der Verschlüsselung}


\section{Geschichte der Verschlüsselung}
\subsection{Klassische Kryptografie}
Die Kryptographie bezeichnet die Entwicklung von Methoden zur Verheimlichung von Nachrichten.
Solche Methoden wurden schon 3000 Jahre vor Christus bei den Ägyptern verwendet(Hyrogliphen). \\
Die Hebräer haben 600 vor Christus das Atbash entwickelt. Dabei wurde das Alphabet umgekehrt und entsprechend verschlüsselt.
Als kleines Beispiel\\
\begin{table}[ht]
\caption{Atbash Verschlüsselung}
\begin{tabular}{p|p|p|p|p}
  \midrule
  a & b & c & d \\
   \midrule
  z & y & x & w \\
  \bottomrule
\end{tabular}
\end{table}


Die Caesar-Verschlüsselung wurde nach Julius Caesar benannt und zur militärischen Korrespondenz im Römischen Reich gebraucht.\\
Dafür wurde damals das Alphabet um 3 Stellen nach hinten verschoben. Aus einem D wird somit ein A. Dieses Verfahren wurde im 15.-Jahrhundert mit einer
Chiffrierscheibe verbessert und ist bis heute noch als Caesar Verschlüsselung bekannt. Die Verschiebung um 13 Stellen wird auch als Rot13 bezeichnet, 
da das Alphabet aus 26 Zeichen besteht, ist die Verschlüsselung und Entschlüsselung bei Rot13 die selbe.


\subsection{Kryptographie im zweiten Weltkrieg}
Im zweiten Weltkrieg wurden erste kompliziertere mathematische Verfahren verwendet. 

\subsection{Probleme der früheren Verschlüsselungen}




\section{Moderne Kryptographie}

\subsubsection{Symmetrische Verschlüsselung}
\subsubsection{Asymmetrische Verschlüsselung}
Vor den 1970er Jahre, als es die asymmetrische Verschlüsselung noch nicht gab, wurden Schlüssel für eine symmetrische Verschlüsselung streng bewacht transportiert.
Vorallem bei Botschaften, wenn man einen neuen Schlüssel brauchte, waren dass die Personen, die den Koffer an ihren Arm mit einer Handschelle befestigten und noch zusätzlich von 5 Bodyguards beschützt in eine Panzerfesten Limusinen von einer Botschaft zur anderen fuhren.\\
Dieses Verfahren war sehr umständlich. Es kostete viel und dauerte zu lang. Wie aber sollte man einen Schlüssel sicher von einem Ort zum andern transportieren und dabei sicher sein, dass niemand einen Blick auf den Schlüssel werfen kann.\\
Anfang 1970 stellten W. Diffe und M.Hellman ein konzept vor, das die Grundidee von einer Asymmetrischen Verschlüsselung zeigte. Sie kannten das Verfahren aber nicht.
Später entwickelten James Ellis, Clifford Cockes und Malcom Willamson, alle vom Britischen Geheimdienst, an einer Asymmetrischen Verschlüsselung. Sie durften aber ihre Ergebnisse nicht veröffentlichen geschweige ein Patent anmelden.\\
Der grosse Durchbruch gelang Ronald L. \textbf{R}ivest, Adi \textbf{S}hamir und Leonard M. \textbf{A}dleman im Jahre 1977. Sie entwickelten den RSA-Algorithmus der bis heute als eines der sichersten Verfahren giltet.\\[2ex]
%
Bei einem assymetrischen Verfahren kann jede Sender einem Empfänger eine verschlüsselte Nachricht senden, die nur der Empfänger entschlüsseln kann. Das Funktioniert dank einem öffentlichen Schlüssel \textbf{Public-Key} und eine privaten Schlüssel \textbf{Private-Key}. \\
Deshalb spricht man auch von einer  \textbf{Public Key-Kryptographie}.
Der Empfänger erstellt zwei Schlüssel. Einen öffentlichen den er verteilen kann und einen geheimen privaten Schlüssel denn er keinem zeigt. 

