\part{Sicherheit}
\section{Allgemeines}
%Referenz:http://netlab.cs.ucla.edu/wiki/files/shannon1949.pdf
%Bzw. Wikipedia: http://de.wikipedia.org/wiki/Beweisbare_Sicherheit
In der Kryptologie spricht man von einer beweisbaren Sicherheit, welche eigentlich keine ist. Um die Sicherheit zu beweisen werden verschiedene Annahmen getroffen. Dies bezieht sich oft auf die zur Verfügung stehenden Rechenleistung und den entsprechenden Zeitaufwand. Da die Rechenleistung zunimmt werden die Schlüssel mit der Zeit länger. \\
Der Mathematische Beweis ist jeweils eine Reduktion des Problems. Es wird dabei angenommen das es schwierig ist z. B. zwei Primzahlen wieder zu faktorisieren. Dabei ist aktuell nur keine effektive Möglichkeit gefunden worden, das Problem zu lösen. Es ist jedoch nicht bewiesen, ob es unmöglich oder gar schwierig ist.

%#######################################################
% Primzahlen und Faktorisierungsproblem
%#######################################################
\section{Primzahlen}
\subsection{Faktorisierungsproblem}
Die Sicherheit der RSA-Verschlüsselung beruht auf dem Faktorisierungsproblem. Es wurde bisher noch keine effektive Methode gefunden um grosse Zahlen in ihre Primzahlen zu faktorisieren. Besonders schwer ist es das die Zahl aus zwei grossen Primzahlen besteht. 
Ein kleines Beispiel:
Die Faktorisierung von 14 in die beiden Primzahlen 2 und 7 fällt nicht sehr schwer\\
Die Faktorisierung von 30518 ist ebenfalls einfach, da nur eine Primzahl gross gewählt wurde. Es handelt sich um 2 und 15259. \\
Die Faktorisierung von 771'151 ist schon wesentlich schwerer. Obwohl es die kleinen Zahlen 823 und 937 sind. Jedoch wird das RSA-Modul N aus zwei Primzahlen mit einer Länge von 2048 Bit (617 Dezimalstellen) erstellt. Wir sehen hier das die Sicherheit von RSA wesentlich von der gewählten Schlüssellänge abhängt.
%
%
\section{Angriffe auf den RSA-Algorithmus}
\subsection{Timing Attack}
\subsection{Exponent Attacks}
\section{Ausblick in die Zukunft}
