\section{Einleitung}
\subsection{Vorwort}
Bei der Themenwahl suchten wir ein Gebiet, welches uns interessiert und in unserem zukünftigen Berufsalltag begegnen wird. Wir entschieden uns für die Verschlüsselung, weil dies in der IT ein sehr wichtiges sowie aktuelles Thema ist. Da die Verschlüsselung zu allgemein definiert ist, haben wir das Thema weiter eingegrenzt und sind auf die RSA-Verschlüsselung gestossen. Diese asymmetrische Verschlüsselung wurde schon früh entwickelt, behielt bis heute ihre Bedeutung und wird in verschiedenen Anwendungen eingesetzt. Mit dieser Themenwahl konnten wir nach mehreren Anläufen auch unsere Betreuer davon überzeugen, dass wir dieses komplexe Thema bewältigen können. 
%
\subsection{Leitfrage}
Der RSA-Algorithmus ist im Jahr 1977 veröffentlicht worden. In der IT hat sich seit dieser Zeit viel verändert. Die Rechner sind wesentlich schneller geworden, es entstanden neue, verbesserte Algorithmen, und durch die grosse Verbreitung des Internets wurden asymmetrische Verschlüsselungen immer wichtiger. In unserer Arbeit möchten wir herausfinden, ob das erste veröffentlichte asymmetrische Verschlüsselungsverfahren auch heute noch sicher ist. Zusätzlich möchten wir verstehen, wie und warum der Algorithmus funktioniert. Um beides abzudecken, haben wir unsere Leitfrage wie folgt definiert:\\
\begin{center}\textit{Wieso ist RSA heute noch sicher?}\end{center}
