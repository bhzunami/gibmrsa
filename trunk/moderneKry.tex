\section{Moderne Kryptographie}
\subsubsection{Symmetrische Verschlüsselung}
\subsubsection{Asymmetrische Verschlüsselung}
Vor den 1970er Jahre, als es die asymmetrische Verschlüsselung noch nicht gab, wurden Schlüssel für eine symmetrische Verschlüsselung unter grossem Schutz transportiert.
% Besser ausdeutschen
% --->
Vorallem bei Botschaften, wenn man einen neuen Schlüssel brauchte, waren dass die Personen, die den Koffer an ihren Arm mit einer Handschelle befestigten und noch zusätzlich von 5 Bodyguards beschützt in eine Panzerfesten Limusinen von einer Botschaft zur anderen fuhren.\\
Dieses Verfahren war sehr umständlich. Es kostete viel und dauerte zu lang. Wie aber sollte man einen Schlüssel sicher von einem Ort zum andern transportieren und dabei sicher sein, dass niemand einen Blick auf den Schlüssel werfen kann.\\
% <---
Anfang 1970 stellten W. Diffe und M.Hellman ein Konzept vor, dass die Grundidee von einer Asymmetrischen Verschlüsselung zeigte. Sie kannten das Verfahren aber nicht exakt sondern stellten theoretische Ansätze vor.\\
Später entwickelten James Ellis, Clifford Cockes und Malcom Willamson, alle vom Britischen Geheimdienst, ein Asymmetrisches Verschlüsselungsverfahren. Sie durften aber ihre Ergebnisse nicht veröffentlichen geschweige ein Patent anmelden. 
Der Erfolg gelang Ronald L. \textbf{R}ivest, Adi \textbf{S}hamir und Leonard M. \textbf{A}dleman im Jahre 1977. Sie entwickelten den RSA-Algorithmus der bis heute als eines der sichersten asymmetrischen Verfahren giltet.\\[2ex]
%
Bei einem assymetrischen Verfahren kann jeder Sender einem Empfänger eine verschlüsselte Nachricht senden, die nur der Empfänger entschlüsseln kann.\\
%
Das Funktioniert dank einem öffentlichen Schlüssel \textbf{Public-Key} und einem privaten Schlüssel \textbf{Private-Key}. \\
Deshalb spricht man auch von einer  \textbf{Public Key-Kryptographie}.
%
% Bei der Formel für die verschlüsselung ist f_e und m verantwortlich für c das heisst es sind beide wichtig!!
\paragraph{Verschlüsseln}
Der Sender braucht den öffentlichen Schlüssel von dem Empfänger. Mit diesem Schlüssel und dem Verschlüsslungsverfahren, kann jetzt der Sender eine Nachricht verschlüsseln und dem Empfänger senden.\\
Aus dieser Idee kann man eine Funktion erstellen die die Verschlüsselung schematisch darstellt.\\
Der Verschlüsselungsalgorithmus f verschlüsselt den Klartext m in ein verschlüsselten Text c mit hilfe des öffentlichen Schlüssel e.
 \begin{center}
 $ c = f_e (m) $
 \end{center}
% c = f(e,m)
\paragraph{Entschlüsseln}
Der Empfänger nimmt jetzt den verschlüsselten Text und entschlüsselt ihn mit hilfe seines privaten Schlüssel und erhält den Klartext.
Auch hier kann man eine Funktion erstellen die das Entschlüsseln schematisch darstellt.
Mit der Entschlüsselungsfunktion f und dem privaten Schlüssel d wird aus dem verschlüsseltem Text c den klartext m' gewonnen.
\begin{center}
$ m' = f_d (c) $
\end{center}
Das wichtigste bei der asymmetrischen Verschlüsselung ist, dass beim entschlüsseln wieder der \textbf{gleiche} Text reproduziert wird, auch wenn zwei unterschiedliche Schlüssel verwednet wurden.\\
Damit das der Fall ist, kann man c Parametrisieren und erhält
\begin{center}
$ m' = f_d (c) = f_d (f_e (m) ) = m $\\
gekürzt\\
$ m = f_d ( f_e (m) )$
\end{center}
Im wirklichkeit sieht das Verfahren so aus. Eine Person A erstellt zwei Schlüssel. Einen privaten Schlüssel, der nur sie kennt und einen öffentlichen, den sie Person B gibt. Jetzt möchte B an A einen Text senden. Hierfür nimmt es den öffentichen Schlüssel von A und verschlüsselt den Text.
Diesen Verschlüsselten Text sendet Person B an A. 
A nimmt jetzt den verschlüsselten Text entgegen und entschlüsselt ihn mit Hilfe des privaten Schlüssel.
Da A nur den privaten Schlüssel hat, mit der sich den verschlüsselten Text entschlüsseln lässt, kann nur A den Text lesen.

Der Empfänger erstellt zwei Schlüssel. Einen öffentlichen den er verteilen kann und einen geheimen privaten Schlüssel denn er keinem zeigt. 
