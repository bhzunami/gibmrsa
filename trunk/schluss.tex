\part{Schluss}
\section{Ausblick in die Zukunft}
\section{Ergebnisse}
\newpage
\section{Abstract}
We wrote this paper for our professional baccalaureate between August and December 2011. The main subject is the RSA-algorithm, especially the mathematical proof of this asymmetric encryption method. We decided to choose this subject because we are apprentices in application engineering and have a particular interest in technical matters. We opted for the combination of English and Maths because the literature dealing with the RSA-algorithm is mostly in English, additionally it is not possible to work without a fundamental understanding of maths. The reasons for the use of latex are firstly the fact that it is a mathematical work and secondly, the extensive character of the document. Furthermore, the advantages are the correct typeset of the formulas. For this paper we worked with subversion (a version controlled environment) to guarantee the backup and work with the latest version anytime. We kept our working journal up to date by means of the google project document. \\
The main goal of this paper was to understand the functionality of the RSA-algorithm as well as a new region of maths. We divided our work in three parts to achieve the above mentioned aims. The first part consist of an overview of cryptography linked with information about the historical background. Already Julius Caesar had thought about encryption and used some primitive algorithm, today known as the Caesar encryption. If the allied forces had not broken the Enigma, the Second World War would have been totally different. %Anders verlaufen
In the second chapter, we show the acquired background knowledge and explain the functionality of the RSA-algorithm. 
On the one hand, the use of encryption and decryption is rather reasonable. On the other hand we were facing the two following difficulties. The one of creating the correct private and public key, which is more complicated due to the use of many different formulas. Furthermore, we wanted to demonstrate why the algorithm works. For this we had deal with the usage of mathematical knowledge and work out a combination to proof the algorithm's correctness. 
The last part is about the subject of security. In this context we answer the following research question: Is the RSA-algorithm secure nowadays? We did not find a final answer for this question, but with the made assumptions, %Gemachten Annahmen
we can say that it is secure in so far as the conditions of a 2048 bit key and the correct implementation to find the randomize primes are full filled. \\
When we started our research we had to acquire a lot of new knowledge in maths. Although this gained knowledge is interesting, it took some time to fully understand the meaning due to its complexity. After this face of collecting information we finally succeeded in delving into the RSA-algorithm. %Sich in ein Thema vertiefen
From day to day we came closer to understanding the principles of the algorithm and the encryption in general. The learning process was intensive and exciting. This new accomplished knowledge can be used in our future job where different encryption algorithms need to be compared or even implemented. We are now able to understand the usage of an asymmetric encryption and even found some practical examples in our daily life which need those encryptions.  
We found a lot of information about the algorithm on the internet. Although this information was properly researched we got most of our processed knowledge out of academic books. We compared these various sources and merged them together in this paper.\\
This paper was successful because we reached the main aim of the subject and the professional baccalaureate. On the one hand we increased our knowledge about primes and some specific algorithms, on the other hand we had to eliminate some derivations  %Herrleitung
because it goes beyond the scope of this paper. The challenges were to concentrate on the fundamental parts as well as to understand the difficult formulas. We are pleased that we solved these problems with a lot of effort and look forward to future developments. 
\newpage
\section{Erklärung}
Hiermit erklären wir, dass wir die Interdisziplinäre Projektarbeit ohne fremde Hilfe angefertigt haben und nur die im Quellenverzeichnis angeführten Quellen und Hilfsmittel benutzt haben.
\newpage
\section{Anhang}
% Abbildungsverzeichnis
\listoffigures
% Tabellenverzeichnis
\listoftables
%

