\part{Schluss}
\section{Abstract}
We wrote this paper for our professional baccalaureate between August and December 2011. The subject is the RSA-algorithm, especially the mathematical proof of this asymmetric encryption method. We decided to chose this subject because we are apprentice in application engineering and have a big interest in technique. We took the combination of English and maths because this algorithm wouldn't work with a fundamental understanding of maths and the literature to learn about it is in English. Because it's a mathematical work and a big document we used latex. The advantage of this tool were the correct typeset of the formulas besides the combination with a version controlled environment. For this paper we worked with subversion to guarantee the backup and to work with the latest version everywhere. \\
The main goal of the work was to understand the functionality of the RSA-algorithm. We worked in three main parts on this goal. The first part is an overview about cryptography with a history background. In the second part we show the acquired background knowledge and explain the functionality of the RSA-algorithm. The last part is about security and we reply our central question: Is the RSA-algorithm today secure? We don't have a final answer on this question, but with the made assumptions, %Gemachten Annahmen
we can say it is secure with a 2048 bit key and the right implementation to find the randomize primes. \\
The beginning of the subject were hard, because we had to acquire a lot of new knowledge in maths. Although the new knowledge were very interesting it took some time to really understand the meaning of the literature. After the initiation in the subject it succeeded to delve into the RSA-algorithm. %Sich in ein Thema vertiefen
We understood from day to day more the principles of the algorithm and the encryption in general. The learning process were intensive and interesting. 
%We had a very well organisation in our team with the efficient using of different medium like the internet and a book. 

\section{Quellen}
\subsection{Abbildungsverzeichnis}
%\listoffigures
\section{Erklärung}
Hiermit erklären wir, dass wir die Interdisziplinäre Projektarbeit ohne fremde Hilfe angefertigt haben und nur die im Quellenverzeichnis angeführten Quellen und Hilfsmittel benutzt haben.
\section{Anhang}
