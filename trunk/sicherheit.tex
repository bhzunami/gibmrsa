\part{Sicherheit}
\section{Primzahlen}
\subsection{Faktorisierungsproblem}
Die RSA-Verschlüsselung beruht auf dem Faktorisierungsproblem. In der Mathematik wurde bisher noch keine schnelle und einfache Methode gefunden um grosse Zahlen zu faktorisieren. Ein kleines Beispiel:
Die Faktorisierung von 14 in die beiden Primzahlen 2 und 7 fällt nicht sehr schwer\\
Die Faktorisierung von 771'151 ist schon wesentlich schwerer. Obwohl es die relativ kleinen Zahlen 823 und 937 sind. Jedoch wird beim RSA-Modul N eine Länge von 2048 Bit (617 Dezimalstellen) lange Zahl verwendet. Um solch eine Zahl zu faktorisieren wurden noch keine Methode gefunden um dieses Problem in vernünftiger Zeit zu lösen. \\
Es ist jedoch nicht bewiesen, das es keine bessere Methode gibt. Somit kann die Sicherheit des Algorithmus nicht Mathematisch bewiesen werden.



\section{Angriffe auf den RSA-Algorithmus}
\subsection{Timing Attack}
\subsection{Exponent Attacks}
\section{Ausblick in die Zukunft}
