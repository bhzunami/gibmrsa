%
% Maturarbeit Denis Augsburger und Nicolas Mauchle
%
% GIBM 2011

\documentclass[12pt,a4paper,german]{article}

\usepackage[left=3cm,right=2cm,bottom=3cm,includeheadfoot]{geometry}
\usepackage[pdftex]{graphicx}
\usepackage{babel}
\usepackage[utf8]{inputenc}
\usepackage{fancyhdr}
\usepackage{lastpage}

% KOPF UND FUSS ZEILEN
\pagestyle{fancy}


\fancyhf{}
%
\fancyfoot[L]{IDPA}
\fancyfoot[R]{Seite \thepage \  von  \pageref{LastPage}}

%Linie unten
\renewcommand{\footrulewidth}{0.5pt}

\begin{document}

% Title Page
\begin{titlepage}
\sffamily
\centering
\includegraphics{images/gibm_logo.png}

\vfill
{\bfseries\Huge RSA-Verschlüsselung}\\
\vfill

{\bfseries\Large Wieso ist RSA auch heute noch sicher}\\
\vfill
Verfasser\\[1ex]
Denis Augsburger\\
Nicolas Mauchle\\
\vfill
Betreuer\\[1ex]
{\large Stefan Kuster}\\
{\large Gary Collins}\\
\vfill

\raggedright
\small
Muttenz, im August 2011\\[2cm]
\begin{tabbing}
Klasse:\quad\quad\quad \=BMI 4B\\
Fächer: \> Englisch Physik \\
Schule: \> GIBM - Pratteln
\end{tabbing}
\end{titlepage}


% PLACE TEXT HERE
\tableofcontents

% EINLEITUNG
\newpage
\part{Einleitung}
\section{Vorwort}
\section{Leitfrage}

% Allgemeines über die Verschlüsselung und ihre Geschichte
\newpage
\part{Verschlüsselung}
\section{Geschichte der Verschlüsselung}
\subsection{Caesar Verschlüsselung}
\subsection{Sicherheit bei alten Verschlüsselungen}
\subsection{Vergleich zu heutigen Verschlüsslungen}

\section{Verschlüsselung Allgemein}
\subsection{Sinn und Zweck der Verschlüsselung}
\subsection{Moderne Kryptographie}
\subsubsection{Symmetrische Verschlüsselung}
\subsubsection{Asymmetrische Verschlüsselung}

%RSA-Algorithmus erklären anhand von Beispielen
\newpage
\part{Der RSA-Algorithmus}
\section{Mathematisches Verfahren}
\subsection{Eulersche Funktion}
\subsection{Modulo}
\subsection{Erweiterter euklidischer Algorithmus}
\subsection{RSA mit dem Chinesischen Restsatz}
\section{Der Schlüssel}
\subsection{Sicherer Schlüssel generieren}
\subsubsection{Zufallszahlen generieren}
\paragraph{Entropie}
\paragraph{Zahlentheoretische Funktionen}
\subsection{Schlüssel austauschen}
\section{RSA Verschlüsselung}
\subsection{Formel Verschlüsselung}
\section{RSA Entschlüsselung}
\subsection{Formel Entschlüsselung}
\section{Beispiel}
\subsection{Einfaches Zahlenbeispiel}
\subsection{Komplexeres Zahlenbeispiel}
\subsection{Beispiel an einem Text}
\section{Anwendung}
\subsection{SSH - Secure Shell}

% Was macht den RSA-Algorithmus so sicher, wie kann man ihn brechen, kann man ihn brechen
\newpage
\part{Sicherheit}
\section{Primzahlen}
\subsection{Faktorisierungsproblem}
\section{Angriffe auf den RSA-Algorithmus}
\subsection{Timing Attack}
\subsection{Exponent Attacks}
\section{Ausblick in die Zukunft}

%SCHLUSS, Quellen, Zusammenfassung usw.
\newpage
\part{Schluss}
\section{Zusammenfassung}
\section{Abstract}
\section{Quellen}
\section{Erklärung}
Hiermit erklären wir, dass wir die Intersiziplinäre Projektarbeit ohne fremde Hilfe angefertigt haben und nur die im Quellenverzeichnis angeführten Quellen und Hilfsmittel benutzt haben.
\section{Anhang}
\end{document}
