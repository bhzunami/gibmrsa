%
% Maturarbeit von Denis Augsburger und Nicolas Mauchle
%
% Betreuer S. Kuster G. Colins
%
% GIBM 2011
%
% Version 0.3
%
\documentclass[12pt,a4paper,german]{article}
%
\author{Denis Augsburger & Nicolas Mauchle}
%
\usepackage[left=3cm,right=2cm,bottom=3cm,includeheadfoot]{geometry}
\usepackage[pdftex]{graphicx}
\usepackage{babel}
\usepackage[utf8]{inputenc}
\usepackage{fancyhdr}
\usepackage{lastpage}
%Mathe Paket Denis%
\usepackage{amsmath}
\usepackage{amssymb}

%\usepackage{biblatex}
\bibliographystyle{plain}	% (uses file "plain.bst")
\usepackage{url}
% suppress boxes around links:
%\usepackage[pdfborder='0 0 0']{hyperref}

\usepackage{hyperref}
\hypersetup{
    pdfborder = {0 0 0}
}

%Damit man label setzten kann wo man will
\usepackage[all]{hypcap}
%

\usepackage[stable]{footmisc}
% KOPF UND FUSS ZEILEN
\pagestyle{fancy}
%
\fancyhf[R]{}
\fancyhf[L]{\leftmark}
%
% suppress page number in bottom center:
\cfoot{}
%
\fancyfoot[L]{}
\fancyfoot[R]{Seite \thepage \  von  \pageref{LastPage}}
%
%Linie unten
\renewcommand{\footrulewidth}{0.5pt}
%
\begin{document}
%
%-----------------------------------------------------------
% Title Page
%-----------------------------------------------------------
\begin{titlepage}
\sffamily
\centering
%LOGO 
\includegraphics{images/gibm_logo.png}
%Titel
\vfill
{\bfseries\Huge RSA-Verschlüsselung}\\
\vfill
%Fragestellung
{\bfseries\Large Wieso ist RSA auch heute noch sicher}\\
\vfill
Verfasser\\[1ex]
Denis Augsburger\\
Nicolas Mauchle\\
\vfill
Betreuer\\[1ex]
{\large Stefan Kuster}\\
{\large Gary Collins}\\
\vfill
\raggedright
\small
Muttenz, im August 2011\\[2cm]
\begin{tabbing}
Klasse:\quad\quad\quad \=BMI 4B\\
Fächer: \> Englisch Physik \\
Schule: \> GIBM - Pratteln
\end{tabbing}
\end{titlepage}

%
%-----------------------------------------------------------
\newpage
\section{Einleitung}
\subsection{Vorwort}
Bei der Themenwahl suchten wir ein Gebiet, welches uns interessiert und in unserem zukünftigen Berufsalltag begegnen wird. Wir entschieden uns für die Verschlüsselung, weil dies in der IT ein sehr wichtiges sowie aktuelles Thema ist. Da die Verschlüsselung zu allgemein definiert ist, haben wir das Thema weiter eingegrenzt und sind auf die RSA-Verschlüsselung gestossen. Diese asymmetrische Verschlüsselung wurde schon früh entwickelt, behielt bis heute ihre Bedeutung und wird in verschiedenen Anwendungen eingesetzt. Mit dieser Themenwahl konnten wir nach mehreren Anläufen auch unsere Betreuer davon überzeugen, dass wir dieses komplexe Thema bewältigen können. \\
Da dieses Thema sehr komplex und umfangreich ist, werden wir verschiedene Sätze nicht herleiten und auf die entsprechende Fachliteratur verweisen. Ebenfalls möchten wir keinen Anspruch auf Vollständigkeit erheben, da dies den Rahmen unserer Arbeit sprengen würde. \\
Bei unserer Arbeit möchten wir unsere IT-Kenntnisse effektiv und gewinnbringend einsetzen. Durch die textbasierten Dateien von Latex\footnote{Latex ist eine Software für die Benutzung vom Textsatzprogramm Tex.} können wir die gesamte Arbeit online versionieren\footnote{Mit der Versionierung können die aktuellen Dokumente abgeglichen werden}. Durch diese Arbeitsweise möchten wir Fehler und repetitive Arbeit verhindern, sowie ein sauberes, klar strukturierte Dokument erstellen. 
%
\subsection{Leitfrage}
Der RSA-Algorithmus ist im Jahr 1977 veröffentlicht worden. In der IT hat sich seit dieser Zeit viel verändert. Die Rechner sind wesentlich schneller geworden, es entstanden neue, verbesserte Algorithmen, und durch die grosse Verbreitung des Internets wurden asymmetrische Verschlüsselungen immer wichtiger. Ziel unserer Arbeit ist herauszufinden, ob das erste veröffentlichte asymmetrische Verschlüsselungsverfahren auch heute noch sicher ist. Zusätzlich möchten wir verstehen, wie und warum der Algorithmus funktioniert. Um beide Ziele abzudecken, haben wir unsere Leitfrage wie folgt definiert:\\
\begin{center}\textit{Wieso ist RSA heute noch sicher?}\end{center}

\newpage
%-----------------------------------------------------------
\tableofcontents
%
%-----------------------------------------------------------
% EINLEITUNG
%-----------------------------------------------------------
%\newpage
%\section{Einleitung}
\subsection{Vorwort}
Bei der Themenwahl suchten wir ein Gebiet, welches uns interessiert und in unserem zukünftigen Berufsalltag begegnen wird. Wir entschieden uns für die Verschlüsselung, weil dies in der IT ein sehr wichtiges sowie aktuelles Thema ist. Da die Verschlüsselung zu allgemein definiert ist, haben wir das Thema weiter eingegrenzt und sind auf die RSA-Verschlüsselung gestossen. Diese asymmetrische Verschlüsselung wurde schon früh entwickelt, behielt bis heute ihre Bedeutung und wird in verschiedenen Anwendungen eingesetzt. Mit dieser Themenwahl konnten wir nach mehreren Anläufen auch unsere Betreuer davon überzeugen, dass wir dieses komplexe Thema bewältigen können. \\
Da dieses Thema sehr komplex und umfangreich ist, werden wir verschiedene Sätze nicht herleiten und auf die entsprechende Fachliteratur verweisen. Ebenfalls möchten wir keinen Anspruch auf Vollständigkeit erheben, da dies den Rahmen unserer Arbeit sprengen würde. \\
Bei unserer Arbeit möchten wir unsere IT-Kenntnisse effektiv und gewinnbringend einsetzen. Durch die textbasierten Dateien von Latex\footnote{Latex ist eine Software für die Benutzung vom Textsatzprogramm Tex.} können wir die gesamte Arbeit online versionieren\footnote{Mit der Versionierung können die aktuellen Dokumente abgeglichen werden}. Durch diese Arbeitsweise möchten wir Fehler und repetitive Arbeit verhindern, sowie ein sauberes, klar strukturierte Dokument erstellen. 
%
\subsection{Leitfrage}
Der RSA-Algorithmus ist im Jahr 1977 veröffentlicht worden. In der IT hat sich seit dieser Zeit viel verändert. Die Rechner sind wesentlich schneller geworden, es entstanden neue, verbesserte Algorithmen, und durch die grosse Verbreitung des Internets wurden asymmetrische Verschlüsselungen immer wichtiger. Ziel unserer Arbeit ist herauszufinden, ob das erste veröffentlichte asymmetrische Verschlüsselungsverfahren auch heute noch sicher ist. Zusätzlich möchten wir verstehen, wie und warum der Algorithmus funktioniert. Um beide Ziele abzudecken, haben wir unsere Leitfrage wie folgt definiert:\\
\begin{center}\textit{Wieso ist RSA heute noch sicher?}\end{center}

%
%-----------------------------------------------------------
% Allgemeines über die Verschlüsselung und ihre Geschichte
%-----------------------------------------------------------
\newpage
\part{Verschlüsselung Allgemein}
\section{Sinn und Zweck der Verschlüsselung}


\section{Geschichte der Verschlüsselung}
\subsection{Klassische Kryptografie}
Die Kryptographie bezeichnet die Entwicklung von Methoden zur Verheimlichung von Nachrichten.
Solche Methoden wurden schon 3000 Jahre vor Christus bei den Ägyptern verwendet(Hyrogliphen). \\
Die Hebräer haben 600 vor Christus das Atbash entwickelt. Dabei wurde das Alphabet umgekehrt und entsprechend verschlüsselt.
Als kleines Beispiel\\
\begin{table}[ht]
\caption{Atbash Verschlüsselung}
\begin{tabular}{p|p|p|p|p}
  \midrule
  a & b & c & d \\
   \midrule
  z & y & x & w \\
  \bottomrule
\end{tabular}
\end{table}


Die Caesar-Verschlüsselung wurde nach Julius Caesar benannt und zur militärischen Korrespondenz im Römischen Reich gebraucht.\\
Dafür wurde damals das Alphabet um 3 Stellen nach hinten verschoben. Aus einem D wird somit ein A. Dieses Verfahren wurde im 15.-Jahrhundert mit einer
Chiffrierscheibe verbessert und ist bis heute noch als Caesar Verschlüsselung bekannt. Die Verschiebung um 13 Stellen wird auch als Rot13 bezeichnet, 
da das Alphabet aus 26 Zeichen besteht, ist die Verschlüsselung und Entschlüsselung bei Rot13 die selbe.


\subsection{Kryptographie im zweiten Weltkrieg}
Im zweiten Weltkrieg wurden erste kompliziertere mathematische Verfahren verwendet. 

\subsection{Probleme der früheren Verschlüsselungen}




\section{Moderne Kryptographie}

\subsubsection{Symmetrische Verschlüsselung}
\subsubsection{Asymmetrische Verschlüsselung}
Vor den 1970er Jahre, als es die asymmetrische Verschlüsselung noch nicht gab, wurden Schlüssel für eine symmetrische Verschlüsselung streng bewacht transportiert.
Vorallem bei Botschaften, wenn man einen neuen Schlüssel brauchte, waren dass die Personen, die den Koffer an ihren Arm mit einer Handschelle befestigten und noch zusätzlich von 5 Bodyguards beschützt in eine Panzerfesten Limusinen von einer Botschaft zur anderen fuhren.\\
Dieses Verfahren war sehr umständlich. Es kostete viel und dauerte zu lang. Wie aber sollte man einen Schlüssel sicher von einem Ort zum andern transportieren und dabei sicher sein, dass niemand einen Blick auf den Schlüssel werfen kann.\\
Anfang 1970 stellten W. Diffe und M.Hellman ein konzept vor, das die Grundidee von einer Asymmetrischen Verschlüsselung zeigte. Sie kannten das Verfahren aber nicht.
Später entwickelten James Ellis, Clifford Cockes und Malcom Willamson, alle vom Britischen Geheimdienst, an einer Asymmetrischen Verschlüsselung. Sie durften aber ihre Ergebnisse nicht veröffentlichen geschweige ein Patent anmelden.\\
Der grosse Durchbruch gelang Ronald L. \textbf{R}ivest, Adi \textbf{S}hamir und Leonard M. \textbf{A}dleman im Jahre 1977. Sie entwickelten den RSA-Algorithmus der bis heute als eines der sichersten Verfahren giltet.\\[2ex]
%
Bei einem assymetrischen Verfahren kann jede Sender einem Empfänger eine verschlüsselte Nachricht senden, die nur der Empfänger entschlüsseln kann. Das Funktioniert dank einem öffentlichen Schlüssel \textbf{Public-Key} und eine privaten Schlüssel \textbf{Private-Key}. \\
Deshalb spricht man auch von einer  \textbf{Public Key-Kryptographie}.
Der Empfänger erstellt zwei Schlüssel. Einen öffentlichen den er verteilen kann und einen geheimen privaten Schlüssel denn er keinem zeigt. 


\section{Moderne Kryptographie}
Dieses Kapitel wurde mit Hilfe des Buches \textit{Moderne Verfahren der Kryptographie} erarbeitet\\[2ex]
Die Moderne Kryptographie unterscheidet sich zur \textit{klassischen} Kryptographie vor allem in der Technik. Was früher mechanisch erstellt wurde, wird in der modernen Kryptographie digital gemacht. Es werden keine langen Buchstabenkolonen analysiert und keine Chiffriermaschienen benötigt, um einen Geheimtext zu erstellen.\\
Auch die immer besser bzw. schnelleren Computer machten die klassischen Kryptographie sehr unsicher, da sie durch Brute-Force\footnote{Brute-Force nennt man ein Vorgehen, wenn man ein Passwort durch ausprobieren knackt.} schnell gelöst werden konnten.\\
Man brauchte also sichere Verschlüssselungs-Methoden, die durch ausprobieren, nicht geknackt werden konnten.\\
Wichtig für die moderne Kryptographie sind die Jahreszahlen 1976 und 1985.\\
1976 wurde das Problem, dass immer beide Seiten einen \textbf{gemeinsamen} Schlüssel benötigen gelöst. Das ist die Geburtstunde des RSA-Algorithmus.\\
1985 entwickelten drei Mathematiker/Innen das \textbf{Zero-Knowlede-Verfahren}. Mit diesem Verfahren ist es möglich, dass ich mit jemanden über ein Geheimniss kommuniziere, ohne dass dieser etwas über mein Geheimnis erfährt. Das Zero-Knowlede-Verfahren baut auf die Public Key-Kryptographie auf. Der wesentliche Unterschied liegt im Protokol.
%
\subsection{Symmetrische Verschlüsselung}
Grundsätzlich braucht es für eine einfache symmetrische Verschlüsselung einen Schlüssel, Text zum verschlüsseln und eine Verschlüsselungsfunktion.
Die Verschlüsselungsfunktion verschlüsselt mit Hilfe des Schlüssel die Nachricht.
In einer Formel ausgedrückt heisst das
 \begin{center}
$c = f ( k, m )$
 \end{center}
wobei \textit{c} für den verschlüsselten Text steht, \textit{f} für die Verschlüsselungsfunktion, \textit{k} für den Schlüssel und \textit{m} für den Text den man verschlüsseln möchte.\\

Dieser Verschlüsselte Text gibt er jetzt dem Empfänger. Hat der Empfänger keinen Schlüssel k, um die Nachricht zu entschlüsseln, kann der die Nachricht nicht lesen. Somit braucht der Empfänger den \textbf{selben} Schlüssel k, die verschlüsselte Nachricht c und eine umkehrfunktion von f. Man möchte ja am Schluss den gleichen Text wieder haben.\\
Daraus ensteht die Formel 
 \begin{center}
$m = f^*(k,c)$
 \end{center}
Man geht davon aus, das die Funktion der Ver- und Entschlüsslung bekannt sind. Das die Funktionen bekannt sind, ist nicht ein sicherheitsproblem.\\
Nach dem Kerckoffs' Prinzip geht der Grundsatz hervor, dass die Sicherheit eines Verschlüsslungsverfahren nicht auf der Geheimhaltung des Algorithmus beruht, sondern auf die geheimhaltung des Schlüssels.
Denn ein Algorithmus kann durch Reverse-Engineering\footnote{Unter dem Begriff Reverse-Engineering versteht man, dass aus einem fertigen Produkt durch Untersuchen der Strukturen und der Verhaltensweise den Source-Code heraus zu finden.} rekonstruiert werden.\\
Auch hat die Geschichte gezeigt dass bis heute \textbf{jeder} Verschlüsselungsalgorithmus gebrochen werden konnte. (siehe Gemstar Development, RC4).\\[2ex]
%
Somit muss die Verschlüsselungsfunktion so komplex sein, dass man nur mit dem Schlüssel das richtige Ergebnis erhaltet.
%
\subsection{Asymmetrische Verschlüsselung}
Vor den 1970er Jahre, als es die asymmetrische Verschlüsselung noch nicht gab, wurden Schlüssel für eine symmetrische Verschlüsselung unter grossem Schutz transportiert.
% Besser ausdeutschen
% --->
Vorallem bei Botschaften, wenn man einen neuen Schlüssel brauchte, waren dass die Personen, die den Koffer an ihren Arm mit einer Handschelle befestigten und noch zusätzlich von 5 Bodyguards beschützt in einer panzerfesten Limusinen von einer Botschaft zur anderen fuhren.\\
Dieses Verfahren war sehr umständlich. Es kostete viel und dauerte zu lang. Wie aber sollte man einen Schlüssel sicher von einem Ort zum andern transportieren und dabei sicher sein, dass niemand einen Blick auf den Schlüssel werfen kann.\\
% <---
Anfang 1976 stellten W. Diffe und M.Hellman ein Konzept vor, dass die Grundidee von einer Asymmetrischen Verschlüsselung zeigte. Sie kannten das Verfahren aber nicht exakt sondern stellten theoretische Ansätze vor.\\ % Stellten das Prinzip vor
Später entwickelten James Ellis, Clifford Cockes und Malcom Willamson, alle vom Britischen Geheimdienst, ein Asymmetrisches Verschlüsselungsverfahren. Sie durften aber ihre Ergebnisse nicht veröffentlichen geschweige ein Patent anmelden. 
Der Erfolg gelang Ronald L. \textbf{R}ivest, Adi \textbf{S}hamir und Leonard M. \textbf{A}dleman im Jahre 1977. Sie entwickelten den RSA-Algorithmus der bis heute als eines der sichersten asymmetrischen Verfahren giltet.\\[2ex]
%
Bei einem assymetrischen Verfahren kann jeder Sender einem Empfänger eine verschlüsselte Nachricht senden, die nur der Empfänger entschlüsseln kann.\\
%
Das Funktioniert dank einem öffentlichen Schlüssel \textbf{Public-Key} und einem privaten Schlüssel \textbf{Private-Key}. \\
Deshalb spricht man auch von einer  \textbf{Public Key-Kryptographie}.
%
% Bei der Formel für die verschlüsselung ist f_e und m verantwortlich für c das heisst es sind beide wichtig!!
\paragraph{Verschlüsseln}
Der Sender braucht den öffentlichen Schlüssel von dem Empfänger. Mit diesem Schlüssel und dem Verschlüsslungsverfahren, kann jetzt der Sender eine Nachricht verschlüsseln und dem Empfänger senden.\\
Aus dieser Idee kann man eine Funktion erstellen die die Verschlüsselung schematisch darstellt.\\
Der Verschlüsselungsalgorithmus f verschlüsselt den Klartext m in ein verschlüsselten Text c mit hilfe des öffentlichen Schlüssel e.
 \begin{center}
 $ c = f_e (m) $
 \end{center}
% c = f(e,m)
\paragraph{Entschlüsseln}
Der Empfänger nimmt jetzt den verschlüsselten Text und entschlüsselt ihn mit hilfe seines privaten Schlüssel und erhält den Klartext.
Auch hier kann man eine Funktion erstellen die das Entschlüsseln schematisch darstellt.\\
Mit der Entschlüsselungsfunktion f und dem privaten Schlüssel d wird aus dem verschlüsseltem Text c den klartext m' gewonnen.
\begin{center}
$ m' = f_d (c) $
\end{center}
Das wichtigste bei der asymmetrischen Verschlüsselung ist, dass beim entschlüsseln wieder der \textbf{gleiche} Text reproduziert wird, auch wenn zwei unterschiedliche Schlüssel verwendet wurden.\\
Damit das der Fall ist, kann man c Parametrisieren und erhält
\begin{center}
$ m' = f_d (c) = f_d (f_e (m) ) = m $\\
\end{center}
gekürzt\\
\begin{center}
$ m = f_d ( f_e (m) )$
\end{center}
\paragraph{Beispiel \footnote{Dieses Beispiel zeigt einen schematischen Ablauf. Es geht nicht auf das Verschlüsselungsverfahren ein.} }
Herr Müller möchte gern Herrn Spinnler seine sensiblen Daten übergeben. Dafür benötigt er den öffentlichen Schlüssel von Herr Spinnler.\\
Jetzt verschlüsselt Herr Müller mit dem öffentlichen Schlüssel seine Daten, die er an Herr Spinnler senden möchte. Das verschlüsselte Paket sendet Herr Müller an Herr Spinnler. Dieser wiederum entschlüsselt sein Paket mit seinem geheimen gut aufbewarten privaten Schlüssel. Somit kann er die Daten lesen.

%Im wirklichkeit sieht das Verfahren so aus. Eine Person A erstellt zwei Schlüssel. Einen privaten Schlüssel, der nur sie kennt und einen öffentlichen, d%en sie Person B gibt. Jetzt möchte B an A einen Text senden. Hierfür nimmt es den öffentichen Schlüssel von A und verschlüsselt den Text.
%Diesen Verschlüsselten Text sendet Person B an A. 
%A nimmt jetzt den verschlüsselten Text entgegen und entschlüsselt ihn mit Hilfe des privaten Schlüssel.
%Da A nur den privaten Schlüssel hat, mit der sich den verschlüsselten Text entschlüsseln lässt, kann nur A den Text lesen.
%Der Empfänger erstellt zwei Schlüssel. Einen öffentlichen den er verteilen kann und einen geheimen privaten Schlüssel denn er keinem zeigt. 

%
%-----------------------------------------------------------
%RSA-Algorithmus erklären anhand von Beispielen
%-----------------------------------------------------------
\newpage
\part{Der RSA-Algorithmus}
%
\section{Mathematisches Verfahren}
Der RSA-Algorithmus baut grundsätzlich auf Primzahlen, Modulo, Eulersche-Funktion, Satz von Euler und den erweiterten euklidischen Algorithmus auf. Verstehn man diese mathematischen Verfahren nicht, kann man auch nicht den RSA-Algorithmus verstehen und verwenden. \\
Im nächsten Abschnitt wollen wir die einzelnen mathematischen Verfahren kurz in ihren Funktionen erläutern.  
\subsection{Modulo}
Modulo oder auch \textit{Division mit Rest} gibt den Rest von einer ganzzahligen Division an. Modulo erklärt sich am einfachsten an einem Beispiel.\\
$ 21 / 5 = 4.2 $ \\
4.2 ist keine ganze Zahl. $ 4 * 5 = 20 $. \\
Bis auf 21 fehlt genau 1. Somit ist $ 21 \bmod 5 = 1 $
\paragraph{Kongruent Modulo}
Kongruent modulo heisst nichts anders, als dass zwei Zahlen modulo eine Zahl den selben Rest haben. Man sagt auch, dass sie in der gleichen Restklasse sind.\\
Als Beispiel nehmen wir die Zahlen 9 und 7 die modulo 2 gerechnet werden.\\
\begin{center}
$ 9 \bmod(2) = 1 $ \\
$ 7 \bmod(2) = 1 $ \\
%$ 9 \bmod(2) $ ist kongruent zu $ 7 \bmod(2) $ \\
$ 9 \bmod(2) \equiv 7 \bmod(2) $ \\
\end{center}
Denn beide ergeben als Resultat 1.
\subsection{Eulersche Funktion}
Die Eulersche Funktion sagt aus, wieviele teilfremde \footnote{Zahlen sind sich teilfremd, wenn ihr grösster gemeinsamer Teiler 1 ist.} natürliche Zahlen zu einer Zahl es gibt.\\
Für den RSA-Algorithmus verwenden wir die Eulersche Funktion nur bei Primzahlen. Deshalb schauen wir uns nur die spezial Formel bei Primzahlen an.\\
Primzahlen sind nur durch sich selber und 1 dividirbar und ihr grösster gemeinsamer Teiler ist immer 1. Somit ist die Eulersche Funktion einer Primzahl p immer p - 1.
\begin{center}
$ \varphi(p) = p - 1 $
\end{center}
Wenn wir jetzt also zwei verschiedene Primzahlen haben, gilt
\begin{center}
$ \varphi(pq) = (p - 1) * (q - 1) $
\end{center}
Der Beweis dafür liefert folgende Überlugungen. Es gibt insgesamt $ p * q -1 $ ganze Zahlen die kleiner sind als $ p * q $. Es ist bei Primzahlen einfacher die \textbf{nicht teilfremden} Zahlen zu zählen und diese dann von allen möglichen Zahlen abzuziehen als allte teilfremden Zahlen zu zählen. Nicht teilfremd zu p sind $ (q - 1) * p $ und für q gilt $ (p - 1) * q $\\
Als Formel geschrieben
\begin{center}
$ \varphi(pq) = p * q -1 - (p - 1) - (q - 1) $ \\
$ \varphi(pq) = p * q -1 - p + 1 - q + 1 $ \\
$ \varphi(pq) = p * q -q - p + 1 $ \\
$ \varphi(pq) = (p -1) * (q - 1) $ 
\end{center}
\subsection{Satz von Euler}
Der Satz von Euler sagt über zwei teilfremde Zahlen folgende Gleichung aus
\begin{center}
$ m^{\varphi(n)} \bmod(n) = 1 $
\end{center}
Nehem wir zwei Primzahlen p = 3 und q = 5. Und als m wählen wir 4 und setzten das ganze in unsere Formel ein.\\
\begin{center}
$ 4^{\varphi(3 * 5)} \bmod(3 * 5) = 1 $ \\
$ 4^8 \bmod(15) = 1 $ \\
\end{center}
Wenn wir jetzt noch eine natürliche Zahl k nehmen, gilt für diese folgende Aussage \\
\begin{center}
$ m^{1 + k * \varphi(n)} \bmod(n)  = m * 1$ \\
%$ m*m^{k * \varphi(n)} \bmod(n)  = m$ \\
\end{center}
Hat man zwei verschiedene Primzahlen p und q und eine natürliche Zahl m die kleiner ist als p * q. Dann gilt für jede natürliche Zahl k \\
\begin{center}
\boxed{m^{k * (p - 1) * (q - 1) +1} \bmod(p * q) = m}
\end{center}
Dieser Satz wird in der Mathematik auch als \textit{kleiner Satz von Fermat} genannt.
\subsection{Euklidischer Algorithmus}
Der Euklidischer Algorithmus errechnet den grössten gemeinsamen Teiler (ggT) von zwei natürlichen Zahlen. Um den ggT herauszufinden gibt es eine ältere das Subtraktions-Verfahren und eine modernere Methode das Modulo-Verfahren. Da das ältere Verfahren nicht so effizient ist, konzentrieren wir uns auf das moderne Verfahren.\\
Die Formel für das modernere Verfahren $a = q * p + r$ ist eine modulo Rechnung. Der Rest r wird zum neuen b und b wird zum neuen a. \\
Zu veranschaulichung nehmen wir zwei Zahlen a = 5984 und b = 302 und setzen diese in usere Gleichung ein.
\begin{center}
$ 5984 = 19 * 302 + 146 $\\
$ 302 = 2 * 146 + 10 $\\
$ 146 = 14 * 10 + 6 $ \\
$ 10 = 1 * 6 + 4 $ \\
$ 6 = 1 * 4 + 2 $ \\
$ 4 = 2 * 2 + 0 $\\
\end{center}
Der grösstegGemeinsame Teiler von 5984 und 19 ist somit 2. \\[2ex]
Um dieses Verfahren zu beweisen gehen wir davon aus das a, b, r und q alles natürliche Zahlen sind. Da wir im zweiten Schritt den grössten gemeinsamen Teiler von b und r errechnet haben, kann man folgende Gleichung aufstellen.
\begin{center}
$ ggT(a,b) = ggT(b,r) $\\
\end{center}
Wenn also eine Zahl a und b teilt, muss diese Zahl auch r teilen. Denn sonst würde die Gleichung oben nicht aufgehen. Umgekehrt muss eine Zahl die b und r teilt auch a teilen. Gehen wir zurück zu unserem Beispiel. Da diese Gleichung stimmt, stimmt auch unseres Verfahren.
\begin{center}
$ ggT(5984,302) = ggT(302,146) = ggT(146,10) = ggT(10,6) = ggT(6,4) = 2 $
\end{center}
\subsubsection{Erweiterter Euklidischer Algorithmus}
%




%
\newpage
\section{RSA Ver- und Entschlüsselung}
Der Name RSA setzt sich aus den Anfangsbuchstaben der Nachnamen der Entwickler zusammen. Das RSA Verfahren wird auch heute noch oft eingesetzt. Das Verfahren gilt bei einer bestimmten Schlüssellänge als sehr sicher. Es gibt verschiedenste Angriffsmöglichkeiten, diese führen jedoch nicht in einem vernünftigen Zeitraum zu Ergebnissen.\\
Das RSA Verfahren ist verwandt mit dem Rabin-Kryptosystem, dass auch auf Primzahlen beruht. Das Verfahren wird zur Signierung und zur Verschlüsselung verwendet. %Verweis%

\subsection{Der Schlüssel}
Dieses Kapitel wurde mit Hilfe des Buches \textit{RSA and Public-Key Cryptography} erarbeitet.\\[2ex]
%
Für eine sichere Verschlüsslung braucht es auf einen guten Schlüssel. Denn nur mit diesem Schlüssel, kann man die verschlüsselte Nachricht entschlüsseln. Der Schlüssel spielt also eine zentrale und wichtige Rolle. Dazu muss er so komplex sein, dass man ihn nicht erraten kann, oder durch Ausprobieren entdeckt.\\
Darum ist es auch sehr wichtig, dass man Zufallszahlen nimmt, die man nicht vorhersagen kann, beziehungsweise im Nachhinein berechnen kann.
%
\subsubsection{Zufallszahlen generieren}
In der Informatik sind Zufallszahlen eine sehr komplexe Angelegenheit. Um solche Zufallszahlen zu generieren, haben grössere Firmen einen Zufallszahlengeneratoren (eng. \textit{random number generator} RNG) entwickelt. Diese messen den radioaktiven Zerfall oder beobachteten die atmosphärischen Bedingungen in der Umgebung. Die Zahlen, die der RNG auswertet, nennt man \textbf{Seed-Zahlen}. Mit den gewonnen Seed-Zahlen, kann man jetzt einen Algorithums füttern, der eine Zahl oder eine Zahlenfolge zurückgibt. Hat man eine solche Zufallszahl, muss noch überprüft werden, ob es eine Primzahl ist. Zur überprüfung gibt es extra Verfahren, die wir hier nicht weiter erleutern werden. Mehr Informationen dazu im Buch \textit{RSA and Public-Key Cryptography - Kapitel 4}.\\
%
Für den normalen Benutzer sind solche Apparate viel zu aufwendig und zu teuer. Deshalb entwickelten Computerhersteller einen Pseudozufallszahlengenerator (eng. \textit{pseudo random number generator} PRNG). Dieser PRNG erstellt die Seed-Zahlen zum Beispiel aus der momentanen Position der Maus, momentane CPU Auslastung oder andere nicht vorhersehbare Elemente.\\
Ob die Zufallszahlen geeignet sind oder nicht ist schwierig zu sagen. Denn wie will man eine Folge von Zufallszahlen beurteilen?\\
Das Einzige, was man errechnen kann, ist die Entopie\footnote{Die Entropie beschreibt das Mass der Unordnung.}. Je höher die Entropie, desto unwarscheinlicher ist es, dass diese Zahlenfolge ein zweites Mal vorkommt.
%Buch RSA and Public-Key Seite 61 
\subsubsection{RSA-Schlüssel generieren}
Für eine RSA-Verschlüsselung brauchen wir zwei Schlüssel, die voneinander abhängig sind. Einen private Key und einen public Key. Im nächsten Abschnitt werden wir erläutern, wie man diese zwei Schlüsser errechnen kann.
%
\paragraph{Public-Key erstellen}\label{sec:public_key}
Als erstes brauchen wir zwei Primzahlen p und q die etwa gleich lang sind. Je grösser desto besser und sicherer. Wie man solche Primzahlen erstellt möchten wir hier nicht erläutern. Es gibt Methoden, bei denen man mit grosser Wahrscheinlichkeit eine Primzahl generieren kann. Man muss die Zahl nachher immer überprüfen.\\
Sind beide Primzahlen gefunden, errechnen wir uns N auch als RSA-Modul genannt.
%
\begin{equation}
  N = p \cdot q
  \label{eqn:rsa_modul}
\end{equation}
%
Als nächstes rechnen wir mit der Eulerschen Funktion [\ref{eqn:eulersche_func}] $\varphi$(N) von p und q aus.\\
Zu der Zahl $\varphi$(N) nehmen wir eine weitere zufällige, \textbf{teilfremde} Zahl e die den RSA verschlüsselungs Exponent bildet.\\
So nun haben wir unseren öffentlichen Schlüssel mit den Zahlen N und e.
\paragraph{Private-Key erstellen}
Wir haben vorher bei der Erstellung des public Keys die Zahlen N und e errechnet. Als nächstes wird aus diesen beiden Zahlen d errechent. d wird als enschlüsselungs Exponent im pirvate Key benötigt.\\
Mit dem erweiterten Euklidischen Algorithmus [\ref{eqn:erw_euklid_algo}] wird d berechnet so dass die Formel
%
% TODO: Auf modulare inverse hinweisen!!!!!
\begin{flalign*}
  e * d \bmod(\varphi(n)) = 1\\
  e * d \equiv 1 (\bmod \varphi(N) )
\end{flalign*}
%
stimmt. \\
Nachdem d ausgerechnet wurde, ist man im Besitz seines privaten Schlüssel mit den Zahlen d und N.
%\subsection{Schlüssel austauschen} Sprengt den Rahmen!!!!

%
\subsection{Formel Verschlüsselung}
Wenn die Schlüssel erstellt wurden, kann ein Text einfach verschlüsselt werden.\\
Dazu dient folgende Formel:
%
\begin{equation}
  c \equiv m^e  \bmod N
  \label{eqn:rsa_encription}
\end{equation}
%
%\subsection{RSA Entschlüsselung}
%Die RSA Entschlüsselung ist ohne den Geheimschlüssel zu wissen nicht möglich. Da die Sicherheit von diesem Geheimschlüssel abhängt, gilt es diesen möglichst lang zu wählen und geheim zu halten. Aktuell gilt eine Schlüssellänge von 2048 Bit als sicher. Als Dezimalzahl ausgedrückt, ist dies eine Zahl von $ 3,2 \cdot 10^{616} $
%Die Angriffe auf die RSA Verschlüsselung zielen darauf ab, den Geheimschlüssel zu ermitteln. Das Problem dabei ist die Primfaktoren Zerlegung von grossen Zahlen. Mehr dazu im Kapitel Angriffe %Verweis%

\subsection{Formel Entschlüsselung}
Die Entschlüsselung der Nachricht wird mit folgender Formel realisiert:
%
\begin{equation}
  m \equiv c^d \bmod N
  \label{eqn:rsa_decription}
\end{equation}
%
Der Klartext m hängt in diesem Fall vom verschlüsselten Text C, dem privaten Schlüssel d und dem RSA-Modul N ab. 
%
%
\newpage
%******************************************************************************
% Mathematischer Beweis
%******************************************************************************
\section{Mathematisches Beweis der Funktionsweise}
Bisher nahmen wir an, dass der RSA-Algorithmus korrekt arbeitet. Diese Annahme möchten wir nun beweisen. Der Algorithmus arbeitet korrekt, wenn die Verschlüsselung und nachherige Entschlüsselung wieder das gleiche ergibt. Dies darf nur mit dem zugehörigen privaten und öffentlichen Schlüssel möglich sein.

\subsection{Verschlüsselung und Entschlüsselung gleichsetzen}
Für den Beweis benötigen wir die ursprünglichen Formeln zur Verschlüsselung und Entschlüsselung. Diese lauten:
\begin{flalign*}
  C & = m^e mod N \\
  m & = C^d mod N
\end{flalign*}
Da in beiden Formeln die Ursprungsnachricht m und die verschlüsselte Nachricht C vorkommen, können wir diese Formeln gleichsetzen. Dazu lösen wir die Entschlüsselungsformel nach C auf:
\begin{flalign*}
  m &= C^d mod N \\
  m &= C^d - k * N \\
  m + k \cdot N & = C^d \\
  \sqrt[d]{m + k \cdot N} & = C
\end{flalign*}
Modulo ergibt jeweils den Rest einer ganzzahligen Division. Durch Umformung kann diese jeweils als $ k \cdot N $ dargestellt werden, wobei k eine ganze Zahl sein muss.\\
Nun können wir C anders ausdrücken und in der Verschlüsselungsformel einsetzen.
%Wir setzen nun die beiden Formeln gleich, so das nur noch die Ursprungsnachricht m in der Gleichung vorhanden ist:
\begin{flalign*}
  m^e mod N & = \sqrt[d]{m + k \cdot N}\\
  m^{e \cdot d} mod N & = m + k \cdot N\\
  m^{e \cdot d} mod N & = m 
  %{m^e}^d mod N & = m \\
  %{m^d}^e mod N & = m
\end{flalign*}
Die Formel $ m^{e \cdot d} mod N = m + k \cdot N $ kann gekürzt werden, da k in jedem Fall 0 sein muss. Dies liegt daran, dass wir auf der linken Seite mit Modulo N den Rest ausgeben. Der Rest kann 1 bis N-1 gross sein. Da m einen Wert hat, muss k in diesem Fall 0 sein. \\
Wir möchten beweisen, dass die Verschlüsselung (hoch e) und nachherige Entschlüsselung (hoch d) wieder die ursprüngliche Nachricht ergibt:
\begin{equation*}
 m^{e \cdot d} mod N = m 
\end{equation*}
%
\subsection{Grundlagen zur Erklärung}
Für den Beweis benötigen wir vorherige Kenntnisse und bestimmte Sätze. Diese möchten wir hier nochmals kurz in Erinnerung rufen.\\ 
Das RSA-Modul N wird aus den ausgewählten Primzahlen p und q erstellt.
\begin{equation*}
  N = p \cdot q
\end{equation*}

e wurde teilerfremd zu $ \varphi(n) $ gewählt. 
$ \varphi(n) = (p-1) \cdot (q-1) $

Zusätzlich wurde d so gewählt, dass folgendes zählt:
\begin{equation*}
 e \cdot d + k \cdot \varphi(N) = 1 = ggT(e,\varphi(N))
\end{equation*}

Für den Beweis müssen wir ausschweifen auf den Satz von Euler Fermat. Dieser bildet die Grundlage zur RSA-Verschlüsselung. Er lautet wie folgt:
\begin{equation*}
	a^{\varphi(n)} \equiv 1\,(\mathrm{mod}\,n)
\end{equation*}
Da wir mit Primzahlen arbeiten, kann dieser Satz auch anders ausgedrückt werden. $ \varphi(n) $ gibt alle teilerfremden Zahlen zu n an. Da n durch zwei Primzahlen gebildet wurde, gibt es $ (p-1) \cdot (q-1) $ teilerfremde Zahlen. 
%
\subsection{Beweis der Funktionsweise}
Die Gleichsetzung der Entschlüsselung und Verschlüsselung dient uns als Grundlage des Beweises:
\begin{equation*}   
 m^{e \cdot d} mod N = m
\end{equation*}
%
Wir könnnen durch die modulare Inverse $ e \cdot d $ anders ausdrücken, siehe [\ref{eqn:mod_inverse}]. Diese lösen wir nach $ e \cdot d $ auf:\\
\begin{flalign*}
 e \cdot d &= 1 \bmod{\varphi(N)}  \\
 % e \cdot d + k \cdot \varphi(N) &= 1  \\
 e \cdot d &= 1 + k \cdot \varphi(N) \\
 e \cdot d &= k \cdot \varphi(N) + 1
\end{flalign*}
Wenn die Formel Modulo beinhaltet, kann diese jeweils auch mit k Mal den Rest ausgedrückt werden. \\
%
Nun ersetzen wir in unserer Ausgangslage $ e \cdot d $ durch $ k \cdot \varphi(n)+1 $ und erhalten die Gleichung wie in Formel [\ref{eqn:kleiner_satz_fermat}].
\begin{flalign*}
 m^{e \cdot d} mod N = m
 m^{ k \cdot \varphi(N) + 1} \bmod(N) = m  \\
 m^{k \cdot \varphi(N)} \cdot m \bmod(N) = m  \\
 { m^{ \varphi(N) }} ^k \cdot m \bmod(N) = m \\
 { m^{ \varphi((p-1)\cdot(q-1)) }} ^k \cdot m \bmod((p-1)\cdot(q-1)) = m
\end{flalign*}
%
%Durch den kleinen Satz von Fermat wissen wir dass $ \varphi(N) $ 1 sein muss. 
Mit dem kleinen fermatschen Satz können wir nun nach $ m^{\varphi(N)} $ auflösen.
\begin{flalign*}
  m^{(p-1)} &= 1 \bmod p \\
  m^{(p-1) \cdot (q-1)} &= 1 \bmod p \cdot q \\
  m^{\varphi(N)} & = 1 \bmod N 
\end{flalign*}
1. Die Nachricht hoch die Anzahl der Teilerfremden Zahlen bei einer Primzahl ergibt 1 Modulo die Zahl \\
2. Bei der RSA-Verschlüsselung haben wir die Kombination aus zwei Primzahlen.
%nicht hoch eine Primzahl gerechnet, sondern hoch $ (p-1) \cdot (q-1) $. 
Daher können wir den kleinen fermatschen Satz so darstellen.\\
3. $ p \cdot q $ ist das RSA-Modul N. $ (p-1) \cdot (q-1) $ sind die Teilerfremden Zahlen von N, da es sich bei p und q um Primzahlen handelt. \\ %Verweis phi(N) und Primzahlen%
%
Daraus resultiert das $ 1 \bmod N $ das gleiche ist wie $ m^\varphi(N) $ und entsprechend eingesetzt werden kann. Danach lösen wir die Formel auf. 
\begin{flalign*}
 { m^{ \varphi(N) }} ^k \cdot m \bmod(N) &= m  \\
 {1 \bmod N }^k \cdot m \bmod(N) &= m  \\
 1^k \cdot m \bmod(N) &= m \\
 1 \cdot m \bmod(N) &= m \\
 m + k \cdot N &= m \\
 m + 0 \cdot N &= m \\
 m &= m 
\end{flalign*}
Wir wissen das $ k = 0 $ sein muss, da m immer kleiner als N gewählt wird. Falls m grösser gleich N ist, würde der RSA-Algorithmus nicht funktionieren. In diesem Fall wird die Nachricht aufgeteilt und später wieder zusammengesetzt.\\
Schlussendlich stellen wir fest, dass m = m ist und beweisen somit die korrekte Funktionsweise des RSA-Algorithmus. Die Verschlüsselung und darauffolgende Entschlüsselung ergibt die Ursprungsnachricht.\\
Die Entschlüsselung der Nachricht funktioniert nur mit dem zugehörigen privaten Schlüssel, da mit dem euklidischen Algorithmus d so bestimmt wurde, das folgendes zählt: 
\begin{flalign*}
 e \cdot d + k \cdot \varphi(N) &= 1	
\end{flalign*}
Mit der modularen Inverse wurde d so bestimmt, das keine negative Zahl entsteht. Aus der vorherigen Formel ist ersichtlich, dass nur ein ganz bestimmtes d zu einem e und N gehört. Jeder andere Entschlüsselungexponent würde die Formel nicht korrekt erfüllen und zu einem anderen Ergebnis führen. 
%
%
%***************************************
% BEISPIEL
%***************************************
\section{Beispiel}

\subsection{Einfaches Zahlenbeispiel}
Ein einfaches Beispiel für den RSA-Algorithmus
\subsubsection{Schlüssel-Paar erstellen}
Ein einfaches Zahlenbeispiel. Wir suchen uns zwei kleine Primzahlen.
%
\begin{flalign*}
  p = 13 \\
  q = 19
\end{flalign*}
%
Wir berechnen nun das RSA-Modul N [\ref{eqn:rsa_modul}] und $\varphi(N) $ [\ref{eqn:eulersche_func}] aus p und q.
\begin{equation*}
  \tag{RSA-Modul}
  247 = 13 \cdot 19
\end{equation*}
%
\begin{equation*}
  \tag{$\varphi(N)$}
  216 = (13 - 1) \cdot (19 - 1)
\end{equation*}
%
Zu $ \varphi(N) $ suchen wir uns eine zweite Zahl e, die teilerfremd zu $ \varphi(N) $ ist, sprich den $ggT(\varphi(N),e) = 1$. Am Besten man verwendet eine weitere Primzahl. Wir nehmen für e die Primzahl 23.
%
\begin{equation*}
    e = 23
\end{equation*}
%
Der öffentliche Schlüssel ist nun berechnet [siehe Section \ref{sec:public_key}]\\
Um den entschlüsselungs Exponenten zu errechnen, müssen wir zuerst den ggT [\ref{eqn:euklidischer_algo}] aus 23 und 216 ausrechnen.
\begin{flalign*}
  216 & = 9 * 23 + 9 \\
  23 & = 2 *  9 + 5 \\
  9 & = 1 *  5 + 4 \\
  5 & = 1 *  4 + 1
\end{flalign*}
%
Jetzt weneden wir den Erweiterter Euklidischer Algorithmus [\ref{eqn:erw_euklid_algo}] an.
\begin{flalign*}
  1 &= 5 - 1 \cdot 4 = 5 - 1 \cdot(9 - 1 \cdot 5) = 5 - 1 \cdot 9 + 1 \cdot 5 = 2 \cdot 5 - 1 \cdot 9\\
  1 &= 2 \cdot 5 - 1 \cdot 9 = 2 \cdot (23 - 2 \cdot 9) - 1 \cdot 9 = 2 \cdot 23 - 4 \cdot 9 - 1 \cdot 9 = 2 \cdot 23 - 5 \cdot 9\\
  1 &= 2 \cdot 23 - 5 \cdot 9 = 2 \cdot 23 - 5 \cdot (216 - 9 \cdot 23) = 2 \cdot 23 - 5 \cdot 216 + 45 \cdot 23 = \textbf{47} \cdot 23 \textbf{- 5} \cdot 216
\end{flalign*}
Wenden die Modulare Inverse auf die letzte Gleichun an und erhalten $d = 47$.
%
Jetzt haben wir alle nötigen Zahlen die wir für eine Verschlüsselung sowie Entschlüsselung benötigen.
\subsubsection{Verschlüsseln}
Da man Text nicht verschlüsseln kann, braucht man für den Text Zahlen. Wir nehmen jetzt zur Vereinfachung eine Zahl.
\begin{flalign*}
  m &= 15 \\
  15^{23} \bmod 247 &= 59
\end{flalign*}
Die Zahl 15 verschlüsselt mit unserem öffentlichen Schlüssel (23,247) ergibt die Zahl 59.
%
\subsubsection{Entschlüsseln}
Um zu überprüfen ob unserer privater Schlüssel auch wirklich funktioniert, entschlüsseln wir die Zahl 59 mit unserem privaten Schlüssel.\\
\begin{flalign*}
  c &= 59  \\
  59^{47} \bmod 247 &= 15
\end{flalign*}
Somit sehen wir, dass unser kleines Beispiel funktioniert hat.
%
\subsection{Beispiel an einem Text}
Damit wir einen Text verschlüsseln können, brauchen wir grössere Zahlen. Für das nächste Beispiel sind folgende Zahlen gegeben.
\begin{flalign*}
  p &= 101\\
  q &= 349\\
  N &= 35'249\\
  \varphi(N) &= 34'800\\
  e &= 509\\
  d &= 7'589
\end{flalign*}
Wir müssen nur noch ein Verfahren wählen, wie wir einen Buchstaben in eine Zahl umwandeln. Für das gibt es \textit{ASCII} Tabellen, die für jedes Zeichen, eine Zahl darstellen.\\
Wir würden gerne das Wort \textit{GIBM MUTTENZ} verschlüsseln. Als erstes müssen wir unseren Text in Zahlen umwandeln. Dazu verwenden wir die Dezimalschreibweise.\\
\textit{GIBM MUTTENZ} würde in dieser Schreibweise \textit{717366773277858484697890} heissen. Da $m < N$ sein muss, müssen wir unsere Zahl in Blöcke aufteillen. Wir machen uns immer 4er Blöcke und verschlüsseln sie.
\begin{flalign*}
  7173^509 \bmod(35'249) = 2330\\
  6677^509 \bmod(35'249) = \\
  3277^509 \bmod(35'249) = \\
  8584^509 \bmod(35'249) = \\
  8469^509 \bmod(35'249) = \\
  7890^509 \bmod(35'249) = \\
\end{flalign*}
$ 72^{23} \bmod 247 = 002 $ \\
$ 10^{23} \bmod 247 = 212 $ \\
$ 11^{23} \bmod 247 = 045 $ \\
$ 08^{23} \bmod 247 = 031 $ \\
$ 10^{23} \bmod 247 = 212 $ \\
$ 81^{23} \bmod 247 = 009 $ \\
$ 11^{23} \bmod 247 = 045 $ \\
$ 32^{23} \bmod 247 = 128 $ \\
$ 87^{23} \bmod 247 = 159 $ \\
$ 11^{23} \bmod 247 = 045 $ \\
$ 11^{23} \bmod 247 = 045 $ \\
$ 14^{23} \bmod 247 = 105 $ \\
$ 10^{23} \bmod 247 = 212 $ \\
$ 81^{23} \bmod 247 = 009 $ \\
$ 00^{23} \bmod 247 = 000 $ \\
Somit würde der Text verschlüsselt \textit{002212045031212009045128159045045105212009000} lauten.
Um zu testen ob es stimmt verschlüsseln wir unser Text. Auch hier wieder die Zahlen in Unterteilen.
$ 002^{47} \bmod 247 = 72 $ \\
$ 212^{47} \bmod 247 = 10 $ \\
$ 045^{47} \bmod 247 = 11 $ \\
$ 031^{47} \bmod 247 = 08 $ \\
$ 212^{47} \bmod 247 = 10 $ \\
$ 009^{47} \bmod 247 = 81 $ \\
$ 045^{47} \bmod 247 = 11 $ \\
$ 128^{47} \bmod 247 = 32 $ \\
$ 159^{47} \bmod 247 = 87 $ \\
$ 045^{47} \bmod 247 = 11 $ \\
$ 045^{47} \bmod 247 = 11 $ \\
$ 105^{47} \bmod 247 = 14 $ \\
$ 212^{47} \bmod 247 = 10 $ \\
$ 009^{47} \bmod 247 = 81 $ \\
$ 000^{47} \bmod 247 = 00 $ \\
Das 000 ist ein bisschen problematisch, denn eigentlich gibt es nur 0 aber da wir ja auf zwei Stellen kommen müssen, wissen wir dass es 00 sein muss. Wie wir sehen haben wir unseren Text \textit{721011081081113287111114108100}  wieder.


\section{Anwendung}
Asymmetrische Verschlüsselungen werden häufig verwendet.  Obwohl verschiedene neue asymmetrische Verschlüsselungen erfunden wurden, deckt der RSA-Algorithmus weiterhin eine grosse Menge der asymmetrischen Verschlüsselung ab.\\
Wir möchten hier einige Beispiele aufzeigen, in denen der RSA-Algorithmus zu tragen kommt. Prinzipiell kommt ein asymmetrisches Verfahren dann zum Einsatz, wenn sich zwei Parteien ohne vorherigen Kontakt eine sichere Kommunikation aufbauen möchten. Es ist naheliegend, dass im IT-Bereich, über ein Netzwerk bzw. das Internet, solche Anforderungen bestehen.

\subsection{SSH - Secure Shell}
Die secure shell dient zum Aufbau einer Verbindung auf ein Gerät. Dies sind meistens Netzwerkkomponente oder Server. Die shell ist eine Konsole mit der Befehle an das Gerät gesendet werden kann.\\
Ohne RSA-Verschlüsselung müsste die Wartung vor Ort mit einem Kabel gemacht werden oder ein gleichbleibender Schlüssel vergeben werden. Aus Sicherheits- und Zeitgründen wären beide Möglichkeiten ziemlich schlecht.

\subsection{RFID}
RFID ist eine Technologie für den Kontaktlosen Austausch von Informationen. Dies funktioniert über elektromagnetische Wellen. Der Einsatz ist sehr unterschiedlich. In der Logistik wird es gebraucht um Waren schneller zu finden und direkt mit Listen abzugleichen, ohne jedes Produkt einzeln über einen Strichcode zu scannen. Das RSA-Kryptosystem wird zum Austausch des Schlüssels bei RFID basierten Zugangssystemen verwendet. Ohne ein solches asymmetrisches Verfahren, könnte jemand die Verbindung abhören und die Karte somit kopieren bzw. den Schlüssel der Karte herausfinden.

\subsection{PGP}
PGP ist ein Programm zur Verschlüsselung von Daten mit verschiedenen Methoden. Seit der ersten Version 1991 kann RSA verwendet werden. Das Programm wurde ausserhalb der USA weiterentwickelt und liegt seit 1998 auch Quelloffen(open source) bereit. Da in den USA Exportbeschränkungen auf Kryptografische Software vorhanden sind, durfte die Software nicht in der üblichen Form vertrieben werden. Um dieses Exportverbot zu Umgehen, wurde der Quellcode der Software in Buchform vertrieben und exportiert. In Europa wurde er dann wieder abgeschrieben und kompiliert(ausführbar gemacht). 


%
%---------------------------------------------------------------------------------------------
% Was macht den RSA-Algorithmus so sicher, wie kann man ihn brechen, kann man ihn brechen
%---------------------------------------------------------------------------------------------
\newpage
\part{Sicherheit}
\section{Allgemeines}
%Referenz:http://netlab.cs.ucla.edu/wiki/files/shannon1949.pdf
%Bzw. Wikipedia: http://de.wikipedia.org/wiki/Beweisbare_Sicherheit
In der Kryptologie spricht man von einer beweisbaren Sicherheit, welche eigentlich keine ist. Um die Sicherheit zu beweisen werden verschiedene Annahmen getroffen. Dies bezieht sich oft auf die zur Verfügung stehenden Rechenleistung und den entsprechenden Zeitaufwand. Da die Rechenleistung zunimmt werden die Schlüssel mit der Zeit länger. \\
Der Mathematische Beweis ist jeweils eine Reduktion des Problems. Es wird dabei angenommen das es schwierig ist z. B. zwei Primzahlen wieder zu faktorisieren. Dabei ist aktuell nur keine effektive Möglichkeit gefunden worden, das Problem zu lösen. Es ist jedoch nicht bewiesen, ob es unmöglich oder gar schwierig ist.

%#######################################################
% Primzahlen und Faktorisierungsproblem
%#######################################################
\section{Faktorisierungsproblem}
\subsection{Schwierigkeit}
Die Sicherheit der RSA-Verschlüsselung beruht auf dem Faktorisierungsproblem. Es wurde bisher noch keine effektive Methode gefunden um grosse Zahlen in ihre Primzahlen zu faktorisieren. Besonders schwer ist es wenn die Zahl aus zwei grossen Primzahlen besteht. 
Ein kleines Beispiel:
Die Faktorisierung von 14 in die beiden Primzahlen 2 und 7 fällt nicht sehr schwer\\
Die Faktorisierung von 30518 ist ebenfalls einfach, da nur eine Primzahl gross gewählt wurde. Es handelt sich um 2 und 15259. \\
Die Faktorisierung von 771'151 ist schon wesentlich schwerer. Obwohl es die kleinen Zahlen 823 und 937 sind. Das RSA-Modul N wird aus zwei Primzahlen mit einer Länge von 2048 Bit (617 Dezimalstellen) erstellt. Wir sehen hier das die Sicherheit von RSA wesentlich von der gewählten Schlüssellänge abhängt. Aus diesem RSA Modul wieder die zwei Primzahlen zu finden kann aktuell nicht effizient bewerkstelligt werden.
%
\section{Angriffe auf die Implementierung des RSA-Algorithmus}
Das Faktorisierungsproblem selbst konnte bisher nicht gelöst werden. Jedoch gibt es verschiedene Ansätze um bei bestimmten Konstellationen das Problem effizient zu lösen. 
Bei diesen Angriffen wurden die Primzahlen p und q falsch gewählt.
Falls die Primzahlen p und q zu nahe beieinander liegen, kann das Faktorisierungsproblem mit versuchen in einem bestimmten Zahlenspektrum überlistet werden. Das RSA-Modul N ist 753'343. Von dieser Zahl ziehen wir die Wurzel und erhalten $ \sqrt{753323} \approx 867.95 $. Wir können  ein Spektrum von +-100 bestimmen und erhalten 768 - 968. Die Zahl 753'343 versuchen wir durch die einzelnen Zahlen des Spektrums zu teilen. Bei der Zahl 859 erhalten wir die Zahl 877 und haben somit das Problem gelöst. Wir machen hier kein Primzahlentest da dies zusätzliche Rechenleistung benötigt und somit die Effizienz verringern würde. Daraus ist ersichtlich das die Primzahlen nicht nahe beieinander liegen dürfen.%
%
\subsection{Side Channel Attack}
Eine side channel attack ist einen Angriff auf die Implementierung des Verschlüsselungssystem. Es greift als nicht den RSA-Algorithmus direkt an.
Dieses Verfahren wurde 1996 von Herrn Paul C. Kocher, einem amerikanischen Kryptologen vorgestellt.
Im wesentlichen geht es bei einer side channel attacke darum, das kryptographische Gerät beim Ausführen ders Algorithmus zu beobachten und Korrelationen. bzw beziehungen feststellt zwischen den beobachteten Daten und dem Schlüssel zu finden.
Man beobachtet zum Beispiel die Laufzeit des Algorithmus, den Energieverbrauch des Prozessors während der Berechunung oder der elektromagnetische Ausstrahlung
%
% http://www.usna.edu/Users/math/wdj/book/node45.html
\subsubsection{Timing Attack}
Die Timing Attack ist eine side channel attack (Seitenkanalattacke). 
Bei ihr mistt man die Rechenzeit, die der Computer bzw. die CPU für die verschieden implementierungen des RSA-Verfahrens braucht. Die meisten Verschlüsselugns implementationen sind so geschrieben, dass sie möglichst schnell rechenen, da man grosse Primzahlen hat. Bei einer CPU, braucht jede Recheneinheit die exakt gleiche Zeit. Durch diese Analyse kann man den Schlüssel nach rekonstruieren. Das geht 
%
Verschiedene Berechnungen werden durchgeführt und ihre Zeiten notiert. Aus dieser Analyse werden Entschidungskriterin aufgestellt, die nur eine mögliche Lösung möglich ist. Diese Lösung hängt von dem gewählten Schlüssel ab.
%
Um sich von solchen Angriffen zu wehren, baut man in die Implementierung geeisse Berechnungen ein, damit jede Berechnung gleich lang dauert. So ist es nicht mehr möglich verschiedene Zeiten festzu stellen.
%
Diese Angriffe sind besonders bei Smart Cards effektiv, da man die Zeit sehr genau messen kann.
%
\subsection{Exponent Attack}
Bei der Exponent Attack, geht man davon aus, dass der private Schlüssel also d klein gewählt wurde. Zur veranschaulichung nehmen wir d = 3.
Die Nachricht m muss kleiner sein als N. Die entschüsselung würde dann so a
\section{Ausblick in die Zukunft}

%
\subsection{Side Channel Attack}
Eine Side Channel Attack (deutch: \textit{Seitenkanalattacke}) ist einen Angriff auf die Implementierung des Verschlüsselungssystem und nicht auf den RSA-Algorithmus. Dieses Verfahren wurde 1996 von Herrn Paul C. Kocher, einem amerikanischen Kryptologen entwickelt und vorgestellt.
Im wesentlichen geht es bei einer Side Channel Attacke darum, das kryptographische Gerät beim Ausführen des Algorithmus zu beobachten.
Man versucht aus den Beobachtungen einen Zusammenhang zwischen Daten und Schlüssel zu finden.
Unter die Beobachtung fallen zum Beispiel die Laufzeit des Algorithmus, den Energieverbrauch des Prozessors während der Berechnung oder der elektromagnetischen Ausstrahlung.
%
% http://www.usna.edu/Users/math/wdj/book/node45.html
\subsubsection{Timing Attack}
Die Timing Attack ist eine Side Channel Attack. 
Bei ihr misst man die Rechenzeit, die der Computer beziehungsweise die CPU für die Implementierung des RSA-Verfahrens braucht. Die meisten Verschlüsselungs-Implementationen sind so geschrieben, dass sie möglichst effizient arbeiten. Bei einer CPU braucht jede Recheneinheit die exakt gleiche Zeit. Durch diese Analyse kann man den Schlüssel nach und nach rekonstruieren.\\
%
Um sich von solchen Angriffen zu schützen, baut man in die Implementierung zusätzliche Berechnungen ein. So ist es nicht mehr möglich auf den Schlüssel zu schliessen.
%
%Diese Angriffe sind besonders bei Smart Cards effektiv, da man die Zeit sehr genau messen kann.
%
\subsection{Exponent Attack}
Bei der Exponent Attack geht man davon aus, dass der Verschlüsselungsexponenten e zu klein gewählt wurde. Das Problem liegt in der Modulo-Rechnung. Ist N grösser als $m^e$ ergibt der RSA-Verschlüsselungs Algorithmus als Resultat $m^e$. Dadurch kann man die n-te Wurzel aus der Verschlüsselte Nachricht ziehen. \\
Wenn wir für $e = 3$, unsere Nachricht $m = 4$ und für $N = 357$ erhalten wir zum Verschlüsseln folgende Formel.
\begin{equation*}
  4^3 \bmod(357) \equiv c
\end{equation*}
Da $4^3 = 64$ kleiner ist als 357 kann somit $64 \bmod(357) = 64$.\\
Da man e als Exponent im öffentlichen Schlüssel hat, kann man bequem $\sqrt[3]{64}$ und erhält als Resultat die korrekte Nachricht.\\
%
Um sich vor solchen Angriffen zu Schützen, ist es sinnvoll eine grösser Zahl für e zu verwenden.

%
%-----------------------------------------------------------
%SCHLUSS, Quellen, Zusammenfassung usw.
%-----------------------------------------------------------
\newpage
\part{Schluss}
\section{Schluss}
\subsection{Leitfrage beantworten}
Durch unsere Leitfrage konnten wir die Arbeit eingrenzen. Trotzdem mussten wir viele verschiedene Themen behandeln oder anschneiden um diese Frage nun beantworten zu können. Unsere Leitfrage war:
\begin{center}\textit{Wieso ist RSA heute noch sicher?}\end{center}
Im Kapitel Sicherheit sind wir unter anderem auf die Schlüssellängen eingegangen. Durch die schnelleren Computer musste die Schlüssellänge erhöht werden, damit die Sicherheit gewährleistet ist. Somit können wir die Frage wie folgt beantworten: Der RSA-Algorithmus ist heute noch sicher weil die Schlüssellänge erhöht wurde.\\
Das Faktorisierungsproblem von grossen Primzahlen, auf welches der RSA-Algorithmus aufbaut, konnte bisher noch nicht effizient gelöst werden. Daher kann die Frage zusätzlich wie folgt beantwortet werden: Der RSA-Algorithmus ist heute noch sicher, weil das Faktorisierungsproblem noch nicht effektiv gelöst werden konnte oder es keine effektive Lösung gibt.
%
\subsection{Ergebnisse}
Unsere Arbeit erfasst das Thema der RSA-Verschlüsselung auf einer zweckmässigen Ebene. Die Werke, welche uns bei der Recherche begegnet sind, waren im Gegensatz sehr kompliziert und detailliert auf gewisse Themen zugeschnitten. Diese einfachere, übersichtliche Auffassung des Themas haben wir vergleichsweise nicht gefunden und ist daher neu.\\
Das Ziel dieser IDPA war den RSA-Algorithmus grundlegend zu verstehen und herauszufinden, was ihn sicher macht.\\
Dieses Ziel haben wir für uns persönlich erreicht und konnten das erarbeitete Wissen in unsere Arbeit einfliessen lassen. Durch die Anwendung und die Hintergründe des RSA-Algorithmus lernten wir für uns neue mathematische Verfahren kennen. Dieses Wissen können wir in unserem zukünftigen Beruf für die Evaluierung von Sicherheitskonzepten oder die Implementierung des Algorithmus nutzen. Im Allgemeinen Teil sind wir auch auf die gesellschaftliche Bedeutung der Verschlüsselung eingegangen. Die sichere Kommunikation zwischen Botschaften oder Behörden ist für die Allgemeinheit von grosser Bedeutung. Die Verschlüsselung kann jeweils für das Gute, wie die sichere Kommunikation zwischen Botschaften, oder das Schlechte, wie der verschlüsselte Austausch von illegalen Dateien, verwendet werden. Trotz der möglichen verwerflichen Einsatzgebiete muss die Verschlüsselung sicher sein und legal, da ansonsten für die Allgemeinheit grosse Einbussen in der Sicherheit bestehen. \\
%
\subsection{Rückblick und Stellungnahme IDPA}
% Kritische Würdigung der gewählten Vorgehensweise und der Ergebnisse
Unsere IDPA sehen wir als Erfolg an. Unser gesteckten Ziele konnten wir erreichen. Wir sind mit unserer Arbeit zufrieden, da sie sehr professionell formatiert und auch ein sehr komplexes Thema behandelt. Wir versuchten jeweils das Wichtigste aus der Literatur zu recherchieren und schrieben diese neuen Erkenntnisse in unserer Arbeit nieder. \\
Folgende Entscheidungen wirkten sich auf unsere Arbeit aus:
\paragraph*{Latex}
Die Verwendung von Latex war ein Erfolg. Die Einarbeitung brauchte einiges an Zeit, jedoch holten wir diese mit den umfangreichen Formeln und Formatierungen wieder auf. Wir sind mit der Formatierung und der klaren Struktur zufrieden.
\paragraph*{Versionierung mit Subversion}
Die Versionierung mit Subversion\footnote{Subversion ist eine Software, welche die Versionierung handhabt und eventuelle Konflikte anzeigt} hatte mehrere Vorteile. Durch die Datensicherung online konnten wir gewährleisten, das bei einem Datenverlust trotzdem die letzte Version wieder zur Verfügung steht. Durch den Onlineabgleich konnten wir an der aktuellen Version arbeiten und stellten fest, falls ein Teammitglied die gleiche Zeile bearbeitet hat. 
\paragraph*{Zusammenarbeit}
Durch die Aufteilung von verschiedenen Teilen im Team, konnten wir schnell verschiedene Gebiete erarbeiten. Durch Zeitaufwändige Besprechungen haben wir uns jeweils auf den aktuellsten Stand gebracht. Ohne die Aufteilung hätten wir jedoch nicht diesen Umfang erreicht.
%
\subsection{Ausblick in die Zukunft}
Der weitere Einsatz des RSA-Algorithmus ist schwierig vorherzusehen. Wir vermuten das die Schlüssellänge ab 2015 auf 4096 Bit erhöht wird, da die Vergangenheit gezeigt hat, das alle 10 Jahre die empfohlene Schlüssellänge verdoppelt wird. Daher vermuten wir das die Sicherheit auch noch in 5 Jahren gegeben ist.\\
Die Entwicklung der Rechenleistung ist exponentiell, daher wird das knacken durch mögliches ausprobieren einfacher. Durch das Netzwerk von vielen PCs und Mobiltelefonen können Angreifern eine enorme Rechenkapazitäten zur Verfügung stehen. Diese Problematik muss weiter im Auge behalten werden.\\
Nicht absehbar ist die Entwicklung im Bereich von Quantencomputer. Falls dadurch neue und effektivere Methoden zur Verfügung stehen, könnte diese die Verschlüsselung in Bedrängnis bringen. \\
Wir vermuten das Faktorisierungsproblem wird in den nächsten Jahren nicht gelöst. Daher bleibt die Grundlage des Algorithmus weiter sicher.
%
%Ob der RSA-Algorithmus auch noch in 20 Jahren sicher ist, kann man nicht vorhersagen. Es ist aber davon auszugehen, dass das Faktorisierungs-Problem nicht gelöst werden kann und somit der RSA-Algorithmus sicher bleibt.\\
%Eine Gefahr geht von den immer schneller werdenden Computern aus, die in einer kürzeren Zeit mehr Berechnungen lösen können. Dieses Problem lässt sich lösen, indem man längere Schlüsselpaare verwendet.\\
%Durch die Vernetzung von verschiedenen Computern steht Angreifern eine noch grössere Rechenkapazität zur Verfügung. Durch das kann die Rechenzeit für einen Angriff massiv verkürzt werden. Daher sollte man nebst dem langen Schlüssel das Schlüsselpaar regelmässig erneuern.
%
%Die Implementierung der RSA-Verschlüsselung wird auch immer weiter entwickelt und verbessert, sodass Angriffe abgewehrt werden können.
%Somit kann man davon ausgehen, dass der RSA-Algorithmus auch in Zukunft sicher sein wird, wenn man die Schlüssellänge stetig vergrössert.
%
%\\[2ex]
\subsection{Dank}
An dieser Stelle möchten wir unseren Betreuern Herrn Stefan Kuster und Garry Colin herzlich für Ihre Unterstützung danken.\\
Auch möchten wir Herrn Heer, Frau Wernli und Herrn Mauchle für ihre Hilfe und ihre Zeit, die sie uns zur Verfügung gestellt haben, danken.
\newpage
\section{Abstract}
We wrote this paper for our professional baccalaureate between August and December 2011. The main subject is the RSA-algorithm, especially the mathematical proof of this asymmetric encryption method. We decided to choose this subject because we are apprentices in application engineering and have a particular interest in technical matters. We opted for the combination of English and Maths because the literature dealing with the RSA-algorithm is mostly in English, additionally it is not possible to work without a fundamental understanding of maths. The reasons for the use of latex are firstly the fact that it is a mathematical work and secondly, the extensive character of the document. %Furthermore, the advantages are the correct typeset of the formulas. 
%For this paper we worked with subversion (a version controlled environment) to guarantee the backup and work with the latest version anytime. \\
%We kept our working journal up to date by means of the google project document. \\
The main goal of this paper was to understand the functionality of the RSA-algorithm as well as a new region of maths. We divided our work in three parts to achieve the above mentioned aims. The first part consist of an overview of cryptography linked with information about the historical background. Already Julius Caesar had thought about encryption and used some primitive algorithm, today known as the Caesar encryption. If the allied forces had not broken the Enigma, the Second World War would have been totally different. %Anders verlaufen
In the second chapter, we show the acquired background knowledge and explain the functionality of the RSA-algorithm. 
On the one hand, the use of encryption and decryption is rather reasonable. On the other hand we were facing the two following difficulties. The one of creating the correct private and public key, which is more complicated due to the use of many different formulas. Furthermore, we wanted to demonstrate why the algorithm works. For this we had deal with the usage of mathematical knowledge and work out a combination to proof the algorithm's correctness. 
The last part is about the subject of security. In this context we answer the following research question: Is the RSA-algorithm secure nowadays? We did not find a final answer for this question, but with the made assumptions, %Gemachten Annahmen
we can say that it is secure in so far as the conditions of a 2048 bit key and the correct implementation to find the randomize primes are full filled. \\
When we started our research we had to acquire a lot of new knowledge in maths. Although this gained knowledge is interesting, it took some time to fully understand the meaning due to its complexity. After this face of collecting information we finally succeeded in delving into the RSA-algorithm. %Sich in ein Thema vertiefen
From day to day we came closer to understanding the principles of the algorithm and the encryption in general. The learning process was intensive and exciting. This new accomplished knowledge can be used in our future job where different encryption algorithms need to be compared or even implemented. We are now able to understand the usage of an asymmetric encryption and even found some practical examples in our daily life which need those encryptions.  
We found a lot of information about the algorithm on the internet. Although this information was properly researched we got most of our processed knowledge out of academic books. We compared these various sources and merged them together in this paper.\\
This paper was successful because we reached the main aim of the subject and the professional baccalaureate. On the one hand we increased our knowledge about primes and some specific algorithms, on the other hand we had to eliminate some derivations  %Herrleitung
because it goes beyond the scope of this paper. The challenges were to concentrate on the fundamental parts as well as to understand the difficult formulas. We are pleased that we solved these problems with a lot of effort and look forward to future developments. 
\newpage
\section{Erklärung}
Hiermit erklären wir, dass wir die Interdisziplinäre Projektarbeit ohne fremde Hilfe angefertigt haben und nur die im Quellenverzeichnis angeführten Quellen und Hilfsmittel benutzt haben.\\
%Ich weiss, nit so schön aber funktioniert :)
  \begin{tabbing}
    Name \= Platz \= Platz \= Platz \=  Unterschrift \kill 
    Nicolas Mauchle:   \> \> \> \>   \rule{50mm}{1pt} \\[2ex]
    Denis Augsburger:   \> \> \> \>  \rule{50mm}{1pt} \\
  \end{tabbing}
Muttenz, 19. Dezember 2011
\newpage
\section{Anhang}
% Abbildungsverzeichnis
\listoffigures
% Tabellenverzeichnis
\listoftables
%


\bibliography{myrefs}		% expects file "myrefs.bib"
%
\end{document}
