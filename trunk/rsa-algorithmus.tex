\part{Der RSA-Algorithmus}
%
\section{Mathematisches Verfahren}
Der RSA-Algorithmus baut grundsätzlich auf Primzahlen, Modulo, Eulersche-Funktion, Satz von Euler und den erweiterten euklidischen Algorithmus auf. Verstehn man diese mathematischen Verfahren nicht, kann man auch nicht den RSA-Algorithmus verstehen und verwenden. \\
Im nächsten Abschnitt wollen wir die einzelnen mathematischen Verfahren kurz in ihren Funktionen erläutern.  
\subsection{Modulo}
Modulo oder auch \textit{Division mit Rest} gibt den Rest von einer ganzzahligen Division an. Modulo erklärt sich am einfachsten an einem Beispiel.\\
$ 21 / 5 = 4.2 $ \\
4.2 ist keine ganze Zahl. $ 4 * 5 = 20 $. \\
Bis auf 21 fehlt genau 1. Somit ist $ 21 \bmod 5 = 1 $
\paragraph{Kongruent Modulo}
Kongruent modulo heisst nichts anders, als dass zwei Zahlen modulo eine Zahl den selben Rest haben. Man sagt auch, dass sie in der gleichen Restklasse sind.\\
Als Beispiel nehmen wir die Zahlen 9 und 7 die modulo 2 gerechnet werden.\\
\begin{center}
$ 9 \bmod(2) = 1 $ \\
$ 7 \bmod(2) = 1 $ \\
%$ 9 \bmod(2) $ ist kongruent zu $ 7 \bmod(2) $ \\
$ 9 \bmod(2) \equiv 7 \bmod(2) $ \\
\end{center}
Denn beide ergeben als Resultat 1.
\subsection{Eulersche Funktion}
Die Eulersche Funktion sagt aus, wieviele teilfremde \footnote{Zahlen sind sich teilfremd, wenn ihr grösster gemeinsamer Teiler 1 ist.} natürliche Zahlen zu einer Zahl es gibt.\\
Für den RSA-Algorithmus verwenden wir die Eulersche Funktion nur bei Primzahlen. Deshalb schauen wir uns nur die spezial Formel bei Primzahlen an.\\
Primzahlen sind nur durch sich selber und 1 dividirbar und ihr grösster gemeinsamer Teiler ist immer 1. Somit ist die Eulersche Funktion einer Primzahl p immer p - 1.
\begin{center}
$ \varphi(p) = p - 1 $
\end{center}
Wenn wir jetzt also zwei verschiedene Primzahlen haben, gilt
\begin{center}
$ \varphi(pq) = (p - 1) * (q - 1) $
\end{center}
Der Beweis dafür liefert folgende Überlugungen. Es gibt insgesamt $ p * q -1 $ ganze Zahlen die kleiner sind als $ p * q $. Es ist bei Primzahlen einfacher die \textbf{nicht teilfremden} Zahlen zu zählen und diese dann von allen möglichen Zahlen abzuziehen als allte teilfremden Zahlen zu zählen. Nicht teilfremd zu p sind $ (q - 1) * p $ und für q gilt $ (p - 1) * q $\\
Als Formel geschrieben
\begin{center}
$ \varphi(pq) = p * q -1 - (p - 1) - (q - 1) $ \\
$ \varphi(pq) = p * q -1 - p + 1 - q + 1 $ \\
$ \varphi(pq) = p * q -q - p + 1 $ \\
$ \varphi(pq) = (p -1) * (q - 1) $ 
\end{center}
\subsection{Satz von Euler}
Der Satz von Euler sagt über zwei teilfremde Zahlen folgende Gleichung aus
\begin{center}
$ m^{\varphi(n)} \bmod(n) = 1 $
\end{center}
Nehem wir zwei Primzahlen p = 3 und q = 5. Und als m wählen wir 4 und setzten das ganze in unsere Formel ein.\\
\begin{center}
$ 4^{\varphi(3 * 5)} \bmod(3 * 5) = 1 $ \\
$ 4^8 \bmod(15) = 1 $ \\
\end{center}
Wenn wir jetzt noch eine natürliche Zahl k nehmen, gilt für diese folgende Aussage \\
\begin{center}
$ m^{1 + k * \varphi(n)} \bmod(n)  = m * 1$ \\
%$ m*m^{k * \varphi(n)} \bmod(n)  = m$ \\
\end{center}
Hat man zwei verschiedene Primzahlen p und q und eine natürliche Zahl m die kleiner ist als p * q. Dann gilt für jede natürliche Zahl k \\
\begin{center}
\boxed{m^{k * (p - 1) * (q - 1) +1} \bmod(p * q) = m}
\end{center}
Dieser Satz wird in der Mathematik auch als \textit{kleiner Satz von Fermat} genannt.
\subsection{Euklidischer Algorithmus}
Der Euklidischer Algorithmus errechnet den grössten gemeinsamen Teiler (ggT) von zwei natürlichen Zahlen. Um den ggT herauszufinden gibt es eine ältere das Subtraktions-Verfahren und eine modernere Methode das Modulo-Verfahren. Da das ältere Verfahren nicht so effizient ist, konzentrieren wir uns auf das moderne Verfahren.\\
Die Formel für das modernere Verfahren $a = q * p + r$ ist eine modulo Rechnung. Der Rest r wird zum neuen b und b wird zum neuen a. \\
Zu veranschaulichung nehmen wir zwei Zahlen a = 5984 und b = 302 und setzen diese in usere Gleichung ein.
\begin{center}
$ 5984 = 19 * 302 + 146 $\\
$ 302 = 2 * 146 + 10 $\\
$ 146 = 14 * 10 + 6 $ \\
$ 10 = 1 * 6 + 4 $ \\
$ 6 = 1 * 4 + 2 $ \\
$ 4 = 2 * 2 + 0 $\\
\end{center}
Der grösstegGemeinsame Teiler von 5984 und 19 ist somit 2. \\[2ex]
Um dieses Verfahren zu beweisen gehen wir davon aus das a, b, r und q alles natürliche Zahlen sind. Da wir im zweiten Schritt den grössten gemeinsamen Teiler von b und r errechnet haben, kann man folgende Gleichung aufstellen.
\begin{center}
$ ggT(a,b) = ggT(b,r) $\\
\end{center}
Wenn also eine Zahl a und b teilt, muss diese Zahl auch r teilen. Denn sonst würde die Gleichung oben nicht aufgehen. Umgekehrt muss eine Zahl die b und r teilt auch a teilen. Gehen wir zurück zu unserem Beispiel. Da diese Gleichung stimmt, stimmt auch unseres Verfahren.
\begin{center}
$ ggT(5984,302) = ggT(302,146) = ggT(146,10) = ggT(10,6) = ggT(6,4) = 2 $
\end{center}
\subsubsection{Erweiterter Euklidischer Algorithmus}
%




%
\newpage
\section{RSA Ver- und Entschlüsselung}
Der Name RSA setzt sich aus den Anfangsbuchstaben der Nachnamen der Entwickler zusammen. Das RSA Verfahren wird auch heute noch oft eingesetzt. Das Verfahren gilt bei einer bestimmten Schlüssellänge als sehr sicher. Es gibt verschiedenste Angriffsmöglichkeiten, diese führen jedoch nicht in einem vernünftigen Zeitraum zu Ergebnissen.\\
Das RSA Verfahren ist verwandt mit dem Rabin-Kryptosystem, dass auch auf Primzahlen beruht. Das Verfahren wird zur Signierung und zur Verschlüsselung verwendet. %Verweis%

\subsection{Der Schlüssel}
Dieses Kapitel wurde mit Hilfe des Buches \textit{RSA and Public-Key Cryptography} erarbeitet.\\[2ex]
%
Für eine sichere Verschlüsslung braucht es auf einen guten Schlüssel. Denn nur mit diesem Schlüssel, kann man die verschlüsselte Nachricht entschlüsseln. Der Schlüssel spielt also eine zentrale und wichtige Rolle. Dazu muss er so komplex sein, dass man ihn nicht erraten kann, oder durch Ausprobieren entdeckt.\\
Darum ist es auch sehr wichtig, dass man Zufallszahlen nimmt, die man nicht vorhersagen kann, beziehungsweise im Nachhinein berechnen kann.
%
\subsubsection{Zufallszahlen generieren}
In der Informatik sind Zufallszahlen eine sehr komplexe Angelegenheit. Um solche Zufallszahlen zu generieren, haben grössere Firmen einen Zufallszahlengeneratoren (eng. \textit{random number generator} RNG) entwickelt. Diese messen den radioaktiven Zerfall oder beobachteten die atmosphärischen Bedingungen in der Umgebung. Die Zahlen, die der RNG auswertet, nennt man \textbf{Seed-Zahlen}. Mit den gewonnen Seed-Zahlen, kann man jetzt einen Algorithums füttern, der eine Zahl oder eine Zahlenfolge zurückgibt. Hat man eine solche Zufallszahl, muss noch überprüft werden, ob es eine Primzahl ist. Zur überprüfung gibt es extra Verfahren, die wir hier nicht weiter erleutern werden. Mehr Informationen dazu im Buch \textit{RSA and Public-Key Cryptography - Kapitel 4}.\\
%
Für den normalen Benutzer sind solche Apparate viel zu aufwendig und zu teuer. Deshalb entwickelten Computerhersteller einen Pseudozufallszahlengenerator (eng. \textit{pseudo random number generator} PRNG). Dieser PRNG erstellt die Seed-Zahlen zum Beispiel aus der momentanen Position der Maus, momentane CPU Auslastung oder andere nicht vorhersehbare Elemente.\\
Ob die Zufallszahlen geeignet sind oder nicht ist schwierig zu sagen. Denn wie will man eine Folge von Zufallszahlen beurteilen?\\
Das Einzige, was man errechnen kann, ist die Entopie\footnote{Die Entropie beschreibt das Mass der Unordnung.}. Je höher die Entropie, desto unwarscheinlicher ist es, dass diese Zahlenfolge ein zweites Mal vorkommt.
%Buch RSA and Public-Key Seite 61 
\subsubsection{RSA-Schlüssel generieren}
Für eine RSA-Verschlüsselung brauchen wir zwei Schlüssel, die voneinander abhängig sind. Einen private Key und einen public Key. Im nächsten Abschnitt werden wir erläutern, wie man diese zwei Schlüsser errechnen kann.
%
\paragraph{Public-Key erstellen}\label{sec:public_key}
Als erstes brauchen wir zwei Primzahlen p und q die etwa gleich lang sind. Je grösser desto besser und sicherer. Wie man solche Primzahlen erstellt möchten wir hier nicht erläutern. Es gibt Methoden, bei denen man mit grosser Wahrscheinlichkeit eine Primzahl generieren kann. Man muss die Zahl nachher immer überprüfen.\\
Sind beide Primzahlen gefunden, errechnen wir uns N auch als RSA-Modul genannt.
%
\begin{equation}
  N = p \cdot q
  \label{eqn:rsa_modul}
\end{equation}
%
Als nächstes rechnen wir mit der Eulerschen Funktion [\ref{eqn:eulersche_func}] $\varphi$(N) von p und q aus.\\
Zu der Zahl $\varphi$(N) nehmen wir eine weitere zufällige, \textbf{teilfremde} Zahl e die den RSA verschlüsselungs Exponent bildet.\\
So nun haben wir unseren öffentlichen Schlüssel mit den Zahlen N und e.
\paragraph{Private-Key erstellen}
Wir haben vorher bei der Erstellung des public Keys die Zahlen N und e errechnet. Als nächstes wird aus diesen beiden Zahlen d errechent. d wird als enschlüsselungs Exponent im pirvate Key benötigt.\\
Mit dem erweiterten Euklidischen Algorithmus [\ref{eqn:erw_euklid_algo}] wird d berechnet so dass die Formel
%
% TODO: Auf modulare inverse hinweisen!!!!!
\begin{flalign*}
  e * d \bmod(\varphi(n)) = 1\\
  e * d \equiv 1 (\bmod \varphi(N) )
\end{flalign*}
%
stimmt. \\
Nachdem d ausgerechnet wurde, ist man im Besitz seines privaten Schlüssel mit den Zahlen d und N.
%\subsection{Schlüssel austauschen} Sprengt den Rahmen!!!!

%
\subsection{Formel Verschlüsselung}
Wenn die Schlüssel erstellt wurden, kann ein Text einfach verschlüsselt werden.\\
Dazu dient folgende Formel:
%
\begin{equation}
  c \equiv m^e  \bmod N
  \label{eqn:rsa_encription}
\end{equation}
%
%\subsection{RSA Entschlüsselung}
%Die RSA Entschlüsselung ist ohne den Geheimschlüssel zu wissen nicht möglich. Da die Sicherheit von diesem Geheimschlüssel abhängt, gilt es diesen möglichst lang zu wählen und geheim zu halten. Aktuell gilt eine Schlüssellänge von 2048 Bit als sicher. Als Dezimalzahl ausgedrückt, ist dies eine Zahl von $ 3,2 \cdot 10^{616} $
%Die Angriffe auf die RSA Verschlüsselung zielen darauf ab, den Geheimschlüssel zu ermitteln. Das Problem dabei ist die Primfaktoren Zerlegung von grossen Zahlen. Mehr dazu im Kapitel Angriffe %Verweis%

\subsection{Formel Entschlüsselung}
Die Entschlüsselung der Nachricht wird mit folgender Formel realisiert:
%
\begin{equation}
  m \equiv c^d \bmod N
  \label{eqn:rsa_decription}
\end{equation}
%
Der Klartext m hängt in diesem Fall vom verschlüsselten Text C, dem privaten Schlüssel d und dem RSA-Modul N ab. 
%
%
\newpage
%******************************************************************************
% Mathematischer Beweis
%******************************************************************************
\section{Mathematisches Beweis der Funktionsweise}
Bisher nahmen wir an, dass der RSA-Algorithmus korrekt arbeitet. Diese Annahme möchten wir nun beweisen. Der Algorithmus arbeitet korrekt, wenn die Verschlüsselung und nachherige Entschlüsselung wieder das gleiche ergibt. Dies darf nur mit dem zugehörigen privaten und öffentlichen Schlüssel möglich sein.

\subsection{Verschlüsselung und Entschlüsselung gleichsetzen}
Für den Beweis benötigen wir die ursprünglichen Formeln zur Verschlüsselung und Entschlüsselung. Diese lauten:
\begin{flalign*}
  C & = m^e mod N \\
  m & = C^d mod N
\end{flalign*}
Da in beiden Formeln die Ursprungsnachricht m und die verschlüsselte Nachricht C vorkommen, können wir diese Formeln gleichsetzen. Dazu lösen wir die Entschlüsselungsformel nach C auf:
\begin{flalign*}
  m &= C^d mod N \\
  m &= C^d - k * N \\
  m + k \cdot N & = C^d \\
  \sqrt[d]{m + k \cdot N} & = C
\end{flalign*}
Modulo ergibt jeweils den Rest einer ganzzahligen Division. Durch Umformung kann diese jeweils als $ k \cdot N $ dargestellt werden, wobei k eine ganze Zahl sein muss.\\
Nun können wir C anders ausdrücken und in der Verschlüsselungsformel einsetzen.
%Wir setzen nun die beiden Formeln gleich, so das nur noch die Ursprungsnachricht m in der Gleichung vorhanden ist:
\begin{flalign*}
  m^e mod N & = \sqrt[d]{m + k \cdot N}\\
  m^{e \cdot d} mod N & = m + k \cdot N\\
  m^{e \cdot d} mod N & = m 
  %{m^e}^d mod N & = m \\
  %{m^d}^e mod N & = m
\end{flalign*}
Die Formel $ m^{e \cdot d} mod N = m + k \cdot N $ kann gekürzt werden, da k in jedem Fall 0 sein muss. Dies liegt daran, dass wir auf der linken Seite mit Modulo N den Rest ausgeben. Der Rest kann 1 bis N-1 gross sein. Da m einen Wert hat, muss k in diesem Fall 0 sein. \\
Wir möchten beweisen, dass die Verschlüsselung (hoch e) und nachherige Entschlüsselung (hoch d) wieder die ursprüngliche Nachricht ergibt:
\begin{equation*}
 m^{e \cdot d} mod N = m 
\end{equation*}
%
\subsection{Grundlagen zur Erklärung}
Für den Beweis benötigen wir vorherige Kenntnisse und bestimmte Sätze. Diese möchten wir hier nochmals kurz in Erinnerung rufen.\\ 
Das RSA-Modul N wird aus den ausgewählten Primzahlen p und q erstellt.
\begin{equation*}
  N = p \cdot q
\end{equation*}

e wurde teilerfremd zu $ \varphi(n) $ gewählt. 
$ \varphi(n) = (p-1) \cdot (q-1) $

Zusätzlich wurde d so gewählt, dass folgendes zählt:
\begin{equation*}
 e \cdot d + k \cdot \varphi(N) = 1 = ggT(e,\varphi(N))
\end{equation*}

Für den Beweis müssen wir ausschweifen auf den Satz von Euler Fermat. Dieser bildet die Grundlage zur RSA-Verschlüsselung. Er lautet wie folgt:
\begin{equation*}
	a^{\varphi(n)} \equiv 1\,(\mathrm{mod}\,n)
\end{equation*}
Da wir mit Primzahlen arbeiten, kann dieser Satz auch anders ausgedrückt werden. $ \varphi(n) $ gibt alle teilerfremden Zahlen zu n an. Da n durch zwei Primzahlen gebildet wurde, gibt es $ (p-1) \cdot (q-1) $ teilerfremde Zahlen. 
%
\subsection{Beweis der Funktionsweise}
Die Gleichsetzung der Entschlüsselung und Verschlüsselung dient uns als Grundlage des Beweises:
\begin{equation*}   
 m^{e \cdot d} mod N = m
\end{equation*}
%
Wir könnnen durch die modulare Inverse $ e \cdot d $ anders ausdrücken, siehe [\ref{eqn:mod_inverse}]. Diese lösen wir nach $ e \cdot d $ auf:\\
\begin{flalign*}
 e \cdot d &= 1 \bmod{\varphi(N)}  \\
 % e \cdot d + k \cdot \varphi(N) &= 1  \\
 e \cdot d &= 1 + k \cdot \varphi(N) \\
 e \cdot d &= k \cdot \varphi(N) + 1
\end{flalign*}
Wenn die Formel Modulo beinhaltet, kann diese jeweils auch mit k Mal den Rest ausgedrückt werden. \\
%
Nun ersetzen wir in unserer Ausgangslage $ e \cdot d $ durch $ k \cdot \varphi(n)+1 $ und erhalten die Gleichung wie in Formel [\ref{eqn:kleiner_satz_fermat}].
\begin{flalign*}
 m^{e \cdot d} mod N = m
 m^{ k \cdot \varphi(N) + 1} \bmod(N) = m  \\
 m^{k \cdot \varphi(N)} \cdot m \bmod(N) = m  \\
 { m^{ \varphi(N) }} ^k \cdot m \bmod(N) = m \\
 { m^{ \varphi((p-1)\cdot(q-1)) }} ^k \cdot m \bmod((p-1)\cdot(q-1)) = m
\end{flalign*}
%
%Durch den kleinen Satz von Fermat wissen wir dass $ \varphi(N) $ 1 sein muss. 
Mit dem kleinen fermatschen Satz können wir nun nach $ m^{\varphi(N)} $ auflösen.
\begin{flalign*}
  m^{(p-1)} &= 1 \bmod p \\
  m^{(p-1) \cdot (q-1)} &= 1 \bmod p \cdot q \\
  m^{\varphi(N)} & = 1 \bmod N 
\end{flalign*}
1. Die Nachricht hoch die Anzahl der Teilerfremden Zahlen bei einer Primzahl ergibt 1 Modulo die Zahl \\
2. Bei der RSA-Verschlüsselung haben wir die Kombination aus zwei Primzahlen.
%nicht hoch eine Primzahl gerechnet, sondern hoch $ (p-1) \cdot (q-1) $. 
Daher können wir den kleinen fermatschen Satz so darstellen.\\
3. $ p \cdot q $ ist das RSA-Modul N. $ (p-1) \cdot (q-1) $ sind die Teilerfremden Zahlen von N, da es sich bei p und q um Primzahlen handelt. \\ %Verweis phi(N) und Primzahlen%
%
Daraus resultiert das $ 1 \bmod N $ das gleiche ist wie $ m^\varphi(N) $ und entsprechend eingesetzt werden kann. Danach lösen wir die Formel auf. 
\begin{flalign*}
 { m^{ \varphi(N) }} ^k \cdot m \bmod(N) &= m  \\
 {1 \bmod N }^k \cdot m \bmod(N) &= m  \\
 1^k \cdot m \bmod(N) &= m \\
 1 \cdot m \bmod(N) &= m \\
 m + k \cdot N &= m \\
 m + 0 \cdot N &= m \\
 m &= m 
\end{flalign*}
Wir wissen das $ k = 0 $ sein muss, da m immer kleiner als N gewählt wird. Falls m grösser gleich N ist, würde der RSA-Algorithmus nicht funktionieren. In diesem Fall wird die Nachricht aufgeteilt und später wieder zusammengesetzt.\\
Schlussendlich stellen wir fest, dass m = m ist und beweisen somit die korrekte Funktionsweise des RSA-Algorithmus. Die Verschlüsselung und darauffolgende Entschlüsselung ergibt die Ursprungsnachricht.\\
Die Entschlüsselung der Nachricht funktioniert nur mit dem zugehörigen privaten Schlüssel, da mit dem euklidischen Algorithmus d so bestimmt wurde, das folgendes zählt: 
\begin{flalign*}
 e \cdot d + k \cdot \varphi(N) &= 1	
\end{flalign*}
Mit der modularen Inverse wurde d so bestimmt, das keine negative Zahl entsteht. Aus der vorherigen Formel ist ersichtlich, dass nur ein ganz bestimmtes d zu einem e und N gehört. Jeder andere Entschlüsselungexponent würde die Formel nicht korrekt erfüllen und zu einem anderen Ergebnis führen. 
%
%
%***************************************
% BEISPIEL
%***************************************
\section{Beispiel}

\subsection{Einfaches Zahlenbeispiel}
Ein einfaches Beispiel für den RSA-Algorithmus
\subsubsection{Schlüssel-Paar erstellen}
Ein einfaches Zahlenbeispiel. Wir suchen uns zwei kleine Primzahlen.
%
\begin{flalign*}
  p = 13 \\
  q = 19
\end{flalign*}
%
Wir berechnen nun das RSA-Modul N [\ref{eqn:rsa_modul}] und $\varphi(N) $ [\ref{eqn:eulersche_func}] aus p und q.
\begin{equation*}
  \tag{RSA-Modul}
  247 = 13 \cdot 19
\end{equation*}
%
\begin{equation*}
  \tag{$\varphi(N)$}
  216 = (13 - 1) \cdot (19 - 1)
\end{equation*}
%
Zu $ \varphi(N) $ suchen wir uns eine zweite Zahl e, die teilerfremd zu $ \varphi(N) $ ist, sprich den $ggT(\varphi(N),e) = 1$. Am Besten man verwendet eine weitere Primzahl. Wir nehmen für e die Primzahl 23.
%
\begin{equation*}
    e = 23
\end{equation*}
%
Der öffentliche Schlüssel ist nun berechnet [siehe Section \ref{sec:public_key}]\\
Um den entschlüsselungs Exponenten zu errechnen, müssen wir zuerst den ggT [\ref{eqn:euklidischer_algo}] aus 23 und 216 ausrechnen.
\begin{flalign*}
  216 & = 9 * 23 + 9 \\
  23 & = 2 *  9 + 5 \\
  9 & = 1 *  5 + 4 \\
  5 & = 1 *  4 + 1
\end{flalign*}
%
Jetzt weneden wir den Erweiterter Euklidischer Algorithmus [\ref{eqn:erw_euklid_algo}] an.
\begin{flalign*}
  1 &= 5 - 1 \cdot 4 = 5 - 1 \cdot(9 - 1 \cdot 5) = 5 - 1 \cdot 9 + 1 \cdot 5 = 2 \cdot 5 - 1 \cdot 9\\
  1 &= 2 \cdot 5 - 1 \cdot 9 = 2 \cdot (23 - 2 \cdot 9) - 1 \cdot 9 = 2 \cdot 23 - 4 \cdot 9 - 1 \cdot 9 = 2 \cdot 23 - 5 \cdot 9\\
  1 &= 2 \cdot 23 - 5 \cdot 9 = 2 \cdot 23 - 5 \cdot (216 - 9 \cdot 23) = 2 \cdot 23 - 5 \cdot 216 + 45 \cdot 23 = \textbf{47} \cdot 23 \textbf{- 5} \cdot 216
\end{flalign*}
Wenden die Modulare Inverse auf die letzte Gleichun an und erhalten $d = 47$.
%
Jetzt haben wir alle nötigen Zahlen die wir für eine Verschlüsselung sowie Entschlüsselung benötigen.
\subsubsection{Verschlüsseln}
Da man Text nicht verschlüsseln kann, braucht man für den Text Zahlen. Wir nehmen jetzt zur Vereinfachung eine Zahl.
\begin{flalign*}
  m &= 15 \\
  15^{23} \bmod 247 &= 59
\end{flalign*}
Die Zahl 15 verschlüsselt mit unserem öffentlichen Schlüssel (23,247) ergibt die Zahl 59.
%
\subsubsection{Entschlüsseln}
Um zu überprüfen ob unserer privater Schlüssel auch wirklich funktioniert, entschlüsseln wir die Zahl 59 mit unserem privaten Schlüssel.\\
\begin{flalign*}
  c &= 59  \\
  59^{47} \bmod 247 &= 15
\end{flalign*}
Somit sehen wir, dass unser kleines Beispiel funktioniert hat.
%
\subsection{Beispiel an einem Text}
Damit wir einen Text verschlüsseln können, brauchen wir grössere Zahlen. Für das nächste Beispiel sind folgende Zahlen gegeben.
\begin{flalign*}
  p &= 101\\
  q &= 349\\
  N &= 35'249\\
  \varphi(N) &= 34'800\\
  e &= 509\\
  d &= 7'589
\end{flalign*}
Wir müssen nur noch ein Verfahren wählen, wie wir einen Buchstaben in eine Zahl umwandeln. Für das gibt es \textit{ASCII} Tabellen, die für jedes Zeichen, eine Zahl darstellen.\\
Wir würden gerne das Wort \textit{GIBM MUTTENZ} verschlüsseln. Als erstes müssen wir unseren Text in Zahlen umwandeln. Dazu verwenden wir die Dezimalschreibweise.\\
\textit{GIBM MUTTENZ} würde in dieser Schreibweise \textit{717366773277858484697890} heissen. Da $m < N$ sein muss, müssen wir unsere Zahl in Blöcke aufteillen. Wir machen uns immer 4er Blöcke und verschlüsseln sie.
\begin{flalign*}
  7173^509 \bmod(35'249) = 2330\\
  6677^509 \bmod(35'249) = \\
  3277^509 \bmod(35'249) = \\
  8584^509 \bmod(35'249) = \\
  8469^509 \bmod(35'249) = \\
  7890^509 \bmod(35'249) = \\
\end{flalign*}
$ 72^{23} \bmod 247 = 002 $ \\
$ 10^{23} \bmod 247 = 212 $ \\
$ 11^{23} \bmod 247 = 045 $ \\
$ 08^{23} \bmod 247 = 031 $ \\
$ 10^{23} \bmod 247 = 212 $ \\
$ 81^{23} \bmod 247 = 009 $ \\
$ 11^{23} \bmod 247 = 045 $ \\
$ 32^{23} \bmod 247 = 128 $ \\
$ 87^{23} \bmod 247 = 159 $ \\
$ 11^{23} \bmod 247 = 045 $ \\
$ 11^{23} \bmod 247 = 045 $ \\
$ 14^{23} \bmod 247 = 105 $ \\
$ 10^{23} \bmod 247 = 212 $ \\
$ 81^{23} \bmod 247 = 009 $ \\
$ 00^{23} \bmod 247 = 000 $ \\
Somit würde der Text verschlüsselt \textit{002212045031212009045128159045045105212009000} lauten.
Um zu testen ob es stimmt verschlüsseln wir unser Text. Auch hier wieder die Zahlen in Unterteilen.
$ 002^{47} \bmod 247 = 72 $ \\
$ 212^{47} \bmod 247 = 10 $ \\
$ 045^{47} \bmod 247 = 11 $ \\
$ 031^{47} \bmod 247 = 08 $ \\
$ 212^{47} \bmod 247 = 10 $ \\
$ 009^{47} \bmod 247 = 81 $ \\
$ 045^{47} \bmod 247 = 11 $ \\
$ 128^{47} \bmod 247 = 32 $ \\
$ 159^{47} \bmod 247 = 87 $ \\
$ 045^{47} \bmod 247 = 11 $ \\
$ 045^{47} \bmod 247 = 11 $ \\
$ 105^{47} \bmod 247 = 14 $ \\
$ 212^{47} \bmod 247 = 10 $ \\
$ 009^{47} \bmod 247 = 81 $ \\
$ 000^{47} \bmod 247 = 00 $ \\
Das 000 ist ein bisschen problematisch, denn eigentlich gibt es nur 0 aber da wir ja auf zwei Stellen kommen müssen, wissen wir dass es 00 sein muss. Wie wir sehen haben wir unseren Text \textit{721011081081113287111114108100}  wieder.


\section{Anwendung}
Asymmetrische Verschlüsselungen werden häufig verwendet.  Obwohl verschiedene neue asymmetrische Verschlüsselungen erfunden wurden, deckt der RSA-Algorithmus weiterhin eine grosse Menge der asymmetrischen Verschlüsselung ab.\\
Wir möchten hier einige Beispiele aufzeigen, in denen der RSA-Algorithmus zu tragen kommt. Prinzipiell kommt ein asymmetrisches Verfahren dann zum Einsatz, wenn sich zwei Parteien ohne vorherigen Kontakt eine sichere Kommunikation aufbauen möchten. Es ist naheliegend, dass im IT-Bereich, über ein Netzwerk bzw. das Internet, solche Anforderungen bestehen.

\subsection{SSH - Secure Shell}
Die secure shell dient zum Aufbau einer Verbindung auf ein Gerät. Dies sind meistens Netzwerkkomponente oder Server. Die shell ist eine Konsole mit der Befehle an das Gerät gesendet werden kann.\\
Ohne RSA-Verschlüsselung müsste die Wartung vor Ort mit einem Kabel gemacht werden oder ein gleichbleibender Schlüssel vergeben werden. Aus Sicherheits- und Zeitgründen wären beide Möglichkeiten ziemlich schlecht.

\subsection{RFID}
RFID ist eine Technologie für den Kontaktlosen Austausch von Informationen. Dies funktioniert über elektromagnetische Wellen. Der Einsatz ist sehr unterschiedlich. In der Logistik wird es gebraucht um Waren schneller zu finden und direkt mit Listen abzugleichen, ohne jedes Produkt einzeln über einen Strichcode zu scannen. Das RSA-Kryptosystem wird zum Austausch des Schlüssels bei RFID basierten Zugangssystemen verwendet. Ohne ein solches asymmetrisches Verfahren, könnte jemand die Verbindung abhören und die Karte somit kopieren bzw. den Schlüssel der Karte herausfinden.

\subsection{PGP}
PGP ist ein Programm zur Verschlüsselung von Daten mit verschiedenen Methoden. Seit der ersten Version 1991 kann RSA verwendet werden. Das Programm wurde ausserhalb der USA weiterentwickelt und liegt seit 1998 auch Quelloffen(open source) bereit. Da in den USA Exportbeschränkungen auf Kryptografische Software vorhanden sind, durfte die Software nicht in der üblichen Form vertrieben werden. Um dieses Exportverbot zu Umgehen, wurde der Quellcode der Software in Buchform vertrieben und exportiert. In Europa wurde er dann wieder abgeschrieben und kompiliert(ausführbar gemacht). 

