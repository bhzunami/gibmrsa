\part{Der RSA-Algorithmus}
\section{Mathematisches Verfahren}
Der RSA-Algorithmus baut grundsätzlich auf Primzahlen, Modulo, Eulersche-Funktion, Satz von Euler und den erweiterten euklidischen Algorithmus auf. Verstehn man diese mathematischen Verfahren nicht, kann man auch nicht den RSA-Algorithmus verstehen und verwenden. \\
Im nächsten Abschnitt wollen wir die einzelnen mathematischen Verfahren kurz in ihren Funktionen erläutern.  
\subsection{Modulo}
Modulo oder auch \textit{Division mit Rest} gibt den Rest von einer ganzzahligen Division an. Modulo erklärt sich am einfachsten an einem Beispiel.\\
$ 21 / 5 = 4.2 $ \\
4.2 ist keine ganze Zahl. $ 4 * 5 = 20 $. \\
Bis auf 21 fehlt genau 1. Somit ist $ 21 \bmod 5 = 1 $
\paragraph{Kongruent Modulo}
Kongruent modulo heisst nichts anders, als dass zwei Zahlen modulo eine Zahl den selben Rest haben. Man sagt auch, dass sie in der gleichen Restklasse sind.\\
Als Beispiel nehmen wir die Zahlen 9 und 7 die modulo 2 gerechnet werden.\\
\begin{center}
$ 9 \bmod(2) = 1 $ \\
$ 7 \bmod(2) = 1 $ \\
%$ 9 \bmod(2) $ ist kongruent zu $ 7 \bmod(2) $ \\
$ 9 \bmod(2) \equiv 7 \bmod(2) $ \\
\end{center}
Denn beide ergeben als Resultat 1.
\subsection{Eulersche Funktion}
Die Eulersche Funktion sagt aus, wieviele teilfremde \footnote{Zahlen sind sich teilfremd, wenn ihr grösster gemeinsamer Teiler 1 ist.} natürliche Zahlen zu einer Zahl es gibt.\\
Für den RSA-Algorithmus verwenden wir die Eulersche Funktion nur bei Primzahlen. Deshalb schauen wir uns nur die spezial Formel bei Primzahlen an.\\
Primzahlen sind nur durch sich selber und 1 dividirbar und ihr grösster gemeinsamer Teiler ist immer 1. Somit ist die Eulersche Funktion einer Primzahl p immer p - 1.
\begin{center}
$ \varphi(p) = p - 1 $
\end{center}
Wenn wir jetzt also zwei verschiedene Primzahlen haben, gilt
\begin{center}
$ \varphi(pq) = (p - 1) * (q - 1) $
\end{center}
Der Beweis dafür liefert folgende Überlugungen. Es gibt insgesamt $ p * q -1 $ ganze Zahlen die kleiner sind als $ p * q $. Es ist bei Primzahlen einfacher die \textbf{nicht teilfremden} Zahlen zu zählen und diese dann von allen möglichen Zahlen abzuziehen als allte teilfremden Zahlen zu zählen. Nicht teilfremd zu p sind $ (q - 1) * p $ und für q gilt $ (p - 1) * q $\\
Als Formel geschrieben
\begin{center}
$ \varphi(pq) = p * q -1 - (p - 1) - (q - 1) $ \\
$ \varphi(pq) = p * q -1 - p + 1 - q + 1 $ \\
$ \varphi(pq) = p * q -q - p + 1 $ \\
$ \varphi(pq) = (p -1) * (q - 1) $ 
\end{center}
\subsection{Satz von Euler}
Der Satz von Euler sagt über zwei teilfremde Zahlen folgende Gleichung aus
\begin{center}
$ m^{\varphi(n)} \bmod(n) = 1 $
\end{center}
Nehem wir zwei Primzahlen p = 3 und q = 5. Und als m wählen wir 4 und setzten das ganze in unsere Formel ein.\\
\begin{center}
$ 4^{\varphi(3 * 5)} \bmod(3 * 5) = 1 $ \\
$ 4^8 \bmod(15) = 1 $ \\
\end{center}
Wenn wir jetzt noch eine natürliche Zahl k nehmen, gilt für diese folgende Aussage \\
\begin{center}
$ m^{1 + k * \varphi(n)} \bmod(n)  = m * 1$ \\
%$ m*m^{k * \varphi(n)} \bmod(n)  = m$ \\
\end{center}
Hat man zwei verschiedene Primzahlen p und q und eine natürliche Zahl m die kleiner ist als p * q. Dann gilt für jede natürliche Zahl k \\
\begin{center}
\boxed{m^{k * (p - 1) * (q - 1) +1} \bmod(p * q) = m}
\end{center}
Dieser Satz wird in der Mathematik auch als \textit{kleiner Satz von Fermat} genannt.
\subsection{Euklidischer Algorithmus}
Der Euklidischer Algorithmus errechnet den grössten gemeinsamen Teiler (ggT) von zwei natürlichen Zahlen. Um den ggT herauszufinden gibt es eine ältere das Subtraktions-Verfahren und eine modernere Methode das Modulo-Verfahren. Da das ältere Verfahren nicht so effizient ist, konzentrieren wir uns auf das moderne Verfahren.\\
Die Formel für das modernere Verfahren $a = q * p + r$ ist eine modulo Rechnung. Der Rest r wird zum neuen b und b wird zum neuen a. \\
Zu veranschaulichung nehmen wir zwei Zahlen a = 5984 und b = 302 und setzen diese in usere Gleichung ein.
\begin{center}
$ 5984 = 19 * 302 + 146 $\\
$ 302 = 2 * 146 + 10 $\\
$ 146 = 14 * 10 + 6 $ \\
$ 10 = 1 * 6 + 4 $ \\
$ 6 = 1 * 4 + 2 $ \\
$ 4 = 2 * 2 + 0 $\\
\end{center}
Der grösstegGemeinsame Teiler von 5984 und 19 ist somit 2. \\[2ex]
Um dieses Verfahren zu beweisen gehen wir davon aus das a, b, r und q alles natürliche Zahlen sind. Da wir im zweiten Schritt den grössten gemeinsamen Teiler von b und r errechnet haben, kann man folgende Gleichung aufstellen.
\begin{center}
$ ggT(a,b) = ggT(b,r) $\\
\end{center}
Wenn also eine Zahl a und b teilt, muss diese Zahl auch r teilen. Denn sonst würde die Gleichung oben nicht aufgehen. Umgekehrt muss eine Zahl die b und r teilt auch a teilen. Gehen wir zurück zu unserem Beispiel. Da diese Gleichung stimmt, stimmt auch unseres Verfahren.
\begin{center}
$ ggT(5984,302) = ggT(302,146) = ggT(146,10) = ggT(10,6) = ggT(6,4) = 2 $
\end{center}
\subsubsection{Erweiterter Euklidischer Algorithmus}
%





\newpage
\section{RSA Ver- und Entschlüsselung}
Der Name RSA setzt sich aus den Anfangsbuchstaben der Nachnamen der Entwickler zusammen. Das RSA Verfahren wird auch heute noch oft eingesetzt. Das Verfahren gilt bei einer bestimmten Schlüssellänge als sehr sicher. Es gibt verschiedenste Angriffsmöglichkeiten, diese führen jedoch nicht in einem vernünftigen Zeitraum zu Ergebnissen.\\
Das RSA Verfahren ist verwandt mit dem Rabin-Kryptosystem, dass auch auf Primzahlen beruht. Das Verfahren wird zur Signierung und zur Verschlüsselung verwendet. %Verweis%

\subsection{Der Schlüssel}
Dieses Kapitel wurde mit Hilfe des Buches \textit{RSA and Public-Key Cryptography} erarbeitet.\\[2ex]
%
Für eine sichere Verschlüsslung braucht es auf einen guten Schlüssel. Denn nur mit diesem Schlüssel, kann man die verschlüsselte Nachricht entschlüsseln. Der Schlüssel spielt also eine zentrale und wichtige Rolle. Dazu muss er so komplex sein, dass man ihn nicht erraten kann, oder durch Ausprobieren entdeckt.\\
Darum ist es auch sehr wichtig, dass man Zufallszahlen nimmt, die man nicht vorhersagen kann, beziehungsweise im Nachhinein berechnen kann.
%
\subsubsection{Zufallszahlen generieren}
In der Informatik sind Zufallszahlen eine sehr komplexe Angelegenheit. Um solche Zufallszahlen zu generieren, haben grössere Firmen einen Zufallszahlengeneratoren (eng. \textit{random number generator} RNG) entwickelt. Diese messen den radioaktiven Zerfall oder beobachteten die atmosphärischen Bedingungen in der Umgebung. Die Zahlen, die der RNG auswertet, nennt man \textbf{Seed-Zahlen}. Mit den gewonnen Seed-Zahlen, kann man jetzt einen Algorithums füttern, der eine Zahl oder eine Zahlenfolge zurückgibt. Hat man eine solche Zufallszahl, muss noch überprüft werden, ob es eine Primzahl ist. Zur überprüfung gibt es extra Verfahren, die wir hier nicht weiter erleutern werden. Mehr Informationen dazu im Buch \textit{RSA and Public-Key Cryptography - Kapitel 4}.\\
%
Für den normalen Benutzer sind solche Apparate viel zu aufwendig und zu teuer. Deshalb entwickelten Computerhersteller einen Pseudozufallszahlengenerator (eng. \textit{pseudo random number generator} PRNG). Dieser PRNG erstellt die Seed-Zahlen zum Beispiel aus der momentanen Position der Maus, momentane CPU Auslastung oder andere nicht vorhersehbare Elemente.\\
Ob die Zufallszahlen geeignet sind oder nicht ist schwierig zu sagen. Denn wie will man eine Folge von Zufallszahlen beurteilen?\\
Das Einzige, was man errechnen kann, ist die Entopie\footnote{Die Entropie beschreibt das Mass der Unordnung.}. Je höher die Entropie, desto unwarscheinlicher ist es, dass diese Zahlenfolge ein zweites Mal vorkommt.
%Buch RSA and Public-Key Seite 61 
\subsubsection{RSA-Schlüssel generieren}
Für eine RSA-Verschlüsselung brauchen wir zwei Schlüssel, die voneinander abhängig sind. Einen private Key und einen public Key. Im nächsten Abschnitt werden wir erläutern, wie man diese zwei Schlüsser errechnen kann.
%
\paragraph{Public-Key erstellen}\label{sec:public_key}
Als erstes brauchen wir zwei Primzahlen p und q die etwa gleich lang sind. Je grösser desto besser und sicherer. Wie man solche Primzahlen erstellt möchten wir hier nicht erläutern. Es gibt Methoden, bei denen man mit grosser Wahrscheinlichkeit eine Primzahl generieren kann. Man muss die Zahl nachher immer überprüfen.\\
Sind beide Primzahlen gefunden, errechnen wir uns N auch als RSA-Modul genannt.
%
\begin{equation}
  N = p \cdot q
  \label{eqn:rsa_modul}
\end{equation}
%
Als nächstes rechnen wir mit der Eulerschen Funktion [\ref{eqn:eulersche_func}] $\varphi$(N) von p und q aus.\\
Zu der Zahl $\varphi$(N) nehmen wir eine weitere zufällige, \textbf{teilfremde} Zahl e die den RSA verschlüsselungs Exponent bildet.\\
So nun haben wir unseren öffentlichen Schlüssel mit den Zahlen N und e.
\paragraph{Private-Key erstellen}
Wir haben vorher bei der Erstellung des public Keys die Zahlen N und e errechnet. Als nächstes wird aus diesen beiden Zahlen d errechent. d wird als enschlüsselungs Exponent im pirvate Key benötigt.\\
Mit dem erweiterten Euklidischen Algorithmus [\ref{eqn:erw_euklid_algo}] wird d berechnet so dass die Formel
%
% TODO: Auf modulare inverse hinweisen!!!!!
\begin{flalign*}
  e * d \bmod(\varphi(n)) = 1\\
  e * d \equiv 1 (\bmod \varphi(N) )
\end{flalign*}
%
stimmt. \\
Nachdem d ausgerechnet wurde, ist man im Besitz seines privaten Schlüssel mit den Zahlen d und N.
%\subsection{Schlüssel austauschen} Sprengt den Rahmen!!!!

\section{RSA Verschlüsselung}
Die RSA Verschlüsselung wurde 1977 von Ronald L. Riverst, Adi Shamir und Leonard Adleman entwickelt. Der Name RSA setzt sich aus den Anfangsbuchstaben der Nachnamen der Entwickler zusammen. Es handelt sich um ein aymetrisches Verfahren. (Siehe Kapitel moderne Kryptographie) %Verweis%
Das RSA Verfahren wird auch heute noch oft eingesetzt. Das Verfahren gilt bei einer bestimmten Schlüssellänge als sicher. Es gibt verschiedenste Angriffsmöglichkeiten, diese führen jedoch nicht in einem vernünftigen Zeitraum zu Ergebnissen. Das Verfahren wird zur Signierung und zur Verschlüsselung verwendet. (Siehe Anwendung) %Verweis%

\subsection{Formel Verschlüsselung}
Wenn der Schlüssel erstellt wurde kann ein Text einfach verschlüsselt werden. Dazu dient folgende Formel:

$ C \equiv m^e  \bmod N $

C steht für Cipher und bezeichnet den Geheimtext. Dieser ist abhängig vom Klartext m, dem öffentlichen Schlüssel e und dem RSA-Modul N. 

\section{RSA Entschlüsselung}
Die RSA Entschlüsselung ist ohne den Geheimschlüssel zu wissen nach aktuellem Wissensstand nicht möglich. Da die Sicherheit von diesem Geheimschlüssel abhängt, gilt es diesen möglichst lang zu wählen und geheim zu halten. Aktuell gilt eine Schlüssellänge von 2048 Bit als sicher. Als Dezimalzahl ausgedrückt, ist dies eine Zahl von $ 3,2 * 10^{616} $
Die Angriffe auf die RSA Verschlüsselung zielen darauf ab, den Geheimschlüssel zu ermitteln. Das Problem dabei ist die Primfaktoren Zerlegung von grossen Zahlen. Mehr dazu im Kapitel Angriffe %Verweis%

\subsection{Formel Entschlüsselung}
Die Entschlüsselung der Nachricht wird mit folgender Formel realisiert
%
$ m \equiv C^d \bmod N $
%
Der Klartext m hängt in diesem Fall vom verschlüsselten Text C, dem privaten Schlüssel d und dem RSA-Modul N ab. 
%
\subsection{Beweis der Funktionsweise}
Jede Verschlüsselung muss bei der korrekten Entschlüsselung des Geheimtextes wieder den Klartext ergeben. Wir möchten hier mathematisch beweisen, dass der RSA-Algorithmus korrekt funktioniert.\\

\subsubsection{Verschlüsselung und Entschlüsselung gleichsetzen}
Zuerst müssen wir die Entschlüsselung und Verschlüsselung in einer Gleichung kombinieren. Dazu dienen uns die Formeln zur Verschlüsselung und Entschlüsselung\\

\begin{align}
  C = m^e mod N \\
  m = C^d mod N
\end{align}
Wir lösen Entschlüsselungsformel nach C auf:
\begin{align}
  m = C^d mod N \\
  m + k * N = C^d \\
  \sqrt[d]{m+k*N} = C \\
  C = \sqrt[d]{m+k*N}
\end{align}
Wir setzen C der Verschlüsselungsformel mit C der Entschlüsselungsformel gleich:
%$ m^e mod N = \sqrt[d]{m+k*N} $ \\
\begin{align}
  m^e mod N = \sqrt[d]{m+k*N}\\
  m^{e*d} mod N = m + k * N
  m^{e*d} mod N = m
\end{align}
Die Formel $ m^{e*d} mod N = m + k * N $ kann gekürzt werden, da k in jedem Fall 0 sein muss. Wir arbeiten auf beiden Seiten mit Modulo N. Eine Zahl mod N kann 1 bis N-1 gross sein.
Wenn die Zahl N gross wär, würde es ja keinen Rest geben. Daher muss k in diesem Fall 0 sein. 


\subsubsection{Grundlagen zur Erklärung}
Das RSA-Modul N wird aus den ausgewählten Primzahlen p und q erstellt.
\begin{align}
  N = p * q
\end{align}

e wurde teilerfremd zu $ \varphi(n) $ gewählt. 
$ \varphi(n) = (p-1)*(q-1) $

Zusätzlich wurde d so gewählt, dass folgendes zählt:
\begin{align}
 e * d + k * \varphi(N) = 1 = ggT(e,\varphi(N))
\end{align}

Für den Beweis müssen wir ausschweifen auf den Satz von Euler Fermat. Dieser bildet die Grundlage 
zur Verschlüsselung. Er lautet wie folgt:
\begin{align}
	a^{\varphi(n)} \equiv 1\,(\mathrm{mod}\,n)
\end{align}
Da wir mit Primzahlen arbeiten, kann dieser Satz auch anders ausgedrückt werden. $ \varphi(n) $ gibt alle Teilerfremden Zahlen zu n an. Primzahlen 
sind nur die Zahl selbst und 1 teilbar. Dabei wird 1 jedoch ebenfalls als Teilerfremde Zahl gewertet. Daher bleiben für $ \varphi(n) = n-1 $ Möglichkeiten übrig.
\part{Der RSA-Algorithmus}
\section{Mathematisches Verfahren}
Der RSA-Algorithmus baut grundsätzlich auf Primzahlen, Modulo, Eulersche-Funktion, Satz von Euler und den erweiterten euklidischen Algorithmus auf. Verstehn man diese mathematischen Verfahren nicht, kann man auch nicht den RSA-Algorithmus verstehen und verwenden. \\
Im nächsten Abschnitt wollen wir die einzelnen mathematischen Verfahren kurz in ihren Funktionen erläutern.  
\subsection{Modulo}
Modulo oder auch \textit{Division mit Rest} gibt den Rest von einer ganzzahligen Division an. Modulo erklärt sich am einfachsten an einem Beispiel.\\
$ 21 / 5 = 4.2 $ \\
4.2 ist keine ganze Zahl. $ 4 * 5 = 20 $. \\
Bis auf 21 fehlt genau 1. Somit ist $ 21 \bmod 5 = 1 $
\paragraph{Kongruent Modulo}
Kongruent modulo heisst nichts anders, als dass zwei Zahlen modulo eine Zahl den selben Rest haben. Man sagt auch, dass sie in der gleichen Restklasse sind.\\
Als Beispiel nehmen wir die Zahlen 9 und 7 die modulo 2 gerechnet werden.\\
\begin{center}
$ 9 \bmod(2) = 1 $ \\
$ 7 \bmod(2) = 1 $ \\
%$ 9 \bmod(2) $ ist kongruent zu $ 7 \bmod(2) $ \\
$ 9 \bmod(2) \equiv 7 \bmod(2) $ \\
\end{center}
Denn beide ergeben als Resultat 1.
\subsection{Eulersche Funktion}
Die Eulersche Funktion sagt aus, wieviele teilfremde \footnote{Zahlen sind sich teilfremd, wenn ihr grösster gemeinsamer Teiler 1 ist.} natürliche Zahlen zu einer Zahl es gibt.\\
Für den RSA-Algorithmus verwenden wir die Eulersche Funktion nur bei Primzahlen. Deshalb schauen wir uns nur die spezial Formel bei Primzahlen an.\\
Primzahlen sind nur durch sich selber und 1 dividirbar und ihr grösster gemeinsamer Teiler ist immer 1. Somit ist die Eulersche Funktion einer Primzahl p immer p - 1.
\begin{center}
$ \varphi(p) = p - 1 $
\end{center}
Wenn wir jetzt also zwei verschiedene Primzahlen haben, gilt
\begin{center}
$ \varphi(pq) = (p - 1) * (q - 1) $
\end{center}
Der Beweis dafür liefert folgende Überlugungen. Es gibt insgesamt $ p * q -1 $ ganze Zahlen die kleiner sind als $ p * q $. Es ist bei Primzahlen einfacher die \textbf{nicht teilfremden} Zahlen zu zählen und diese dann von allen möglichen Zahlen abzuziehen als allte teilfremden Zahlen zu zählen. Nicht teilfremd zu p sind $ (q - 1) * p $ und für q gilt $ (p - 1) * q $\\
Als Formel geschrieben
\begin{center}
$ \varphi(pq) = p * q -1 - (p - 1) - (q - 1) $ \\
$ \varphi(pq) = p * q -1 - p + 1 - q + 1 $ \\
$ \varphi(pq) = p * q -q - p + 1 $ \\
$ \varphi(pq) = (p -1) * (q - 1) $ 
\end{center}
\subsection{Satz von Euler}
Der Satz von Euler sagt über zwei teilfremde Zahlen folgende Gleichung aus
\begin{center}
$ m^{\varphi(n)} \bmod(n) = 1 $
\end{center}
Nehem wir zwei Primzahlen p = 3 und q = 5. Und als m wählen wir 4 und setzten das ganze in unsere Formel ein.\\
\begin{center}
$ 4^{\varphi(3 * 5)} \bmod(3 * 5) = 1 $ \\
$ 4^8 \bmod(15) = 1 $ \\
\end{center}
Wenn wir jetzt noch eine natürliche Zahl k nehmen, gilt für diese folgende Aussage \\
\begin{center}
$ m^{1 + k * \varphi(n)} \bmod(n)  = m * 1$ \\
%$ m*m^{k * \varphi(n)} \bmod(n)  = m$ \\
\end{center}
Hat man zwei verschiedene Primzahlen p und q und eine natürliche Zahl m die kleiner ist als p * q. Dann gilt für jede natürliche Zahl k \\
\begin{center}
\boxed{m^{k * (p - 1) * (q - 1) +1} \bmod(p * q) = m}
\end{center}
Dieser Satz wird in der Mathematik auch als \textit{kleiner Satz von Fermat} genannt.
\subsection{Euklidischer Algorithmus}
Der Euklidischer Algorithmus errechnet den grössten gemeinsamen Teiler (ggT) von zwei natürlichen Zahlen. Um den ggT herauszufinden gibt es eine ältere das Subtraktions-Verfahren und eine modernere Methode das Modulo-Verfahren. Da das ältere Verfahren nicht so effizient ist, konzentrieren wir uns auf das moderne Verfahren.\\
Die Formel für das modernere Verfahren $a = q * p + r$ ist eine modulo Rechnung. Der Rest r wird zum neuen b und b wird zum neuen a. \\
Zu veranschaulichung nehmen wir zwei Zahlen a = 5984 und b = 302 und setzen diese in usere Gleichung ein.
\begin{center}
$ 5984 = 19 * 302 + 146 $\\
$ 302 = 2 * 146 + 10 $\\
$ 146 = 14 * 10 + 6 $ \\
$ 10 = 1 * 6 + 4 $ \\
$ 6 = 1 * 4 + 2 $ \\
$ 4 = 2 * 2 + 0 $\\
\end{center}
Der grösstegGemeinsame Teiler von 5984 und 19 ist somit 2. \\[2ex]
Um dieses Verfahren zu beweisen gehen wir davon aus das a, b, r und q alles natürliche Zahlen sind. Da wir im zweiten Schritt den grössten gemeinsamen Teiler von b und r errechnet haben, kann man folgende Gleichung aufstellen.
\begin{center}
$ ggT(a,b) = ggT(b,r) $\\
\end{center}
Wenn also eine Zahl a und b teilt, muss diese Zahl auch r teilen. Denn sonst würde die Gleichung oben nicht aufgehen. Umgekehrt muss eine Zahl die b und r teilt auch a teilen. Gehen wir zurück zu unserem Beispiel. Da diese Gleichung stimmt, stimmt auch unseres Verfahren.
\begin{center}
$ ggT(5984,302) = ggT(302,146) = ggT(146,10) = ggT(10,6) = ggT(6,4) = 2 $
\end{center}
\subsubsection{Erweiterter Euklidischer Algorithmus}
%





\newpage
\section{RSA Ver- und Entschlüsselung}
Der Name RSA setzt sich aus den Anfangsbuchstaben der Nachnamen der Entwickler zusammen. Das RSA Verfahren wird auch heute noch oft eingesetzt. Das Verfahren gilt bei einer bestimmten Schlüssellänge als sehr sicher. Es gibt verschiedenste Angriffsmöglichkeiten, diese führen jedoch nicht in einem vernünftigen Zeitraum zu Ergebnissen.\\
Das RSA Verfahren ist verwandt mit dem Rabin-Kryptosystem, dass auch auf Primzahlen beruht. Das Verfahren wird zur Signierung und zur Verschlüsselung verwendet. %Verweis%

\subsection{Der Schlüssel}
Dieses Kapitel wurde mit Hilfe des Buches \textit{RSA and Public-Key Cryptography} erarbeitet.\\[2ex]
%
Für eine sichere Verschlüsslung braucht es auf einen guten Schlüssel. Denn nur mit diesem Schlüssel, kann man die verschlüsselte Nachricht entschlüsseln. Der Schlüssel spielt also eine zentrale und wichtige Rolle. Dazu muss er so komplex sein, dass man ihn nicht erraten kann, oder durch Ausprobieren entdeckt.\\
Darum ist es auch sehr wichtig, dass man Zufallszahlen nimmt, die man nicht vorhersagen kann, beziehungsweise im Nachhinein berechnen kann.
%
\subsubsection{Zufallszahlen generieren}
In der Informatik sind Zufallszahlen eine sehr komplexe Angelegenheit. Um solche Zufallszahlen zu generieren, haben grössere Firmen einen Zufallszahlengeneratoren (eng. \textit{random number generator} RNG) entwickelt. Diese messen den radioaktiven Zerfall oder beobachteten die atmosphärischen Bedingungen in der Umgebung. Die Zahlen, die der RNG auswertet, nennt man \textbf{Seed-Zahlen}. Mit den gewonnen Seed-Zahlen, kann man jetzt einen Algorithums füttern, der eine Zahl oder eine Zahlenfolge zurückgibt. Hat man eine solche Zufallszahl, muss noch überprüft werden, ob es eine Primzahl ist. Zur überprüfung gibt es extra Verfahren, die wir hier nicht weiter erleutern werden. Mehr Informationen dazu im Buch \textit{RSA and Public-Key Cryptography - Kapitel 4}.\\
%
Für den normalen Benutzer sind solche Apparate viel zu aufwendig und zu teuer. Deshalb entwickelten Computerhersteller einen Pseudozufallszahlengenerator (eng. \textit{pseudo random number generator} PRNG). Dieser PRNG erstellt die Seed-Zahlen zum Beispiel aus der momentanen Position der Maus, momentane CPU Auslastung oder andere nicht vorhersehbare Elemente.\\
Ob die Zufallszahlen geeignet sind oder nicht ist schwierig zu sagen. Denn wie will man eine Folge von Zufallszahlen beurteilen?\\
Das Einzige, was man errechnen kann, ist die Entopie\footnote{Die Entropie beschreibt das Mass der Unordnung.}. Je höher die Entropie, desto unwarscheinlicher ist es, dass diese Zahlenfolge ein zweites Mal vorkommt.
%Buch RSA and Public-Key Seite 61 
\subsubsection{RSA-Schlüssel generieren}
Für eine RSA-Verschlüsselung brauchen wir zwei Schlüssel, die voneinander abhängig sind. Einen private Key und einen public Key. Im nächsten Abschnitt werden wir erläutern, wie man diese zwei Schlüsser errechnen kann.
%
\paragraph{Public-Key erstellen}\label{sec:public_key}
Als erstes brauchen wir zwei Primzahlen p und q die etwa gleich lang sind. Je grösser desto besser und sicherer. Wie man solche Primzahlen erstellt möchten wir hier nicht erläutern. Es gibt Methoden, bei denen man mit grosser Wahrscheinlichkeit eine Primzahl generieren kann. Man muss die Zahl nachher immer überprüfen.\\
Sind beide Primzahlen gefunden, errechnen wir uns N auch als RSA-Modul genannt.
%
\begin{equation}
  N = p \cdot q
  \label{eqn:rsa_modul}
\end{equation}
%
Als nächstes rechnen wir mit der Eulerschen Funktion [\ref{eqn:eulersche_func}] $\varphi$(N) von p und q aus.\\
Zu der Zahl $\varphi$(N) nehmen wir eine weitere zufällige, \textbf{teilfremde} Zahl e die den RSA verschlüsselungs Exponent bildet.\\
So nun haben wir unseren öffentlichen Schlüssel mit den Zahlen N und e.
\paragraph{Private-Key erstellen}
Wir haben vorher bei der Erstellung des public Keys die Zahlen N und e errechnet. Als nächstes wird aus diesen beiden Zahlen d errechent. d wird als enschlüsselungs Exponent im pirvate Key benötigt.\\
Mit dem erweiterten Euklidischen Algorithmus [\ref{eqn:erw_euklid_algo}] wird d berechnet so dass die Formel
%
% TODO: Auf modulare inverse hinweisen!!!!!
\begin{flalign*}
  e * d \bmod(\varphi(n)) = 1\\
  e * d \equiv 1 (\bmod \varphi(N) )
\end{flalign*}
%
stimmt. \\
Nachdem d ausgerechnet wurde, ist man im Besitz seines privaten Schlüssel mit den Zahlen d und N.
%\subsection{Schlüssel austauschen} Sprengt den Rahmen!!!!

\section{RSA Verschlüsselung}
Die RSA Verschlüsselung wurde 1977 von Ronald L. Riverst, Adi Shamir und Leonard Adleman entwickelt. Der Name RSA setzt sich aus den Anfangsbuchstaben der Nachnamen der Entwickler zusammen. Es handelt sich um ein aymetrisches Verfahren. (Siehe Kapitel moderne Kryptographie) %Verweis%
Das RSA Verfahren wird auch heute noch oft eingesetzt. Das Verfahren gilt bei einer bestimmten Schlüssellänge als sicher. Es gibt verschiedenste Angriffsmöglichkeiten, diese führen jedoch nicht in einem vernünftigen Zeitraum zu Ergebnissen. Das Verfahren wird zur Signierung und zur Verschlüsselung verwendet. (Siehe Anwendung) %Verweis%

\subsection{Formel Verschlüsselung}
Wenn der Schlüssel erstellt wurde kann ein Text einfach verschlüsselt werden. Dazu dient folgende Formel:

$ C \equiv m^e  \bmod N $

C steht für Cipher und bezeichnet den Geheimtext. Dieser ist abhängig vom Klartext m, dem öffentlichen Schlüssel e und dem RSA-Modul N. 

\section{RSA Entschlüsselung}
Die RSA Entschlüsselung ist ohne den Geheimschlüssel zu wissen nach aktuellem Wissensstand nicht möglich. Da die Sicherheit von diesem Geheimschlüssel abhängt, gilt es diesen möglichst lang zu wählen und geheim zu halten. Aktuell gilt eine Schlüssellänge von 2048 Bit als sicher. Als Dezimalzahl ausgedrückt, ist dies eine Zahl von $ 3,2 * 10^{616} $
Die Angriffe auf die RSA Verschlüsselung zielen darauf ab, den Geheimschlüssel zu ermitteln. Das Problem dabei ist die Primfaktoren Zerlegung von grossen Zahlen. Mehr dazu im Kapitel Angriffe %Verweis%

\subsection{Formel Entschlüsselung}
Die Entschlüsselung der Nachricht wird mit folgender Formel realisiert
%
$ m \equiv C^d \bmod N $
%
Der Klartext m hängt in diesem Fall vom verschlüsselten Text C, dem privaten Schlüssel d und dem RSA-Modul N ab. 
%
\subsection{Beweis der Funktionsweise}
Jede Verschlüsselung muss bei der korrekten Entschlüsselung des Geheimtextes wieder den Klartext ergeben. Wir möchten hier mathematisch beweisen, dass der RSA-Algorithmus korrekt funktioniert.\\

\subsubsection{Verschlüsselung und Entschlüsselung gleichsetzen}
Zuerst müssen wir die Entschlüsselung und Verschlüsselung in einer Gleichung kombinieren. Dazu dienen uns die Formeln zur Verschlüsselung und Entschlüsselung\\
\begin{align}
  C = m^e mod N \\
  m = C^d mod N
\end{align}
Wir lösen Entschlüsselungsformel nach C auf:
\begin{align}
  m = C^d mod N \\
  m + k * N = C^d \\
  \sqrt[d]{m+k*N} = C \\
  C = \sqrt[d]{m+k*N}
\end{align}
Wir setzen C der Verschlüsselungsformel mit C der Entschlüsselungsformel gleich:
\begin{align}
  m^e mod N = \sqrt[d]{m+k*N}\\
  m^{e*d} mod N = m + k * N
\end{align}

\subsubsection{Grundlagen zur Erklärung}
Das RSA-Modul N wird aus den ausgewählten Primzahlen p und q erstellt.
\begin{align}
  N = p * q
\end{align}
Zusätzlich wurde e und d so gewählt, dass folgendes zählt:
\begin{align}
 e * d \bmod \varphi(n) = 1\\
 e * d = 1 + k * \varphi(n)
\end{align}
Dies liegt daran, dass e und d als Multiplikativ Inverses von $ \varphi(N) $ gewählt wurde. (Siehe Schlüssel erstellen) \\%Verweis%

% Hier weiter machen
\subsubsection{Beweis der Funktionsweise des RSA-Algorithmus}


% Alt %
Es wird folgende Behauptung aufgestellt: \\
$ m^{e*d} \equiv m (mod n) $

Eine Nachricht zu verschlüsseln und danach wieder zu entschlüsseln, ergibt wieder die Nachricht selbst. Dabei ergibt der Modulo mit dem RSA-Modul immer noch die selbe Nachricht

Für unser RSA-Modul N haben wir zwei Primzahlen multipliziert. Zu diesem RSA-Modul sind alle Zahlen Teilerfremd, ausser sie sind durch p oder q teilbar. 
$ \varphi(n) $ kann auch mit $ (p-1) * (q-1) $ ausgedrückt werden. Folgender abgeleiteter Satz entsteht 
\begin{align}
  a^{(p-1) * (q-1)} \equiv 1\,(\mathrm{mod}\,p*q)
\end{align}

Durch den erweiterten euklidischen Algorithmus wurde d bestimmt. Dazu haben wir folgende Gleichung verwendet:
\begin{align}   
	e \cdot d \equiv 1 \pmod{\varphi(N)} 
\end{align}


\subsubsection{Beweis der Funktionsweise}
Wir machen folgende Aussage:
\begin{align}   
 m^{e*d} mod N = m
\end{align}

Dazu drücken wir e*d anders aus. Wir nehmen die Formel zum die multiplikative Inverse auszudrücken und lösen diese nach $ e*d $ auf:\\
$ e \cdot d \equiv 1 \bmod{\varphi(N)} $
% \begin{align}
%e \cdot d \equiv 1 \bmod{\varphi(N)} 
 % Kuster fragen wie das gemeint ist?
%   Wegen e*d=1 (mod j(n)) gibt es ein k Î N mit e*d=k* j(n)+1
%\end{align}

Nun ersetzen wir in unserer Aussage $ e*d $ durch $ k* \varphi(n)+1 $
\begin{center}
$ m^{ k* \varphi(n) +1} = m $ \\
$ m^{k* \varphi(n)} * m = m $ \\
$ { m^{ \varphi(n) }} ^k * m = m $ \\
\end{center}
% \begin{align}
%   m^{k*\varphi(n)+1} = m
%   m^{k*\varphi(n)} * m = m
%   m^{\varphi(n)}^k * m = m
% \end{align}

Durch den Satz von Fermat wissen wir dass $ \varphi(n) $ 1 sein muss:
\begin{align}
  m^{(p-1)} = 1 \bmod p
  m^{(p-1)*(q-1)} = 1^{q-1} \bmod p*q
  m^{(p-1)*(q-1)} = 1 \bmod p*q
  m^{\varphi(N)} = 1 \bmod N
\end{align}
1. Eine Zahl hoch die Anzahl der Teilerfremden Zahlen bei einer Primzahl ergibt 1 mod die Zahl
2. Wir rechnen beide Seiten hoch (q-1), was in unserem Fall die zweite Primzahl ist % a = b mod m ist wie a^k = b^k mod m
3. 1 hoch eine Zahl ergibt immer 1

Schlussfolgerung daraus:
% \begin{align}
%	m^{\varphi(n)}^k * m = m
%	1(mod N)^k * m = m
%	1^k * m = m
%	1 * m = m
% \end{align}
 
 Der Algorithmus funktioniert korrekt. 


Für den Beweis brauchen wir die Primzahlen p und q. Aus diesem wurde das RSA-Modul N gebildet
$ n = p * q $

Zusätzlich wurde und e und d so gewählt, dass folgendes zählt:
$ e * d \bmod \varphi(n) = 1 $

Lösen wir dies nach e*d kommt k hinzu:
$ e * d = 1 + k * \varphi(n) $
Die Modulo-Operation ergibt den Rest, welcher bei der Division durch eine Zahl anfällt. In unserem Fall ist der Rest immer 1.

Nun können wir e * d einsetzen:
$ m^e*d = m^k*\varphi(n)+1 $

% Alt 2 %

Die allgemeine Formel zur Entschlüsselung ist:
$ m \equiv C^d \bmod N $

Setzen wir nun für den verschlüsselten Text die verschlüsselte Form von K ein muss dies immer noch K ergeben.
$ m \equiv (m^e)^d \bmod N $

$ m \equiv m^{e*d} \bmod N $
Wenn diese Bedingung erfüllt ist, arbeitet der Algorithmus korrekt.

Um die Richtigkeit zu beweisen kommen wir auf die Anfangszahlen p und q zurück, welche das RSA-Modul N bilden. Es handelt sich um Primzahlen. Wir zeigen das Vorgehen anhand von p:
$ 0 \equiv m^{e*d} \bmod p - K \bmod p $

$ 0 \equiv (m^e*d-m) \bmod p $

Da wir e und d so gewählt haben, dass 
$ ed \bmod \varphi(n) = 1 $ 
%Verweis% 


\section{Beispiel}
\subsection{Einfaches Zahlenbeispiel}
\subsection{Komplexeres Zahlenbeispiel}
\subsection{Beispiel an einem Text}
\section{Anwendung}
\subsection{SSH - Secure Shell}
