\part{Der RSA-Algorithmus}
\section{Mathematisches Verfahren}
Dieses Kapitel wurde mit Hilfe des Buches \textit{Moderne Verfahren der Kryptographie}\cite{mod_kry}, \textit{Zahlentheorie für Einsteiger}\cite{zahlentheorie_fuer_einsteiger} und \textit{Kryptologie}\cite{kryptologie} erarbeitet.\\[2ex]
%
Der RSA-Algorithmus baut grundsätzlich auf Primzahlen, der Modulo und Eulerschen-Funktion, dem Satz von Euler und dem erweiterten euklidischen Algorithmus auf.\\
Im nächsten Abschnitt werden die einzelnen mathematischen Verfahren kurz in ihren Funktionen erläutert.\\
Wir werden für die beiden Primzahlen durchgehend die Variablen $p$ und $q$ verwenden. $N$ steht immer für das Produkt von $p \cdot q$. Die Variable $r$ wird für das Resultat der Modulo-Berechnungen verwendet. $e$ steht für den Verschlüsselungs-Exponenten und $d$ für den Entschlüsselungs-Exponenten.
%
\subsection{Modulo}
Modulo oder auch \textit{Division mit Rest} gibt den ganzzahligen Rest einer Division zweier natürlichen Zahlen an. Er wird beim RSA-Algorithmus bei der Berechnung des Entschlüsselungs-Exponenten sowie in der Verschlüsselung verwendet.\\
Modulo erklärt sich am Einfachsten an einem Beispiel:
%
\begin{equation*}
  21 \bmod(5) = 1
\end{equation*}
%
Es gilt herauszufinden, wie viel Mal 5 ganz in 21 vorhanden ist. Dafür teilen wir 21 durch 5 und schneiden den Dezimalrest ab. Um den ganzzahligen Rest zu berechnen subtrahieren wir von 21 das Produkt aus 4 mal 5.
%
\begin{flalign*}
  21 \colon 5 &= 4.2\\
  21 - 4 \cdot 5 &= 1\\
  21 &= 4 \cdot 5 + 1
\end{flalign*}
%Oder für den RSA Algorithmus veralgemeinert:
%\begin{equation}
%  \varphi(N) = q \cdot e + r
%  \label{eqn:mod_rsa}
%\end{equation}
%
\paragraph{Kongruent Modulo}
Kongruent Modulo bedeutet nichts anders, als dass zwei verschiedene Zahlen Modulo einer gleichen Zahl den selben Rest haben. Man sagt auch, dass sie in der gleichen Restklasse sind.\\
Als Beispiel nehmen wir die Zahlen 9 und 7 die Modulo 2 gerechnet werden.
%
\begin{flalign*}
  9 \bmod(2) &= 1 \\
  7 \bmod(2) &= 1  \\
  9 & \equiv 7 \bmod(2)
\end{flalign*}
%
%******************************************************************************
% Eulersche Funktion
%******************************************************************************
%Verbessern die Zahlen sind kleiner als die Zahl
\subsection{Eulersche Funktion}

Die Eulersche Funktion zählt die natürlichen, teilerfremden\footnote{Zwei Zahlen sind teilerfremd zueinander, wenn ihr grösster gemeinsamer Teiler 1 ist.}  Zahlen von n, die kleiner als n sind.\\
Für die Eulersche Funktion stellen die Primzahlen einen Spezialfall dar.
Primzahlen sind nur durch 1 und sich selbst ohne Rest teilbar. Somit ist das Resultat der die Eulerschen Funktion einer Primzahl p immer $p - 1$.\\
Um den privaten Schlüssel zu errechnen, muss man $\varphi(N)$ beziehungsweise $\varphi(pq)$ berechnen. Aus diesem Grund erläutern wir die Eulersche Funktion:
%
\begin{equation*}
  \varphi(p) = p - 1
\end{equation*}
%
Der grösste gemeinsame Teiler zweier Primzahlen ist immer 1.
Für die Eulersche Funktion zweier Primzahlen gilt:
\begin{equation}
  \varphi(pq) = \varphi(p) \cdot \varphi(q) = (p - 1) \cdot (q - 1)
  \label{eqn:eulersche_func}
\end{equation}
Der Beweis dafür liefert folgende Überlegung. Es gibt insgesamt $p \cdot q -1$ ganze Zahlen die kleiner sind als $p \cdot q$.\\
Es ist bei Primzahlen einfacher, die \textbf{nicht teilerfremden} Zahlen zu zählen und diese dann von allen möglichen Zahlen abzuziehen, als alle teilerfremden Zahlen zu zählen. Nicht teilerfremd zu p sind $(q - 1) \cdot p$ und für q gilt $ (p - 1) \cdot q$. \cite{kryptologie}\\
Als Formel geschrieben:
%
\begin{equation*}
  \begin{split}
    \varphi(pq) & = p \cdot q -1 - (p - 1) - (q - 1)  \\
     & = p \cdot q -1 - p + 1 - q + 1  \\
     & = p \cdot q -q - p + 1  \\
     & = (p -1) \cdot (q - 1)
    \label{eqn:herleitung_eulersche_func}
  \end{split}
\end{equation*}
%
% Satz von Euler
%=====================
%Verweis auf Buch Beutelspacher
\subsection{Satz von Euler}
Der Satz von Euler gewährleistet die korrekte Ver- und Entschlüsselung.\cite{zahlentheorie_fuer_einsteiger}. Wir gehen hier nicht weiter auf den Satz von Euler ein und werden später beim Beweis nochmals auf ihn zurückkommen.\\
Der Satz von Euler sagt über zwei natürliche, teilerfremde Zahlen, hier $n$ und $m$, folgendes aus: \cite{kryptologie}
%
\begin{equation}
  m^{\varphi(n)} \bmod(n) = 1
  \label{eqn:satz_von_euler}
\end{equation}
%
Nehmen wir für $n = 15$ und $m = 4$ und setzen Formel [\ref{eqn:satz_von_euler}] ein.
%
% Zeigen das varphi 15 = 8 ist
\begin{flalign*}
  4^{\varphi(15)} \bmod(15) = 1  \\
  4^8 \bmod(15) = 1
\end{flalign*}
%
%
% Kleiner Satz von Verma
%
%Wenn wir jetzt eine natürliche Zahl k nehmen, gilt für diese folgende Gleichung (Erleuterung siehe Buch \textit{Kryptologie}):
%
%\begin{equation}
%  \begin{split}
%    m^{1 + k \cdot \varphi(N)} \bmod(N)  &= m \cdot 1 \\
%    m \cdot m^{k \cdot \varphi(N)} \bmod(N) &= m \cdot 1
%    \label{eqn:satz_von_euler_erweitert}
%  \end{split}
%\end{equation}
%
%
%Hat man zwei verschiedene Primzahlen p und q und eine natürliche Zahl m, die kleiner ist als $p \cdot q$, dann gilt für jede natürliche Zahl k:
%
%\begin{equation}
%  m^{k \cdot (p - 1) \cdot (q - 1) +1} \bmod(p \cdot q) = m
%  \label{eqn:kleiner_satz_fermat}
%\end{equation}
%
%Dieser Satz wird in der Mathematik auch \textit{kleiner Satz von Fermat} genannt.
% Buch S 109 / 110
%
%******************************************************************************
% Euklidischer Algorithmus
%******************************************************************************
\subsection{Euklidischer Algorithmus}\label{euklidischer_Algorithmus}
Mit den Euklidischen Algorithmen lässt sich der grösste gemeinsame Teiler (ggT) zweier natürlichen Zahlen berechnen. Es existiert das etwas ältere Subtraktionsverfahren und das moderne Modulo-Verfahren. Da das Subtraktions-Verfahren nicht die gleiche Effizienz hat wie das Subtraktionsverfahren, konzentrieren wir uns auf das Modulo-Verfahren. \cite{zahlentheorie_fuer_einsteiger}\\
Für den RSA-Algorithmus ist es wichtig, dass wir zwei teilerfremde Zahlen $\varphi(N)$ und $e$ haben.\\
%
Die Formel für das Modulo-Verfahren lautet:
%
\begin{equation}
  \label{eqn:euklidischer_algo}
  a = b \cdot q + r 
\end{equation}
%
wobei
\begin{equation*}
  0 \leq r \leq e - 1
\end{equation*}
gelten.\\
%
Auf den RSA-Algorithmus bezogen sieht die Formel so aus:
\begin{equation}
  \label{eqn:euklidischer_algo_RSA}
  \varphi(N) = q \cdot e + r 
\end{equation}
%
Dabei sind $\varphi(N)$ und e die beiden Zahlen von denen wir den grössten gemeinsamen Teiler ausrechnen.
Wir gehen davon aus das $\varphi(N)$ grösser ist als e. Wäre das nicht der Fall, müsste e auf die linke Seite des Gleichheitszeichen.
Zur Veranschaulichung nehmen wir zwei Zahlen $\varphi(N) = 839$ und $e = 199$ und setzen diese in unsere Gleichung ein.
\begin{enumerate}
  \item Schritt: Zahlen in Gleichung einsetzen, q und r ausrechnen.\\
    \begin{equation*}
      839 = 4 \cdot 199 + 43
    \end{equation*}
  \item Schritt: 199 als neues $\varphi(N)$ und 43 als neues e einsetzten und wieder q und r ausrechnen.\\
    \begin{equation*}
      199 = 4 \cdot 43 + 27
    \end{equation*}
\end{enumerate}
Den zweiten Schritt solange wiederholen bis man auf Rest 1 oder 0 kommt. Falls die Zahl 0 herauskommt, nehmen wir den Rest der vorangegangenen Gleichung als grössten gemeinsamen Teiler.\\
%
Ganzer Ablauf:
\begin{equation*}
  \begin{split}
    839 &= 4 \cdot 199 + 43\\
    199 &= 4 \cdot 43 + 27\\
    43 &= 1 \cdot 27 + 16\\
    27 &= 1 \cdot 16 + 11\\
    16 &= 1 \cdot 11 + 5\\
    11 &= 2 \cdot 5 + 1\\
    \label{eqn:euqulid_beweis}
  \end{split}
\end{equation*}
%
Der grösste gemeinsame Teiler von 839 und 199 ist 1. Sie sind daher teilerfremd. \\[2ex]
Um dieses Verfahren zu beweisen, gehen wir davon aus, dass $\varphi(N)$, e, r und q alles natürliche nicht negative Zahlen sind. Da wir im zweiten Schritt den grössten gemeinsamen Teiler von $e$ und r errechnet haben, kann man basierend auf Formel \ref{eqn:euklidischer_algo_RSA} folgende Behauptung aufstellen:
%
\begin{equation}
  ggT(\varphi(N),e) = ggT(e,r)
  \label{eqn:ggT}
\end{equation}
%
Nur der grösste gemeinsame Teiler von $\varphi(N)$ und $e$ teilt sowohl $\varphi(N)$ als auch e und r ohne Rest. Auf unser Beispiel angewendet ergibt das $ggT(839,199) \leq ggT(199,43)$.\\
Und es gilt: Nur der grösste gemeinsame Teiler von e und r teilt auch $\varphi(N)$ ohne Rest.\\
Das heisst: $ggT(199,43) \leq ggT(839,199)$.
Somit stimmen die beiden grössten gemeinsamen Teiler überein\cite{zahlentheorie_fuer_einsteiger}.\\[2ex]
Gehen wir zurück zu unserem Beispiel:
%
\begin{equation*}
 ggT(839,199) = ggT(199,43) = ggT(43,27) = ggT(27,16) = ggT(16,11) = ggT(11,5) = 1
\end{equation*}
%
%******************************************************************************
% Erweiterter Euklidischer Algorithmus
%******************************************************************************
\subsubsection{Erweiterter Euklidischer Algorithmus}
Der erweiterte Euklidische Algorithmus berechnet neben dem grössten gemeinsamen Teiler $r$ zweier natürlichen Zahlen auch noch die dazugehörigen natürlichen Zahlen d und k, dass folgende Gleichung stimmt:
%
\begin{equation}
  ggT(a,b) = s \cdot a + t \cdot b
  \label{eqn:erw_euklid_algo}
\end{equation}
%
Auf den RSA-Algorithmus angewendet sieht die Formel so aus:
%
\begin{equation}
  r = d \cdot e + k \cdot \varphi(N) 
  \label{eqn:erw_euklid_algo_RSA}
\end{equation}
%
Mit seiner Hilfe lässt sich der Entschlüsselungs-Exponent d berechnen, welchen wir für die Berechnung des privaten Schlüssels brauchen.\\
Dieser Algorithmus wird auch \textit{Vielfachsummendarstellung} genannt, denn um auf die Zahlen d und k zu kommen, muss man die Vielfachsummendarstellung anwenden.\\
Es ist eine erweiterte Form des Euklidischen Algorithmus.\\
Wir schreiben jetzt die Rechnung \ref{eqn:euqulid_beweis} so um, dass wir den Rest auf einer Seite isolieren.\\ 
\begin{equation}
  r = \varphi(N) - a \cdot e
  \label{eqn:form_erw_euklid}
\end{equation}
%
\begin{flalign*}
  43 &= 839 - 4 \cdot 199\\
  27 &= 199 - 4 \cdot 43\\
  16 &= 43 - 1 \cdot 27\\
  11 &= 27 - 1 \cdot 16\\
  5 &= 16 - 1 \cdot 11\\
  1 &= 11 - 2 \cdot 5
  \label{eqn:erw_euklid_10}
\end{flalign*}
%
% TODO: AUSSCHREIBEN
Man startet bei der untersten Zeile und ersetzt die Zahl, die den nächsten oberen Rest beschreibt, durch die Gleichung. Das heisst man ersetzt in der ersten Gleichung die Zahl $5$ mit dem Term $16 - 1 \cdot 11$. Das wiederholt man so lange, bis man die letzte Gleichung eingesetzt hat. 
Es ist zu beachten, dass auf der rechten Seite eine Summe aus zwei Produkten stehen bleibt.
%
\begin{flalign*}
  1  &= 11 - 2 \cdot 5 = 11 - 2 \cdot 16 + 2 \cdot 11 = 3 \cdot 11 - 2 \cdot 16 \\
  &= 3 \cdot 11 - 2 \cdot 16 = 3 \cdot 27 - 3 \cdot 1 \cdot 16 - 2 \cdot 16 = 3 \cdot 27 - 5 \cdot 16\\
  &= 3 \cdot 27 - 5 \cdot 16 = 3 \cdot 27 - 5 \cdot 43 + 5 \cdot 1 \cdot 27 = 8 \cdot 27 - 5 \cdot 43\\
  &= 8 \cdot 27 - 5 \cdot 43 = 8 \cdot 199 - 8 \cdot 4 \cdot 43 - 5 \cdot 43 = 8 \cdot 199 - 37 \cdot 43\\
  &= 8 \cdot 199 - 37 \cdot 43 = 8 \cdot 199 - 37 \cdot 839 + 37 \cdot 4 \cdot 199 = 156 \cdot 199 - 37 \cdot 839
\end{flalign*}
%
Am Schluss bekommen wir die Formel
%
\begin{equation*}
 1 = 156 \cdot 199 + (-37) \cdot 839
 \label{eqn:erw_euklid_end}
\end{equation*}
%
Damit wir jetzt den Entschlüsselungs-Exponenten d bekommen, müssen wir die Gleichung in die Modulare Inverse umschreiben. Wir berechnen somit die Modulare Inverse von $\varphi(N)$ und e.\\
Die Formel der modularen Inverse auf den RSA-Algorithmus angewendet lautet:
\begin{equation}
  e \cdot d \bmod(\varphi(N)) \equiv 1
\end{equation}
%
Wir gehen von Formel \ref{eqn:erw_euklid_end} aus und setzen den Term $(-37) \cdot 839$ auf die linke Seite.
%
\begin{equation*}
 1 - 37 \cdot 839 = 156 \cdot 199
\end{equation*}
Damit man den nächsten Schritt versteht, muss man wissen, dass man eine Gleichung
\begin{equation*}
  1 + k \cdot \varphi(N) = d \cdot e
\end{equation*}
mit Modulo ausdrücken kann.
\begin{equation*}
  1 \equiv d \cdot e \bmod(\varphi(N))
  \label{eqn:erw_eukl_fertig}
\end{equation*}
%
Aufgrund dessen können wir die Modulare Inverse von $\varphi(N)$ und e so ausdrücken:
%
\begin{equation*}
 1 \equiv 156 \cdot 199 \bmod(839)
\end{equation*}
Wir erhalten den Entschlüsselungs-Exponent $d = 156$.
%
%Um das zu beweisen schauen wir uns nochmal die Formel \ref{eqn:euklidischer_algo} an. Aus einer Vielfachsummendarstellung von $q$ und $r$, lässt sich dei Vielfachsummendarstellung von $p$ und $q$ ableiten. Man kann die Formel
%{eqn:ggT}
%\begin{equation*}
%  2 = x \cdot p + y \cdot q
%  \label{eqn:erw_eukl_fertig}
%\end{equation*}
%
%auch mit r und a darstellen.
%
%\begin{equation*}
%  2 = x \cdot r + y \cdot a
%\end{equation*}
%
%Jetzt setzen wir die Formel \ref{eqn:form_erw_euklid} anstatt r ein und erhalten diese Gleichung
%\begin{equation*}
%  2 = x \cdot (p - a \cdot q)  + y \cdot a = x \cdot p + (y - xa) \cdot q
%\end{equation*}
%
%Somit haben wie die Vielfachsummendarstellung für $p$ und $q$.
%
%\subsubsection{Modulare Inverse}
%Haben wir zwei natürliche, teilerfremde Zahlen, gibt es eine Zahl x, sodass die Formel
%
%\begin{equation*}
%  p \cdot x \bmod(q) \equiv 1
%\end{equation*}
%
%Es gibt zwei ganze Zahlen $x$ und $y$ sodass die Gleichung \ref{eqn:erw_eukl_fertig} entsteht.
%Weil $y cdot q$ durch $q$ teilbar ist, gibt es bei der Divison von $x \cdot p$ durch $q$ immer 1. Somit kann man sagen
%
%\begin{equation}
%  p \cdot x \bmod q = 1
%  \label{eqn:mod_inverse}
%\end{equation}
%
%Die Modulareinverse von zwei teilerfremden natürlichen Zahlen $p$ und $q$, errechnet man mit dem erweiterten Euklidischen Algorithmus. Dann ist $x$ die modulare Inverse von $p$ und $q$.


\section{Der Schlüssel}
Für eine sichere Verschlüsslung kommt es auf den Schlüssel an. Denn nur mit diesem Schlüssel, kann man die verschlüsselte Nachricht enschlüsseln. Der Schlüssel spielt also eine zentrale Rolle und muss sicher sein.

\subsection{Sicherer Schlüssel generieren}
Für einen sicheren Schlüssel sollte man immer Zufallszahlen verwenden. Zufallszahlen sind Zahlenfolgen bei denen man die nächste Zahl nicht durch mathematische Berechunungen vorhersagen kann - also Zufällige Zahlen. 
\subsubsection{Zufallszahlen generieren}
Um solche Zufallszahlen zu generieren hat man Zufallszahlengeneratoren (engl random number generator = RNG) gebaut. Diese messten den Radioaktiven Zerfall oder beobachteten die atmosphärischen Bedingungen in der Umgebung. Die Zahlen die der RNG zurückgibt, kann man dann Verwenden um eine Zufallszahl zu generieren, da der Output auf einem Input basiert, der sich permanent ändert und nicht wiederholbar ist.

Was, wenn man keine solchen Messgeräte hat, die solche Zahlen generieren können. 
Für das gibt es Pseudozufallszahlengeneratoren (engl pseudo random number generator = PRNG)

\paragraph{Entropie}
\paragraph{Zahlentheoretische Funktionen}
\subsection{Schlüssel austauschen}

\section{RSA Verschlüsselung}
Die RSA Verschlüsselung wurde 1977 von Ronald L. Riverst, Adi Shamir und Leonard Adleman entwickelt. Der Name RSA setzt sich aus den Anfangsbuchstaben der Nachnamen der Entwickler zusammen. Es handelt sich um ein aymetrisches Verfahren. (Siehe Kapitel moderne Kryptographie) %Verweis%
Das RSA Verfahren wird auch heute noch oft eingesetzt. Das Verfahren gilt bei einer bestimmten Schlüssellänge als sicher. Es gibt verschiedenste Angriffsmöglichkeiten, diese führen jedoch nicht in einem vernünftigen Zeitraum zu Ergebnissen. Das Verfahren wird zur Signierung und zur Verschlüsselung verwendet. (Siehe Anwendung) %Verweis%

\subsection{Formel Verschlüsselung}
Wenn der Schlüssel erstellt wurde kann ein Text einfach verschlüsselt werden. Dazu dient folgende Formel:

$ C \equiv m^e  \bmod N $

C steht für Cipher und bezeichnet den Geheimtext. Dieser ist abhängig vom Klartext m, dem öffentlichen Schlüssel e und dem RSA-Modul N. 

\section{RSA Entschlüsselung}
Die RSA Entschlüsselung ist ohne den Geheimschlüssel zu wissen nach aktuellem Wissensstand nicht möglich. Da die Sicherheit von diesem Geheimschlüssel abhängt, gilt es diesen möglichst lang zu wählen und geheim zu halten. Aktuell gilt eine Schlüssellänge von 2048 Bit als sicher. Als Dezimalzahl ausgedrückt, ist dies eine Zahl von $ 3,2 * 10^{616} $
Die Angriffe auf die RSA Verschlüsselung zielen darauf ab, den Geheimschlüssel zu ermitteln. Das Problem dabei ist die Primfaktoren Zerlegung von grossen Zahlen. Mehr dazu im Kapitel Angriffe %Verweis%

\subsection{Formel Entschlüsselung}
Die Entschlüsselung der Nachricht wird mit folgender Formel realisiert
%
$ m \equiv C^d \bmod N $
%
Der Klartext m hängt in diesem Fall vom verschlüsselten Text C, dem privaten Schlüssel d und dem RSA-Modul N ab. 
%
\subsection{Beweis der Funktionsweise}
Jede Verschlüsselung muss bei der korrekten Entschlüsselung des Geheimtextes wieder den Klartext ergeben. Wir möchten hier mathematisch beweisen, dass der RSA-Algorithmus korrekt funktioniert.\\

\subsubsection{Verschlüsselung und Entschlüsselung gleichsetzen}
Zuerst müssen wir die Entschlüsselung und Verschlüsselung in einer Gleichung kombinieren. Dazu dienen uns die Formeln zur Verschlüsselung und Entschlüsselung\\
\begin{align}
  C = m^e mod N \\
  m = C^d mod N
\end{align}
Wir lösen Entschlüsselungsformel nach C auf:
\begin{align}
  m = C^d mod N \\
  m + k * N = C^d \\
  \sqrt[d]{m+k*N} = C \\
  C = \sqrt[d]{m+k*N}
\end{align}
Wir setzen C der Verschlüsselungsformel mit C der Entschlüsselungsformel gleich:
\begin{align}
  m^e mod N = \sqrt[d]{m+k*N}\\
  m^{e*d} mod N = m + k * N
  m^{e*d} mod N = m
\end{align}
Die Formel $ m^{e*d} mod N = m + k * N $ kann gekürzt werden, da k in jedem Fall 0 sein muss. Wir arbeiten auf beiden Seiten mit Modulo N. Eine Zahl mod N kann 1 bis N-1 gross sein.
Wenn die Zahl N gross wär, würde es ja keinen Rest geben. Daher muss k in diesem Fall 0 sein. 


\subsubsection{Grundlagen zur Erklärung}
Das RSA-Modul N wird aus den ausgewählten Primzahlen p und q erstellt.
\begin{align}
  N = p * q
\end{align}

e wurde teilerfremd zu $ \varphi(n) $ gewählt. 
$ \varphi(n) = (p-1)*(q-1) $

Zusätzlich wurde d so gewählt, dass folgendes zählt:
\begin{align}
 e * d + k * \varphi(N) = 1 = ggT(e,\varphi(N))
\end{align}

Für den Beweis müssen wir ausschweifen auf den Satz von Euler Fermat. Dieser bildet die Grundlage 
zur Verschlüsselung. Er lautet wie folgt:
\begin{align}
	a^{\varphi(n)} \equiv 1\,(\mathrm{mod}\,n)
\end{align}
Da wir mit Primzahlen arbeiten, kann dieser Satz auch anders ausgedrückt werden. $ \varphi(n) $ gibt alle Teilerfremden Zahlen zu n an. Primzahlen 
sind nur die Zahl selbst und 1 teilbar. Dabei wird 1 jedoch ebenfalls als Teilerfremde Zahl gewertet. Daher bleiben für $ \varphi(n) = n-1 $ Möglichkeiten übrig.
\part{Der RSA-Algorithmus}
\section{Mathematisches Verfahren}
Dieses Kapitel wurde mit Hilfe des Buches \textit{Moderne Verfahren der Kryptographie}\cite{mod_kry}, \textit{Zahlentheorie für Einsteiger}\cite{zahlentheorie_fuer_einsteiger} und \textit{Kryptologie}\cite{kryptologie} erarbeitet.\\[2ex]
%
Der RSA-Algorithmus baut grundsätzlich auf Primzahlen, der Modulo und Eulerschen-Funktion, dem Satz von Euler und dem erweiterten euklidischen Algorithmus auf.\\
Im nächsten Abschnitt werden die einzelnen mathematischen Verfahren kurz in ihren Funktionen erläutert.\\
Wir werden für die beiden Primzahlen durchgehend die Variablen $p$ und $q$ verwenden. $N$ steht immer für das Produkt von $p \cdot q$. Die Variable $r$ wird für das Resultat der Modulo-Berechnungen verwendet. $e$ steht für den Verschlüsselungs-Exponenten und $d$ für den Entschlüsselungs-Exponenten.
%
\subsection{Modulo}
Modulo oder auch \textit{Division mit Rest} gibt den ganzzahligen Rest einer Division zweier natürlichen Zahlen an. Er wird beim RSA-Algorithmus bei der Berechnung des Entschlüsselungs-Exponenten sowie in der Verschlüsselung verwendet.\\
Modulo erklärt sich am Einfachsten an einem Beispiel:
%
\begin{equation*}
  21 \bmod(5) = 1
\end{equation*}
%
Es gilt herauszufinden, wie viel Mal 5 ganz in 21 vorhanden ist. Dafür teilen wir 21 durch 5 und schneiden den Dezimalrest ab. Um den ganzzahligen Rest zu berechnen subtrahieren wir von 21 das Produkt aus 4 mal 5.
%
\begin{flalign*}
  21 \colon 5 &= 4.2\\
  21 - 4 \cdot 5 &= 1\\
  21 &= 4 \cdot 5 + 1
\end{flalign*}
%Oder für den RSA Algorithmus veralgemeinert:
%\begin{equation}
%  \varphi(N) = q \cdot e + r
%  \label{eqn:mod_rsa}
%\end{equation}
%
\paragraph{Kongruent Modulo}
Kongruent Modulo bedeutet nichts anders, als dass zwei verschiedene Zahlen Modulo einer gleichen Zahl den selben Rest haben. Man sagt auch, dass sie in der gleichen Restklasse sind.\\
Als Beispiel nehmen wir die Zahlen 9 und 7 die Modulo 2 gerechnet werden.
%
\begin{flalign*}
  9 \bmod(2) &= 1 \\
  7 \bmod(2) &= 1  \\
  9 & \equiv 7 \bmod(2)
\end{flalign*}
%
%******************************************************************************
% Eulersche Funktion
%******************************************************************************
%Verbessern die Zahlen sind kleiner als die Zahl
\subsection{Eulersche Funktion}

Die Eulersche Funktion zählt die natürlichen, teilerfremden\footnote{Zwei Zahlen sind teilerfremd zueinander, wenn ihr grösster gemeinsamer Teiler 1 ist.}  Zahlen von n, die kleiner als n sind.\\
Für die Eulersche Funktion stellen die Primzahlen einen Spezialfall dar.
Primzahlen sind nur durch 1 und sich selbst ohne Rest teilbar. Somit ist das Resultat der die Eulerschen Funktion einer Primzahl p immer $p - 1$.\\
Um den privaten Schlüssel zu errechnen, muss man $\varphi(N)$ beziehungsweise $\varphi(pq)$ berechnen. Aus diesem Grund erläutern wir die Eulersche Funktion:
%
\begin{equation*}
  \varphi(p) = p - 1
\end{equation*}
%
Der grösste gemeinsame Teiler zweier Primzahlen ist immer 1.
Für die Eulersche Funktion zweier Primzahlen gilt:
\begin{equation}
  \varphi(pq) = \varphi(p) \cdot \varphi(q) = (p - 1) \cdot (q - 1)
  \label{eqn:eulersche_func}
\end{equation}
Der Beweis dafür liefert folgende Überlegung. Es gibt insgesamt $p \cdot q -1$ ganze Zahlen die kleiner sind als $p \cdot q$.\\
Es ist bei Primzahlen einfacher, die \textbf{nicht teilerfremden} Zahlen zu zählen und diese dann von allen möglichen Zahlen abzuziehen, als alle teilerfremden Zahlen zu zählen. Nicht teilerfremd zu p sind $(q - 1) \cdot p$ und für q gilt $ (p - 1) \cdot q$. \cite{kryptologie}\\
Als Formel geschrieben:
%
\begin{equation*}
  \begin{split}
    \varphi(pq) & = p \cdot q -1 - (p - 1) - (q - 1)  \\
     & = p \cdot q -1 - p + 1 - q + 1  \\
     & = p \cdot q -q - p + 1  \\
     & = (p -1) \cdot (q - 1)
    \label{eqn:herleitung_eulersche_func}
  \end{split}
\end{equation*}
%
% Satz von Euler
%=====================
%Verweis auf Buch Beutelspacher
\subsection{Satz von Euler}
Der Satz von Euler gewährleistet die korrekte Ver- und Entschlüsselung.\cite{zahlentheorie_fuer_einsteiger}. Wir gehen hier nicht weiter auf den Satz von Euler ein und werden später beim Beweis nochmals auf ihn zurückkommen.\\
Der Satz von Euler sagt über zwei natürliche, teilerfremde Zahlen, hier $n$ und $m$, folgendes aus: \cite{kryptologie}
%
\begin{equation}
  m^{\varphi(n)} \bmod(n) = 1
  \label{eqn:satz_von_euler}
\end{equation}
%
Nehmen wir für $n = 15$ und $m = 4$ und setzen Formel [\ref{eqn:satz_von_euler}] ein.
%
% Zeigen das varphi 15 = 8 ist
\begin{flalign*}
  4^{\varphi(15)} \bmod(15) = 1  \\
  4^8 \bmod(15) = 1
\end{flalign*}
%
%
% Kleiner Satz von Verma
%
%Wenn wir jetzt eine natürliche Zahl k nehmen, gilt für diese folgende Gleichung (Erleuterung siehe Buch \textit{Kryptologie}):
%
%\begin{equation}
%  \begin{split}
%    m^{1 + k \cdot \varphi(N)} \bmod(N)  &= m \cdot 1 \\
%    m \cdot m^{k \cdot \varphi(N)} \bmod(N) &= m \cdot 1
%    \label{eqn:satz_von_euler_erweitert}
%  \end{split}
%\end{equation}
%
%
%Hat man zwei verschiedene Primzahlen p und q und eine natürliche Zahl m, die kleiner ist als $p \cdot q$, dann gilt für jede natürliche Zahl k:
%
%\begin{equation}
%  m^{k \cdot (p - 1) \cdot (q - 1) +1} \bmod(p \cdot q) = m
%  \label{eqn:kleiner_satz_fermat}
%\end{equation}
%
%Dieser Satz wird in der Mathematik auch \textit{kleiner Satz von Fermat} genannt.
% Buch S 109 / 110
%
%******************************************************************************
% Euklidischer Algorithmus
%******************************************************************************
\subsection{Euklidischer Algorithmus}\label{euklidischer_Algorithmus}
Mit den Euklidischen Algorithmen lässt sich der grösste gemeinsame Teiler (ggT) zweier natürlichen Zahlen berechnen. Es existiert das etwas ältere Subtraktionsverfahren und das moderne Modulo-Verfahren. Da das Subtraktions-Verfahren nicht die gleiche Effizienz hat wie das Subtraktionsverfahren, konzentrieren wir uns auf das Modulo-Verfahren. \cite{zahlentheorie_fuer_einsteiger}\\
Für den RSA-Algorithmus ist es wichtig, dass wir zwei teilerfremde Zahlen $\varphi(N)$ und $e$ haben.\\
%
Die Formel für das Modulo-Verfahren lautet:
%
\begin{equation}
  \label{eqn:euklidischer_algo}
  a = b \cdot q + r 
\end{equation}
%
wobei
\begin{equation*}
  0 \leq r \leq e - 1
\end{equation*}
gelten.\\
%
Auf den RSA-Algorithmus bezogen sieht die Formel so aus:
\begin{equation}
  \label{eqn:euklidischer_algo_RSA}
  \varphi(N) = q \cdot e + r 
\end{equation}
%
Dabei sind $\varphi(N)$ und e die beiden Zahlen von denen wir den grössten gemeinsamen Teiler ausrechnen.
Wir gehen davon aus das $\varphi(N)$ grösser ist als e. Wäre das nicht der Fall, müsste e auf die linke Seite des Gleichheitszeichen.
Zur Veranschaulichung nehmen wir zwei Zahlen $\varphi(N) = 839$ und $e = 199$ und setzen diese in unsere Gleichung ein.
\begin{enumerate}
  \item Schritt: Zahlen in Gleichung einsetzen, q und r ausrechnen.\\
    \begin{equation*}
      839 = 4 \cdot 199 + 43
    \end{equation*}
  \item Schritt: 199 als neues $\varphi(N)$ und 43 als neues e einsetzten und wieder q und r ausrechnen.\\
    \begin{equation*}
      199 = 4 \cdot 43 + 27
    \end{equation*}
\end{enumerate}
Den zweiten Schritt solange wiederholen bis man auf Rest 1 oder 0 kommt. Falls die Zahl 0 herauskommt, nehmen wir den Rest der vorangegangenen Gleichung als grössten gemeinsamen Teiler.\\
%
Ganzer Ablauf:
\begin{equation*}
  \begin{split}
    839 &= 4 \cdot 199 + 43\\
    199 &= 4 \cdot 43 + 27\\
    43 &= 1 \cdot 27 + 16\\
    27 &= 1 \cdot 16 + 11\\
    16 &= 1 \cdot 11 + 5\\
    11 &= 2 \cdot 5 + 1\\
    \label{eqn:euqulid_beweis}
  \end{split}
\end{equation*}
%
Der grösste gemeinsame Teiler von 839 und 199 ist 1. Sie sind daher teilerfremd. \\[2ex]
Um dieses Verfahren zu beweisen, gehen wir davon aus, dass $\varphi(N)$, e, r und q alles natürliche nicht negative Zahlen sind. Da wir im zweiten Schritt den grössten gemeinsamen Teiler von $e$ und r errechnet haben, kann man basierend auf Formel \ref{eqn:euklidischer_algo_RSA} folgende Behauptung aufstellen:
%
\begin{equation}
  ggT(\varphi(N),e) = ggT(e,r)
  \label{eqn:ggT}
\end{equation}
%
Nur der grösste gemeinsame Teiler von $\varphi(N)$ und $e$ teilt sowohl $\varphi(N)$ als auch e und r ohne Rest. Auf unser Beispiel angewendet ergibt das $ggT(839,199) \leq ggT(199,43)$.\\
Und es gilt: Nur der grösste gemeinsame Teiler von e und r teilt auch $\varphi(N)$ ohne Rest.\\
Das heisst: $ggT(199,43) \leq ggT(839,199)$.
Somit stimmen die beiden grössten gemeinsamen Teiler überein\cite{zahlentheorie_fuer_einsteiger}.\\[2ex]
Gehen wir zurück zu unserem Beispiel:
%
\begin{equation*}
 ggT(839,199) = ggT(199,43) = ggT(43,27) = ggT(27,16) = ggT(16,11) = ggT(11,5) = 1
\end{equation*}
%
%******************************************************************************
% Erweiterter Euklidischer Algorithmus
%******************************************************************************
\subsubsection{Erweiterter Euklidischer Algorithmus}
Der erweiterte Euklidische Algorithmus berechnet neben dem grössten gemeinsamen Teiler $r$ zweier natürlichen Zahlen auch noch die dazugehörigen natürlichen Zahlen d und k, dass folgende Gleichung stimmt:
%
\begin{equation}
  ggT(a,b) = s \cdot a + t \cdot b
  \label{eqn:erw_euklid_algo}
\end{equation}
%
Auf den RSA-Algorithmus angewendet sieht die Formel so aus:
%
\begin{equation}
  r = d \cdot e + k \cdot \varphi(N) 
  \label{eqn:erw_euklid_algo_RSA}
\end{equation}
%
Mit seiner Hilfe lässt sich der Entschlüsselungs-Exponent d berechnen, welchen wir für die Berechnung des privaten Schlüssels brauchen.\\
Dieser Algorithmus wird auch \textit{Vielfachsummendarstellung} genannt, denn um auf die Zahlen d und k zu kommen, muss man die Vielfachsummendarstellung anwenden.\\
Es ist eine erweiterte Form des Euklidischen Algorithmus.\\
Wir schreiben jetzt die Rechnung \ref{eqn:euqulid_beweis} so um, dass wir den Rest auf einer Seite isolieren.\\ 
\begin{equation}
  r = \varphi(N) - a \cdot e
  \label{eqn:form_erw_euklid}
\end{equation}
%
\begin{flalign*}
  43 &= 839 - 4 \cdot 199\\
  27 &= 199 - 4 \cdot 43\\
  16 &= 43 - 1 \cdot 27\\
  11 &= 27 - 1 \cdot 16\\
  5 &= 16 - 1 \cdot 11\\
  1 &= 11 - 2 \cdot 5
  \label{eqn:erw_euklid_10}
\end{flalign*}
%
% TODO: AUSSCHREIBEN
Man startet bei der untersten Zeile und ersetzt die Zahl, die den nächsten oberen Rest beschreibt, durch die Gleichung. Das heisst man ersetzt in der ersten Gleichung die Zahl $5$ mit dem Term $16 - 1 \cdot 11$. Das wiederholt man so lange, bis man die letzte Gleichung eingesetzt hat. 
Es ist zu beachten, dass auf der rechten Seite eine Summe aus zwei Produkten stehen bleibt.
%
\begin{flalign*}
  1  &= 11 - 2 \cdot 5 = 11 - 2 \cdot 16 + 2 \cdot 11 = 3 \cdot 11 - 2 \cdot 16 \\
  &= 3 \cdot 11 - 2 \cdot 16 = 3 \cdot 27 - 3 \cdot 1 \cdot 16 - 2 \cdot 16 = 3 \cdot 27 - 5 \cdot 16\\
  &= 3 \cdot 27 - 5 \cdot 16 = 3 \cdot 27 - 5 \cdot 43 + 5 \cdot 1 \cdot 27 = 8 \cdot 27 - 5 \cdot 43\\
  &= 8 \cdot 27 - 5 \cdot 43 = 8 \cdot 199 - 8 \cdot 4 \cdot 43 - 5 \cdot 43 = 8 \cdot 199 - 37 \cdot 43\\
  &= 8 \cdot 199 - 37 \cdot 43 = 8 \cdot 199 - 37 \cdot 839 + 37 \cdot 4 \cdot 199 = 156 \cdot 199 - 37 \cdot 839
\end{flalign*}
%
Am Schluss bekommen wir die Formel
%
\begin{equation*}
 1 = 156 \cdot 199 + (-37) \cdot 839
 \label{eqn:erw_euklid_end}
\end{equation*}
%
Damit wir jetzt den Entschlüsselungs-Exponenten d bekommen, müssen wir die Gleichung in die Modulare Inverse umschreiben. Wir berechnen somit die Modulare Inverse von $\varphi(N)$ und e.\\
Die Formel der modularen Inverse auf den RSA-Algorithmus angewendet lautet:
\begin{equation}
  e \cdot d \bmod(\varphi(N)) \equiv 1
\end{equation}
%
Wir gehen von Formel \ref{eqn:erw_euklid_end} aus und setzen den Term $(-37) \cdot 839$ auf die linke Seite.
%
\begin{equation*}
 1 - 37 \cdot 839 = 156 \cdot 199
\end{equation*}
Damit man den nächsten Schritt versteht, muss man wissen, dass man eine Gleichung
\begin{equation*}
  1 + k \cdot \varphi(N) = d \cdot e
\end{equation*}
mit Modulo ausdrücken kann.
\begin{equation*}
  1 \equiv d \cdot e \bmod(\varphi(N))
  \label{eqn:erw_eukl_fertig}
\end{equation*}
%
Aufgrund dessen können wir die Modulare Inverse von $\varphi(N)$ und e so ausdrücken:
%
\begin{equation*}
 1 \equiv 156 \cdot 199 \bmod(839)
\end{equation*}
Wir erhalten den Entschlüsselungs-Exponent $d = 156$.
%
%Um das zu beweisen schauen wir uns nochmal die Formel \ref{eqn:euklidischer_algo} an. Aus einer Vielfachsummendarstellung von $q$ und $r$, lässt sich dei Vielfachsummendarstellung von $p$ und $q$ ableiten. Man kann die Formel
%{eqn:ggT}
%\begin{equation*}
%  2 = x \cdot p + y \cdot q
%  \label{eqn:erw_eukl_fertig}
%\end{equation*}
%
%auch mit r und a darstellen.
%
%\begin{equation*}
%  2 = x \cdot r + y \cdot a
%\end{equation*}
%
%Jetzt setzen wir die Formel \ref{eqn:form_erw_euklid} anstatt r ein und erhalten diese Gleichung
%\begin{equation*}
%  2 = x \cdot (p - a \cdot q)  + y \cdot a = x \cdot p + (y - xa) \cdot q
%\end{equation*}
%
%Somit haben wie die Vielfachsummendarstellung für $p$ und $q$.
%
%\subsubsection{Modulare Inverse}
%Haben wir zwei natürliche, teilerfremde Zahlen, gibt es eine Zahl x, sodass die Formel
%
%\begin{equation*}
%  p \cdot x \bmod(q) \equiv 1
%\end{equation*}
%
%Es gibt zwei ganze Zahlen $x$ und $y$ sodass die Gleichung \ref{eqn:erw_eukl_fertig} entsteht.
%Weil $y cdot q$ durch $q$ teilbar ist, gibt es bei der Divison von $x \cdot p$ durch $q$ immer 1. Somit kann man sagen
%
%\begin{equation}
%  p \cdot x \bmod q = 1
%  \label{eqn:mod_inverse}
%\end{equation}
%
%Die Modulareinverse von zwei teilerfremden natürlichen Zahlen $p$ und $q$, errechnet man mit dem erweiterten Euklidischen Algorithmus. Dann ist $x$ die modulare Inverse von $p$ und $q$.


\section{Der Schlüssel}
Für eine sichere Verschlüsslung kommt es auf den Schlüssel an. Denn nur mit diesem Schlüssel, kann man die verschlüsselte Nachricht enschlüsseln. Der Schlüssel spielt also eine zentrale Rolle und muss sicher sein.

\subsection{Sicherer Schlüssel generieren}
Für einen sicheren Schlüssel sollte man immer Zufallszahlen verwenden. Zufallszahlen sind Zahlenfolgen bei denen man die nächste Zahl nicht durch mathematische Berechunungen vorhersagen kann - also Zufällige Zahlen. 
\subsubsection{Zufallszahlen generieren}
Um solche Zufallszahlen zu generieren hat man Zufallszahlengeneratoren (engl random number generator = RNG) gebaut. Diese messten den Radioaktiven Zerfall oder beobachteten die atmosphärischen Bedingungen in der Umgebung. Die Zahlen die der RNG zurückgibt, kann man dann Verwenden um eine Zufallszahl zu generieren, da der Output auf einem Input basiert, der sich permanent ändert und nicht wiederholbar ist.

Was, wenn man keine solchen Messgeräte hat, die solche Zahlen generieren können. 
Für das gibt es Pseudozufallszahlengeneratoren (engl pseudo random number generator = PRNG)

\paragraph{Entropie}
\paragraph{Zahlentheoretische Funktionen}
\subsection{Schlüssel austauschen}

\section{RSA Verschlüsselung}
Die RSA Verschlüsselung wurde 1977 von Ronald L. Riverst, Adi Shamir und Leonard Adleman entwickelt. Der Name RSA setzt sich aus den Anfangsbuchstaben der Nachnamen der Entwickler zusammen. Es handelt sich um ein aymetrisches Verfahren. (Siehe Kapitel moderne Kryptographie) %Verweis%
Das RSA Verfahren wird auch heute noch oft eingesetzt. Das Verfahren gilt bei einer bestimmten Schlüssellänge als sicher. Es gibt verschiedenste Angriffsmöglichkeiten, diese führen jedoch nicht in einem vernünftigen Zeitraum zu Ergebnissen. Das Verfahren wird zur Signierung und zur Verschlüsselung verwendet. (Siehe Anwendung) %Verweis%

\subsection{Formel Verschlüsselung}
Wenn der Schlüssel erstellt wurde kann ein Text einfach verschlüsselt werden. Dazu dient folgende Formel:

$ C \equiv m^e  \bmod N $

C steht für Cipher und bezeichnet den Geheimtext. Dieser ist abhängig vom Klartext m, dem öffentlichen Schlüssel e und dem RSA-Modul N. 

\section{RSA Entschlüsselung}
Die RSA Entschlüsselung ist ohne den Geheimschlüssel zu wissen nach aktuellem Wissensstand nicht möglich. Da die Sicherheit von diesem Geheimschlüssel abhängt, gilt es diesen möglichst lang zu wählen und geheim zu halten. Aktuell gilt eine Schlüssellänge von 2048 Bit als sicher. Als Dezimalzahl ausgedrückt, ist dies eine Zahl von $ 3,2 * 10^{616} $
Die Angriffe auf die RSA Verschlüsselung zielen darauf ab, den Geheimschlüssel zu ermitteln. Das Problem dabei ist die Primfaktoren Zerlegung von grossen Zahlen. Mehr dazu im Kapitel Angriffe %Verweis%

\subsection{Formel Entschlüsselung}
Die Entschlüsselung der Nachricht wird mit folgender Formel realisiert
%
$ m \equiv C^d \bmod N $
%
Der Klartext m hängt in diesem Fall vom verschlüsselten Text C, dem privaten Schlüssel d und dem RSA-Modul N ab. 
%
\subsection{Beweis der Funktionsweise}
Jede Verschlüsselung muss bei der korrekten Entschlüsselung des Geheimtextes wieder den Klartext ergeben. Wir möchten hier mathematisch beweisen, dass der RSA-Algorithmus korrekt funktioniert.\\

\subsubsection{Verschlüsselung und Entschlüsselung gleichsetzen}
Zuerst müssen wir die Entschlüsselung und Verschlüsselung in einer Gleichung kombinieren. Dazu dienen uns die Formeln zur Verschlüsselung und Entschlüsselung\\
\begin{align}
  C = m^e mod N \\
  m = C^d mod N
\end{align}
Wir lösen Entschlüsselungsformel nach C auf:
\begin{align}
  m = C^d mod N \\
  m + k * N = C^d \\
  \sqrt[d]{m+k*N} = C \\
  C = \sqrt[d]{m+k*N}
\end{align}
Wir setzen C der Verschlüsselungsformel mit C der Entschlüsselungsformel gleich:
\begin{align}
  m^e mod N = \sqrt[d]{m+k*N}\\
  m^{e*d} mod N = m + k * N
\end{align}

\subsubsection{Grundlagen zur Erklärung}
Das RSA-Modul N wird aus den ausgewählten Primzahlen p und q erstellt.
\begin{align}
  N = p * q
\end{align}
Zusätzlich wurde e und d so gewählt, dass folgendes zählt:
\begin{align}
 e * d \bmod \varphi(n) = 1\\
 e * d = 1 + k * \varphi(n)
\end{align}
Dies liegt daran, dass e und d als Multiplikativ Inverses von $ \varphi(N) $ gewählt wurde. (Siehe Schlüssel erstellen) \\%Verweis%

% Hier weiter machen
\subsubsection{Beweis der Funktionsweise des RSA-Algorithmus}


% Alt %
Es wird folgende Behauptung aufgestellt: \\
$ m^{e*d} \equiv m (mod n) $

Eine Nachricht zu verschlüsseln und danach wieder zu entschlüsseln, ergibt wieder die Nachricht selbst. Dabei ergibt der Modulo mit dem RSA-Modul immer noch die selbe Nachricht

Für unser RSA-Modul N haben wir zwei Primzahlen multipliziert. Zu diesem RSA-Modul sind alle Zahlen Teilerfremd, ausser sie sind durch p oder q teilbar. 
$ \varphi(n) $ kann auch mit $ (p-1) * (q-1) $ ausgedrückt werden. Folgender abgeleiteter Satz entsteht 
\begin{align}
	a^{(p-1) * (q-1)} \equiv 1\,(\mathrm{mod}\,p*q)
\end{align}

Durch den erweiterten euklidischen Algorithmus wurde d bestimmt. Dazu haben wir folgende Gleichung verwendet:
\begin{align}   
	e \cdot d \equiv 1 \pmod{\varphi(N)} 
\end{align}


\subsubsection{Beweis der Funktionsweise}
Wir machen folgende Aussage:
\begin{align}   
 m^{e*d} mod N = m
\end{align}

Dazu drücken wir e*d anders aus. Wir nehmen die Formel zum die multiplikative Inverse auszudrücken und lösen diese nach e*d auf:
 \begin{align}    
	e \cdot d \equiv 1 \pmod{\varphi(N)} 
	
	% Kuster fragen wie das gemeint ist?
	Wegen e*d=1 (mod j(n)) gibt es ein k Î N mit e*d=k* j(n)+1
\end{align}

Nun ersetzen wir in unserer Aussage $ e*d $ durch $ k* \varphi(n)+1 $
 \begin{align}
	m^{k*\varphi(n)+1} = m
	m^{k*\varphi(n)} * m = m
	m^{\varphi(n)}^k * m = m
 \end{align}
 
Durch den Satz von Fermat wissen wir dass $ \varphi(n) $ 1 sein muss:
\begin{align}
  m^{(p-1)} = 1 \bmod p
  m^{(p-1)*(q-1)} = 1^{q-1} \bmod p*q
  m^{(p-1)*(q-1)} = 1 \bmod p*q
  m^{\varphi(N)} = 1 \bmod N
\end{align}
1. Eine Zahl hoch die Anzahl der Teilerfremden Zahlen bei einer Primzahl ergibt 1 mod die Zahl
2. Wir rechnen beide Seiten hoch (q-1), was in unserem Fall die zweite Primzahl ist % a = b mod m ist wie a^k = b^k mod m
3. 1 hoch eine Zahl ergibt immer 1

Schlussfolgerung daraus:
 \begin{align}
	m^{\varphi(n)}^k * m = m
	1(mod N)^k * m = m
	1^k * m = m
	1 * m = m
 \end{align}
 
 Der Algorithmus funktioniert korrekt. 



%Verweis% 


\section{Beispiel}
\subsection{Einfaches Zahlenbeispiel}
\subsection{Komplexeres Zahlenbeispiel}
\subsection{Beispiel an einem Text}
\section{Anwendung}
\subsection{SSH - Secure Shell}

Für den Beweis brauchen wir die Primzahlen p und q. Aus diesem wurde das RSA-Modul N gebildet
$ n = p * q $

Zusätzlich wurde und e und d so gewählt, dass folgendes zählt:
$ e * d \bmod \varphi(n) = 1 $

Lösen wir dies nach e*d kommt k hinzu:
$ e * d = 1 + k * \varphi(n) $
Die Modulo-Operation ergibt den Rest, welcher bei der Division durch eine Zahl anfällt. In unserem Fall ist der Rest immer 1.

Nun können wir e * d einsetzen:
$ m^e*d = m^k*\varphi(n)+1 $

% Alt 2 %

Die allgemeine Formel zur Entschlüsselung ist:
$ m \equiv C^d \bmod N $

Setzen wir nun für den verschlüsselten Text die verschlüsselte Form von K ein muss dies immer noch K ergeben.
$ m \equiv (m^e)^d \bmod N $

$ m \equiv m^{e*d} \bmod N $
Wenn diese Bedingung erfüllt ist, arbeitet der Algorithmus korrekt.

Um die Richtigkeit zu beweisen kommen wir auf die Anfangszahlen p und q zurück, welche das RSA-Modul N bilden. Es handelt sich um Primzahlen. Wir zeigen das Vorgehen anhand von p:
$ 0 \equiv m^{e*d} \bmod p - K \bmod p $

$ 0 \equiv (m^e*d-m) \bmod p $

Da wir e und d so gewählt haben, dass 
$ ed \bmod \varphi(n) = 1 $ 
%Verweis% 


\section{Beispiel}
\subsection{Einfaches Zahlenbeispiel}
\subsection{Komplexeres Zahlenbeispiel}
\subsection{Beispiel an einem Text}
\section{Anwendung}
\subsection{SSH - Secure Shell}
