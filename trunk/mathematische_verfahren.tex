\section{Mathematisches Verfahren}
Der RSA-Algorithmus baut grundsätzlich auf Primzahlen, Modulo, Eulersche-Funktion, Satz von Euler und den erweiterten euklidischen Algorithmus auf. Verstehn man diese mathematischen Verfahren nicht, kann man auch nicht den RSA-Algorithmus verstehen und verwenden. \\
Im nächsten Abschnitt wollen wir die einzelnen mathematischen Verfahren kurz in ihren Funktionen erläutern.  
\subsection{Modulo}
Modulo oder auch \textit{Division mit Rest} gibt den Rest von einer ganzzahligen Division an. Modulo erklärt sich am einfachsten an einem Beispiel.\\
$ 21 / 5 = 4.2 $ \\
4.2 ist keine ganze Zahl. $ 4 * 5 = 20 $. \\
Bis auf 21 fehlt genau 1. Somit ist $ 21 \bmod 5 = 1 $
\paragraph{Kongruent Modulo}
Kongruent modulo heisst nichts anders, als dass zwei Zahlen modulo eine Zahl den selben Rest haben. Man sagt auch, dass sie in der gleichen Restklasse sind.\\
Als Beispiel nehmen wir die Zahlen 9 und 7 die modulo 2 gerechnet werden.\\
\begin{center}
$ 9 \bmod(2) = 1 $ \\
$ 7 \bmod(2) = 1 $ \\
%$ 9 \bmod(2) $ ist kongruent zu $ 7 \bmod(2) $ \\
$ 9 \bmod(2) \equiv 7 \bmod(2) $ \\
\end{center}
Denn beide ergeben als Resultat 1.
\subsection{Eulersche Funktion}
Die Eulersche Funktion sagt aus, wieviele teilfremde \footnote{Zahlen sind sich teilfremd, wenn ihr grösster gemeinsamer Teiler 1 ist.} natürliche Zahlen zu einer Zahl es gibt.\\
Für den RSA-Algorithmus verwenden wir die Eulersche Funktion nur bei Primzahlen. Deshalb schauen wir uns nur die spezial Formel bei Primzahlen an.\\
Primzahlen sind nur durch sich selber und 1 dividirbar und ihr grösster gemeinsamer Teiler ist immer 1. Somit ist die Eulersche Funktion einer Primzahl p immer p - 1.
\begin{center}
$ \varphi(p) = p - 1 $
\end{center}
Wenn wir jetzt also zwei verschiedene Primzahlen haben, gilt
\begin{center}
$ \varphi(pq) = (p - 1) * (q - 1) $
\end{center}
Der Beweis dafür liefert folgende Überlugungen. Es gibt insgesamt $ p * q -1 $ ganze Zahlen die kleiner sind als $ p * q $. Es ist bei Primzahlen einfacher die \textbf{nicht teilfremden} Zahlen zu zählen und diese dann von allen möglichen Zahlen abzuziehen als allte teilfremden Zahlen zu zählen. Nicht teilfremd zu p sind $ (q - 1) * p $ und für q gilt $ (p - 1) * q $\\
Als Formel geschrieben
\begin{center}
$ \varphi(pq) = p * q -1 - (p - 1) - (q - 1) $ \\
$ \varphi(pq) = p * q -1 - p + 1 - q + 1 $ \\
$ \varphi(pq) = p * q -q - p + 1 $ \\
$ \varphi(pq) = (p -1) * (q - 1) $ 
\end{center}
\subsection{Satz von Euler}
Der Satz von Euler sagt über zwei teilfremde Zahlen folgende Gleichung aus
\begin{center}
$ m^{\varphi(n)} \bmod(n) = 1 $
\end{center}
Nehem wir zwei Primzahlen p = 3 und q = 5. Und als m wählen wir 4 und setzten das ganze in unsere Formel ein.\\
\begin{center}
$ 4^{\varphi(3 * 5)} \bmod(3 * 5) = 1 $ \\
$ 4^8 \bmod(15) = 1 $ \\
\end{center}
Wenn wir jetzt noch eine natürliche Zahl k nehmen, gilt für diese folgende Aussage \\
\begin{center}
$ m^{1 + k * \varphi(n)} \bmod(n)  = m * 1$ \\
%$ m*m^{k * \varphi(n)} \bmod(n)  = m$ \\
\end{center}
Hat man zwei verschiedene Primzahlen p und q und eine natürliche Zahl m die kleiner ist als p * q. Dann gilt für jede natürliche Zahl k \\
\begin{center}
\boxed{m^{k * (p - 1) * (q - 1) +1} \bmod(p * q) = m}
\end{center}
Dieser Satz wird in der Mathematik auch als \textit{kleiner Satz von Fermat} genannt.
\subsection{Euklidischer Algorithmus}
Der Euklidischer Algorithmus errechnet den grössten gemeinsamen Teiler (ggT) von zwei natürlichen Zahlen. Um den ggT herauszufinden gibt es eine ältere das Subtraktions-Verfahren und eine modernere Methode das Modulo-Verfahren. Da das ältere Verfahren nicht so effizient ist, konzentrieren wir uns auf das moderne Verfahren.\\
Die Formel für das modernere Verfahren $a = q * p + r$ ist eine modulo Rechnung. Der Rest r wird zum neuen b und b wird zum neuen a. \\
Zu veranschaulichung nehmen wir zwei Zahlen a = 5984 und b = 302 und setzen diese in usere Gleichung ein.
\begin{center}
$ 5984 = 19 * 302 + 146 $\\
$ 302 = 2 * 146 + 10 $\\
$ 146 = 14 * 10 + 6 $ \\
$ 10 = 1 * 6 + 4 $ \\
$ 6 = 1 * 4 + 2 $ \\
$ 4 = 2 * 2 + 0 $\\
\end{center}
Der grösstegGemeinsame Teiler von 5984 und 19 ist somit 2. \\[2ex]
Um dieses Verfahren zu beweisen gehen wir davon aus das a, b, r und q alles natürliche Zahlen sind. Da wir im zweiten Schritt den grössten gemeinsamen Teiler von b und r errechnet haben, kann man folgende Gleichung aufstellen.
\begin{center}
$ ggT(a,b) = ggT(b,r) $\\
\end{center}
Wenn also eine Zahl a und b teilt, muss diese Zahl auch r teilen. Denn sonst würde die Gleichung oben nicht aufgehen. Umgekehrt muss eine Zahl die b und r teilt auch a teilen. Gehen wir zurück zu unserem Beispiel. Da diese Gleichung stimmt, stimmt auch unseres Verfahren.
\begin{center}
$ ggT(5984,302) = ggT(302,146) = ggT(146,10) = ggT(10,6) = ggT(6,4) = 2 $
\end{center}
\subsubsection{Erweiterter Euklidischer Algorithmus}
%



