\section{Mathematisches Verfahren}
Dieses Kapitel wurde mit Hilfe von dem Buch \textit{Moderne Verfahren der Kryptographie} und \textit{Kryptologie} erarbeitet\\[2ex]
%
Der RSA-Algorithmus baut grundsätzlich auf Primzahlen, Modulo, Eulersche-Funktion, den Satz von Euler und den erweiterten euklidischen Algorithmus auf. Versteht man diese mathematischen Verfahren nicht, kann man ebensowenig den RSA-Algorithmus verstehen oder verwenden. \\
Im nächsten Abschnitt werden die einzelnen mathematischen Verfahren kurz in ihren Funktionen erläutert.  
\subsection{Modulo}
Modulo oder auch \textit{Division mit Rest} gibt den Rest von einer ganzzahligen Division an. Modulo erklärt sich am einfachsten an einem Beispiel.
%
\begin{equation*}
  21 \bmod(5) = 1
\end{equation*}
%
Man sucht die nächst kleinere Zahl die durch 5 eine natürliche Zahl ergibt. Der Rest, der bis zu 21 fehlt, ist dann das Resultat.
%
\begin{equation*}
  4 * 5 = 20
\end{equation*}
%
Bis auf 21 fehlt genau 1. Somit ist 
%
\begin{equation*}
  21 \bmod(5) = 1
\end{equation*}
%
\paragraph{Kongruent Modulo}
Kongruent Modulo bedeutet nichts anders, als dass zwei verschiedene Zahlen modulo einer gleichen Zahl den selben Rest haben. Man sagt auch, dass sie in der gleichen Restklasse sind.\\
Als Beispiel nehmen wir die Zahlen 9 und 7 die modulo 2 gerechnet werden.
%
\begin{flalign*}
  9 \bmod(2) &= 1 \\
  7 \bmod(2) &= 1  \\
  9 \bmod(2) & \equiv 7 \bmod(2)
\end{flalign*}
%
%******************************************************************************
% Eulersche Funktion
%******************************************************************************
%Verbessern die Zahlen sind kleiner als die Zahl -> HAHA
\subsection{Eulersche Funktion}
Die Eulersche Funktion sagt aus, wieviele teilfremde \footnote{Zahlen sind sich teilfremd, wenn ihr grösster gemeinsamer Teiler 1 ist.} natürliche Zahlen zu einer Zahl es gibt.(Wikipedia)\\
Für den RSA-Algorithmus verwenden wir die Eulersche Funktion nur bei Primzahlen. Deshalb schauen wir uns nur die Spezialformel bei Primzahlen an.\\
Primzahlen sind durch sich selbst und durch 1 dividierbar und ihr grösster gemeinsamer Teiler ist immer 1. Somit ist die Eulersche Funktion einer Primzahl p immer $p - 1$.
%
\begin{equation*}
  \varphi(p) = p - 1
\end{equation*}
%
Wenn wir jetzt also zwei verschiedene Primzahlen haben, gilt
\begin{equation}
  \varphi(pq) = (p - 1) \cdot (q - 1)
  \label{eqn:eulersche_func}
\end{equation}
Der Beweis dafür liefert folgende Überlegung. Es gibt insgesamt $p \cdot q -1$ ganze Zahlen die kleiner sind als $p \cdot q$.\\
Es ist bei Primzahlen einfacher, die \textbf{nicht teilfremden} Zahlen zu zählen und diese dann von allen möglichen Zahlen abzuziehen, als alle teilfremden Zahlen zu zählen. Nicht teilfremd zu p sind $(q - 1) \cdot p$ und für q gilt $ (p - 1) \cdot q$.\\
Als Formel geschrieben:
%
\begin{equation*}
  \begin{split}
    \varphi(pq) & = p \cdot q -1 - (p - 1) - (q - 1)  \\
    \varphi(pq) & = p \cdot q -1 - p + 1 - q + 1  \\
    \varphi(pq) & = p \cdot q -q - p + 1  \\
    \varphi(pq) & = (p -1) \cdot (q - 1)
    \label{eqn:herleitung_eulersche_func}
  \end{split}
\end{equation*}
%
% Satz von Euler
%=====================
\subsection{Satz von Euler}
Der Satz von Euler sagt über zwei teilfremde Zahlen Folgendes aus
%
\begin{equation}
  m^{\varphi(n)} \bmod(n) = 1
  \label{eqn:satz_von_euler}
\end{equation}
%
Nehmen wir zwei Primzahlen p = 3 und q = 5. Als m wählen wir 4 und setzten das ganze in die obige Formel ein.
%
\begin{flalign*}
  4^{\varphi(3 * 5)} \bmod(3 * 5) = 1  \\
  4^8 \bmod(15) = 1
\end{flalign*}
%
Wenn wir jetzt eine natürliche Zahl k nehmen, gilt für diese folgende Gleichung:
%
\begin{equation}
  m^{1 + k * \varphi(n)} \bmod(n)  = m * 1 \\
  \label{eqn:satz_von_euler_erweitert}
\end{equation}
%
Hat man zwei verschiedene Primzahlen p und q und eine natürliche Zahl m, die kleiner ist als $p \cdot q$, dann gilt für jede natürliche Zahl k:
%
\begin{equation}
  m^{k * (p - 1) * (q - 1) +1} \bmod(p * q) = m
  \label{eqn:kleiner_satz_fermat}
\end{equation}
%
Dieser Satz wird in der Mathematik auch \textit{kleiner Satz von Fermat} genannt.
% Buch S 109 / 110
%
%******************************************************************************
% Euklidischer Algorithmus
%******************************************************************************
\subsection{Euklidischer Algorithmus}
Der Euklidischer Algorithmus errechnet den grössten gemeinsamen Teiler (ggT) von zwei natürlichen Zahlen. Um den ggT herauszufinden gibt es eine ältere das Subtraktions-Verfahren und eine modernere Methode das Modulo-Verfahren. Da das ältere Verfahren nicht so effizient ist, konzentrieren wir uns auf das moderne Verfahren.\\
Die Formel für das modernere Verfahren 
%
\begin{equation}
  \label{eqn:euklidischer_algo}
  p = a \cdot q + r
\end{equation}
%
ist eine modulo Rechnung. Der Rest r wird zum neuen q und q wird zum neuen p. \\
Zur Veranschaulichung nehmen wir zwei Zahlen $q = 302$ und $p = 146$ und setzen diese in usere Gleichung ein.
%
\begin{equation}
  \begin{split}
    302 & = 2 \cdot 146 + 10 \\
    146 & = 14 \cdot 10 + 6  \\
    10 & = 1 \cdot 6 + 4  \\
    6 & = 1 \cdot 4 + 2  \\
    4 & = 2 \cdot 2 + 0
    \label{eqn:euqulid_beweis}
  \end{split}
\end{equation}
%
Der grösste Gemeinsame Teiler von 302 und 146 ist somit 2. \\[2ex]
Um dieses Verfahren zu beweisen gehen wir davon aus das q, p, r und a alles natürliche Zahlen sind. Da wir im zweiten Schritt den grössten gemeinsamen Teiler von q und r errechnet haben, kann man folgende Gleichung aufstellen:
%
\begin{equation}
  ggT(q,p) = ggT(q,r)
\end{equation}
%
Wenn also eine Zahl q und p teilt, muss diese Zahl auch r teilen. Denn sonst würde die Gleichung \ref{eqn:euqulid_beweis} nicht aufgehen. Umgekehrt muss eine Zahl die q und r teilt auch p teilen. Gehen wir zurück zu unserem Beispiel. Da diese Gleichung stimmt, stimmt auch unseres Verfahren.
%
\begin{equation*}
 ggT(5984,302) = ggT(302,146) = ggT(146,10) = ggT(10,6) = ggT(6,4) = 2 
\end{equation*}
%
%******************************************************************************
% Erweiterter Euklidischer Algorithmus
%******************************************************************************
\subsubsection{Erweiterter Euklidischer Algorithmus}
Der erweiterte Euklidischer Algorithmus oder auch die Vielfachsummendarstellung besagt folgende Gleichung.
%
\begin{equation}
  d = x \cdot q + y \cdot p 
  \label{eqn:erw_euklid_algo}
\end{equation}
%
Das wollen wir am Besten an einem Beispiel zeigen. Wir nehem dazu die gleichen Zahlen wie vorhin beim Euklidischer Algorithmus \ref{eqn:euqulid_beweis}.
\\
Um x und y heraus zu finden, schreibt man sich am Besten eine neue Gleichung bei der der Rest alleine auf der rechten Seite steht.
%
\begin{flalign*}
  2 & = 6 - 1 \cdot 4  \\
  4 & = 10 - 1 \cdot 6  \\
  6 & = 146 - 14 \cdot 10 \\
  10 & = 302 - 2 \cdot 146
  \label{eqn:erw_eukl_rest}
\end{flalign*}
%
% TODO: AUSSCHREIBEN
Dann startet man vom kleinsten Rest aus. gleichung $2 = 6 - 1 \cdot 4$ 
Wir nehmen die Gleichung  $2 = 6 - 1 \cdot 4$  und ersetzen 4 mit der Formel  $4 = 10 - 1 \cdot 6$. So erhalten wir:
%
\begin{flalign*}
  2  &= 6 - 1 \cdot 4  =  6 - 1 \cdot (10 - 1 \cdot 6) =  6 - 1 \cdot 10 + 1 \cdot 6 = 2 \cdot 6 - 1 \cdot 10 \\
  &= 2 \cdot 6 - 1 \cdot 10  = 2 \cdot (146 - 14 \cdot 10) - 1 \cdot 10 = 2 \cdot 146 - 2 \cdot 14 \cdot 10 - 1 \cdot 10 = 2 \cdot 146 - 29 \cdot 10   \\
  &= 2 \cdot 146 - 29 \cdot 10 = 2 \cdot 146 - 29 (302 - 2 \cdot 146) = 2 \cdot 146 - 29 \cdot 302 + 58 \cdot 146 = 60 \cdot 146 - 29 \cdot 302
\end{flalign*}
%
Jetzt haben wir unsere Formel
%
\begin{equation}
 2 = 60 \cdot 146 + (-29) \cdot 302 = (-29) \cdot 302 + 60 \cdot 146  
\end{equation}
%
Nun haben 
\subsubsection{Modularen Inverse}
Haben wir nun die Vielfachsumme gebildet, bilden wir aus diesen Zahlen die modulare Inverse. Die modulare Inverse drückt folgende Gleichung aus.
%
\begin{equation}
 a \cdot x \bmod n = 1
\end{equation}
%
Das geht nur wen der ggT von a n 1 ist also diese Zahlen teilfremd sind. a ist also modulo n inventierbar.
%
\begin{equation}
  1 = x \cdot a + y \cdot b
\end{equation}
%
Weil $n \cdot y$ teilbar ist durch n ergibt $ (a \cdot x) / N $ einen Rest von 1. Deshalb ist
%
\begin{equation}
  a \cdot x \bmod n = 1
\end{equation}
%
So hat man mit x die modulare Inverse von $ a \bmod n $
