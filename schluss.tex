\part{Schluss}
\section{Schluss}
\subsection{Leitfrage beantworten}
Durch unsere Leitfrage konnten wir die Arbeit eingrenzen. Trotzdem mussten wir viele verschiedene Themen behandeln oder anschneiden um diese Frage nun beantworten zu können. Unsere Leitfrage war:
\begin{center}\textit{Wieso ist RSA heute noch sicher?}\end{center}
Im Kapitel Sicherheit sind wir unter anderem auf die Schlüssellängen eingegangen. Durch die schnelleren Computer musste die Schlüssellänge erhöht werden, damit die Sicherheit gewährleistet ist. Somit können wir die Frage wie folgt beantworten: Der RSA-Algorithmus ist heute noch sicher weil die Schlüssellänge erhöht wurde.\\
Das Faktorisierungsproblem von grossen Primzahlen, auf welches der RSA-Algorithmus aufbaut, konnte bisher noch nicht effizient gelöst werden. Daher kann die Frage zusätzlich wie folgt beantwortet werden: Der RSA-Algorithmus ist heute noch sicher, weil das Faktorisierungsproblem noch nicht effektiv gelöst werden konnte oder es keine effektive Lösung gibt.
%
\subsection{Ergebnisse}
Unsere Arbeit erfasst das Thema der RSA-Verschlüsselung auf einer zweckmässigen Ebene. Die Werke, welche uns bei der Recherche begegnet sind, waren im Gegensatz sehr kompliziert und detailliert auf gewisse Themen zugeschnitten. Diese einfachere, übersichtliche Auffassung des Themas haben wir vergleichsweise nicht gefunden und ist daher neu.\\
Das Ziel dieser IDPA war den RSA-Algorithmus grundlegend zu verstehen und herauszufinden, was ihn sicher macht.\\
Dieses Ziel haben wir für uns persönlich erreicht und konnten das erarbeitete Wissen in unsere Arbeit einfliessen lassen. Durch die Anwendung und die Hintergründe des RSA-Algorithmus lernten wir für uns neue mathematische Verfahren kennen. Dieses Wissen können wir in unserem zukünftigen Beruf für die Evaluierung von Sicherheitskonzepten oder die Implementierung des Algorithmus nutzen. Im Allgemeinen Teil sind wir auch auf die gesellschaftliche Bedeutung der Verschlüsselung eingegangen. Die sichere Kommunikation zwischen Botschaften oder Behörden ist für die Allgemeinheit von grosser Bedeutung. Die Verschlüsselung kann jeweils für das Gute, wie die sichere Kommunikation zwischen Botschaften, oder das Schlechte, wie der verschlüsselte Austausch von illegalen Dateien, verwendet werden. Trotz der möglichen verwerflichen Einsatzgebiete muss die Verschlüsselung sicher sein und legal, da ansonsten für die Allgemeinheit grosse Einbussen in der Sicherheit bestehen. \\
%
\subsection{Rückblick und Stellungnahme IDPA}
% Kritische Würdigung der gewählten Vorgehensweise und der Ergebnisse
Unsere IDPA sehen wir als Erfolg an. Unser gesteckten Ziele konnten wir erreichen. Wir sind mit unserer Arbeit zufrieden, da sie sehr professionell formatiert und auch ein sehr komplexes Thema behandelt. Wir versuchten jeweils das Wichtigste aus der Literatur zu recherchieren und schrieben diese neuen Erkenntnisse in unserer Arbeit nieder. \\
Folgende Entscheidungen wirkten sich auf unsere Arbeit aus:
\paragraph*{Latex}
Die Verwendung von Latex war ein Erfolg. Die Einarbeitung brauchte einiges an Zeit, jedoch holten wir diese mit den umfangreichen Formeln und Formatierungen wieder auf. Wir sind mit der Formatierung und der klaren Struktur zufrieden.
\paragraph*{Versionierung mit Subversion}
Die Versionierung mit Subversion\footnote{Subversion ist eine Software, welche die Versionierung handhabt und eventuelle Konflikte anzeigt} hatte mehrere Vorteile. Durch die Datensicherung online konnten wir gewährleisten, das bei einem Datenverlust trotzdem die letzte Version wieder zur Verfügung steht. Durch den Onlineabgleich konnten wir an der aktuellen Version arbeiten und stellten fest, falls ein Teammitglied die gleiche Zeile bearbeitet hat. 
\paragraph*{Zusammenarbeit}
Durch die Aufteilung von verschiedenen Teilen im Team, konnten wir schnell verschiedene Gebiete erarbeiten. Durch Zeitaufwändige Besprechungen haben wir uns jeweils auf den aktuellsten Stand gebracht. Ohne die Aufteilung hätten wir jedoch nicht diesen Umfang erreicht.
%
\subsection{Ausblick in die Zukunft}
Der weitere Einsatz des RSA-Algorithmus ist schwierig vorherzusehen. Wir vermuten das die Schlüssellänge ab 2015 auf 4096 Bit erhöht wird, da die Vergangenheit gezeigt hat, das alle 10 Jahre die empfohlene Schlüssellänge verdoppelt wird. Daher vermuten wir das die Sicherheit auch noch in 5 Jahren gegeben ist.\\
Die Entwicklung der Rechenleistung ist exponentiell, daher wird das knacken durch mögliches ausprobieren einfacher. Durch das Netzwerk von vielen PCs und Mobiltelefonen können Angreifern eine enorme Rechenkapazitäten zur Verfügung stehen. Diese Problematik muss weiter im Auge behalten werden.\\
Nicht absehbar ist die Entwicklung im Bereich von Quantencomputer. Falls dadurch neue und effektivere Methoden zur Verfügung stehen, könnte diese die Verschlüsselung in Bedrängnis bringen. \\
Wir vermuten das Faktorisierungsproblem wird in den nächsten Jahren nicht gelöst. Daher bleibt die Grundlage des Algorithmus weiter sicher.
%
%Ob der RSA-Algorithmus auch noch in 20 Jahren sicher ist, kann man nicht vorhersagen. Es ist aber davon auszugehen, dass das Faktorisierungs-Problem nicht gelöst werden kann und somit der RSA-Algorithmus sicher bleibt.\\
%Eine Gefahr geht von den immer schneller werdenden Computern aus, die in einer kürzeren Zeit mehr Berechnungen lösen können. Dieses Problem lässt sich lösen, indem man längere Schlüsselpaare verwendet.\\
%Durch die Vernetzung von verschiedenen Computern steht Angreifern eine noch grössere Rechenkapazität zur Verfügung. Durch das kann die Rechenzeit für einen Angriff massiv verkürzt werden. Daher sollte man nebst dem langen Schlüssel das Schlüsselpaar regelmässig erneuern.
%
%Die Implementierung der RSA-Verschlüsselung wird auch immer weiter entwickelt und verbessert, sodass Angriffe abgewehrt werden können.
%Somit kann man davon ausgehen, dass der RSA-Algorithmus auch in Zukunft sicher sein wird, wenn man die Schlüssellänge stetig vergrössert.
%
%\\[2ex]
\subsection{Dank}
An dieser Stelle möchten wir unseren Betreuern Herrn Stefan Kuster und Garry Colin herzlich für Ihre Unterstützung danken.\\
Auch möchten wir Herrn Heer, Frau Wernli und Herrn Mauchle für ihre Hilfe und ihre Zeit, die sie uns zur Verfügung gestellt haben, danken.
\newpage
\section{Abstract}
We wrote this paper for our professional baccalaureate between August and December 2011. The main subject is the RSA-algorithm, especially the mathematical proof of this asymmetric encryption method. We decided to choose this subject because we are apprentices in application engineering and have a particular interest in technical matters. We opted for the combination of English and Maths because the literature dealing with the RSA-algorithm is mostly in English, additionally it is not possible to work without a fundamental understanding of maths. The reasons for the use of latex are firstly the fact that it is a mathematical work and secondly, the extensive character of the document. %Furthermore, the advantages are the correct typeset of the formulas. 
%For this paper we worked with subversion (a version controlled environment) to guarantee the backup and work with the latest version anytime. \\
%We kept our working journal up to date by means of the google project document. \\
The main goal of this paper was to understand the functionality of the RSA-algorithm as well as a new region of maths. We divided our work in three parts to achieve the above mentioned aims. The first part consist of an overview of cryptography linked with information about the historical background. Already Julius Caesar had thought about encryption and used some primitive algorithm, today known as the Caesar encryption. If the allied forces had not broken the Enigma, the Second World War would have been totally different. %Anders verlaufen
In the second chapter, we show the acquired background knowledge and explain the functionality of the RSA-algorithm. 
On the one hand, the use of encryption and decryption is rather reasonable. On the other hand we were facing the two following difficulties. The one of creating the correct private and public key, which is more complicated due to the use of many different formulas. Furthermore, we wanted to demonstrate why the algorithm works. For this we had deal with the usage of mathematical knowledge and work out a combination to proof the algorithm's correctness. 
The last part is about the subject of security. In this context we answer the following research question: Is the RSA-algorithm secure nowadays? We did not find a final answer for this question, but with the made assumptions, %Gemachten Annahmen
we can say that it is secure in so far as the conditions of a 2048 bit key and the correct implementation to find the randomize primes are full filled. \\
When we started our research we had to acquire a lot of new knowledge in maths. Although this gained knowledge is interesting, it took some time to fully understand the meaning due to its complexity. After this face of collecting information we finally succeeded in delving into the RSA-algorithm. %Sich in ein Thema vertiefen
From day to day we came closer to understanding the principles of the algorithm and the encryption in general. The learning process was intensive and exciting. This new accomplished knowledge can be used in our future job where different encryption algorithms need to be compared or even implemented. We are now able to understand the usage of an asymmetric encryption and even found some practical examples in our daily life which need those encryptions.  
We found a lot of information about the algorithm on the internet. Although this information was properly researched we got most of our processed knowledge out of academic books. We compared these various sources and merged them together in this paper.\\
This paper was successful because we reached the main aim of the subject and the professional baccalaureate. On the one hand we increased our knowledge about primes and some specific algorithms, on the other hand we had to eliminate some derivations  %Herrleitung
because it goes beyond the scope of this paper. The challenges were to concentrate on the fundamental parts as well as to understand the difficult formulas. We are pleased that we solved these problems with a lot of effort and look forward to future developments. 
\newpage
\section{Erklärung}
Hiermit erklären wir, dass wir die Interdisziplinäre Projektarbeit ohne fremde Hilfe angefertigt haben und nur die im Quellenverzeichnis angeführten Quellen und Hilfsmittel benutzt haben.\\
%Ich weiss, nit so schön aber funktioniert :)
  \begin{tabbing}
    Name \= Platz \= Platz \= Platz \=  Unterschrift \kill 
    Nicolas Mauchle:   \> \> \> \>   \rule{50mm}{1pt} \\[2ex]
    Denis Augsburger:   \> \> \> \>  \rule{50mm}{1pt} \\
  \end{tabbing}
Muttenz, 19. Dezember 2011
\newpage
\section{Anhang}
% Abbildungsverzeichnis
\listoffigures
% Tabellenverzeichnis
\listoftables
%

