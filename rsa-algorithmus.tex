\part{Der RSA-Algorithmus}
%
\section{Mathematisches Verfahren}
Dieses Kapitel wurde mit Hilfe des Buches \textit{Moderne Verfahren der Kryptographie}\cite{mod_kry}, \textit{Zahlentheorie für Einsteiger}\cite{zahlentheorie_fuer_einsteiger} und \textit{Kryptologie}\cite{kryptologie} erarbeitet.\\[2ex]
%
Der RSA-Algorithmus baut grundsätzlich auf Primzahlen, der Modulo und Eulerschen-Funktion, dem Satz von Euler und dem erweiterten euklidischen Algorithmus auf.\\
Im nächsten Abschnitt werden die einzelnen mathematischen Verfahren kurz in ihren Funktionen erläutert.\\
Wir werden für die beiden Primzahlen durchgehend die Variablen $p$ und $q$ verwenden. $N$ steht immer für das Produkt von $p \cdot q$. Die Variable $r$ wird für das Resultat der Modulo-Berechnungen verwendet. $e$ steht für den Verschlüsselungs-Exponenten und $d$ für den Entschlüsselungs-Exponenten.
%
\subsection{Modulo}
Modulo oder auch \textit{Division mit Rest} gibt den ganzzahligen Rest einer Division zweier natürlichen Zahlen an. Er wird beim RSA-Algorithmus bei der Berechnung des Entschlüsselungs-Exponenten sowie in der Verschlüsselung verwendet.\\
Modulo erklärt sich am Einfachsten an einem Beispiel:
%
\begin{equation*}
  21 \bmod(5) = 1
\end{equation*}
%
Es gilt herauszufinden, wie viel Mal 5 ganz in 21 vorhanden ist. Dafür teilen wir 21 durch 5 und schneiden den Dezimalrest ab. Um den ganzzahligen Rest zu berechnen subtrahieren wir von 21 das Produkt aus 4 mal 5.
%
\begin{flalign*}
  21 \colon 5 &= 4.2\\
  21 - 4 \cdot 5 &= 1\\
  21 &= 4 \cdot 5 + 1
\end{flalign*}
%Oder für den RSA Algorithmus veralgemeinert:
%\begin{equation}
%  \varphi(N) = q \cdot e + r
%  \label{eqn:mod_rsa}
%\end{equation}
%
\paragraph{Kongruent Modulo}
Kongruent Modulo bedeutet nichts anders, als dass zwei verschiedene Zahlen Modulo einer gleichen Zahl den selben Rest haben. Man sagt auch, dass sie in der gleichen Restklasse sind.\\
Als Beispiel nehmen wir die Zahlen 9 und 7 die Modulo 2 gerechnet werden.
%
\begin{flalign*}
  9 \bmod(2) &= 1 \\
  7 \bmod(2) &= 1  \\
  9 & \equiv 7 \bmod(2)
\end{flalign*}
%
%******************************************************************************
% Eulersche Funktion
%******************************************************************************
%Verbessern die Zahlen sind kleiner als die Zahl
\subsection{Eulersche Funktion}

Die Eulersche Funktion zählt die natürlichen, teilerfremden\footnote{Zwei Zahlen sind teilerfremd zueinander, wenn ihr grösster gemeinsamer Teiler 1 ist.}  Zahlen von n, die kleiner als n sind.\\
Für die Eulersche Funktion stellen die Primzahlen einen Spezialfall dar.
Primzahlen sind nur durch 1 und sich selbst ohne Rest teilbar. Somit ist das Resultat der die Eulerschen Funktion einer Primzahl p immer $p - 1$.\\
Um den privaten Schlüssel zu errechnen, muss man $\varphi(N)$ beziehungsweise $\varphi(pq)$ berechnen. Aus diesem Grund erläutern wir die Eulersche Funktion:
%
\begin{equation*}
  \varphi(p) = p - 1
\end{equation*}
%
Der grösste gemeinsame Teiler zweier Primzahlen ist immer 1.
Für die Eulersche Funktion zweier Primzahlen gilt:
\begin{equation}
  \varphi(pq) = \varphi(p) \cdot \varphi(q) = (p - 1) \cdot (q - 1)
  \label{eqn:eulersche_func}
\end{equation}
Der Beweis dafür liefert folgende Überlegung. Es gibt insgesamt $p \cdot q -1$ ganze Zahlen die kleiner sind als $p \cdot q$.\\
Es ist bei Primzahlen einfacher, die \textbf{nicht teilerfremden} Zahlen zu zählen und diese dann von allen möglichen Zahlen abzuziehen, als alle teilerfremden Zahlen zu zählen. Nicht teilerfremd zu p sind $(q - 1) \cdot p$ und für q gilt $ (p - 1) \cdot q$. \cite{kryptologie}\\
Als Formel geschrieben:
%
\begin{equation*}
  \begin{split}
    \varphi(pq) & = p \cdot q -1 - (p - 1) - (q - 1)  \\
     & = p \cdot q -1 - p + 1 - q + 1  \\
     & = p \cdot q -q - p + 1  \\
     & = (p -1) \cdot (q - 1)
    \label{eqn:herleitung_eulersche_func}
  \end{split}
\end{equation*}
%
% Satz von Euler
%=====================
%Verweis auf Buch Beutelspacher
\subsection{Satz von Euler}
Der Satz von Euler gewährleistet die korrekte Ver- und Entschlüsselung.\cite{zahlentheorie_fuer_einsteiger}. Wir gehen hier nicht weiter auf den Satz von Euler ein und werden später beim Beweis nochmals auf ihn zurückkommen.\\
Der Satz von Euler sagt über zwei natürliche, teilerfremde Zahlen, hier $n$ und $m$, folgendes aus: \cite{kryptologie}
%
\begin{equation}
  m^{\varphi(n)} \bmod(n) = 1
  \label{eqn:satz_von_euler}
\end{equation}
%
Nehmen wir für $n = 15$ und $m = 4$ und setzen Formel [\ref{eqn:satz_von_euler}] ein.
%
% Zeigen das varphi 15 = 8 ist
\begin{flalign*}
  4^{\varphi(15)} \bmod(15) = 1  \\
  4^8 \bmod(15) = 1
\end{flalign*}
%
%
% Kleiner Satz von Verma
%
%Wenn wir jetzt eine natürliche Zahl k nehmen, gilt für diese folgende Gleichung (Erleuterung siehe Buch \textit{Kryptologie}):
%
%\begin{equation}
%  \begin{split}
%    m^{1 + k \cdot \varphi(N)} \bmod(N)  &= m \cdot 1 \\
%    m \cdot m^{k \cdot \varphi(N)} \bmod(N) &= m \cdot 1
%    \label{eqn:satz_von_euler_erweitert}
%  \end{split}
%\end{equation}
%
%
%Hat man zwei verschiedene Primzahlen p und q und eine natürliche Zahl m, die kleiner ist als $p \cdot q$, dann gilt für jede natürliche Zahl k:
%
%\begin{equation}
%  m^{k \cdot (p - 1) \cdot (q - 1) +1} \bmod(p \cdot q) = m
%  \label{eqn:kleiner_satz_fermat}
%\end{equation}
%
%Dieser Satz wird in der Mathematik auch \textit{kleiner Satz von Fermat} genannt.
% Buch S 109 / 110
%
%******************************************************************************
% Euklidischer Algorithmus
%******************************************************************************
\subsection{Euklidischer Algorithmus}\label{euklidischer_Algorithmus}
Mit den Euklidischen Algorithmen lässt sich der grösste gemeinsame Teiler (ggT) zweier natürlichen Zahlen berechnen. Es existiert das etwas ältere Subtraktionsverfahren und das moderne Modulo-Verfahren. Da das Subtraktions-Verfahren nicht die gleiche Effizienz hat wie das Subtraktionsverfahren, konzentrieren wir uns auf das Modulo-Verfahren. \cite{zahlentheorie_fuer_einsteiger}\\
Für den RSA-Algorithmus ist es wichtig, dass wir zwei teilerfremde Zahlen $\varphi(N)$ und $e$ haben.\\
%
Die Formel für das Modulo-Verfahren lautet:
%
\begin{equation}
  \label{eqn:euklidischer_algo}
  a = b \cdot q + r 
\end{equation}
%
wobei
\begin{equation*}
  0 \leq r \leq e - 1
\end{equation*}
gelten.\\
%
Auf den RSA-Algorithmus bezogen sieht die Formel so aus:
\begin{equation}
  \label{eqn:euklidischer_algo_RSA}
  \varphi(N) = q \cdot e + r 
\end{equation}
%
Dabei sind $\varphi(N)$ und e die beiden Zahlen von denen wir den grössten gemeinsamen Teiler ausrechnen.
Wir gehen davon aus das $\varphi(N)$ grösser ist als e. Wäre das nicht der Fall, müsste e auf die linke Seite des Gleichheitszeichen.
Zur Veranschaulichung nehmen wir zwei Zahlen $\varphi(N) = 839$ und $e = 199$ und setzen diese in unsere Gleichung ein.
\begin{enumerate}
  \item Schritt: Zahlen in Gleichung einsetzen, q und r ausrechnen.\\
    \begin{equation*}
      839 = 4 \cdot 199 + 43
    \end{equation*}
  \item Schritt: 199 als neues $\varphi(N)$ und 43 als neues e einsetzten und wieder q und r ausrechnen.\\
    \begin{equation*}
      199 = 4 \cdot 43 + 27
    \end{equation*}
\end{enumerate}
Den zweiten Schritt solange wiederholen bis man auf Rest 1 oder 0 kommt. Falls die Zahl 0 herauskommt, nehmen wir den Rest der vorangegangenen Gleichung als grössten gemeinsamen Teiler.\\
%
Ganzer Ablauf:
\begin{equation*}
  \begin{split}
    839 &= 4 \cdot 199 + 43\\
    199 &= 4 \cdot 43 + 27\\
    43 &= 1 \cdot 27 + 16\\
    27 &= 1 \cdot 16 + 11\\
    16 &= 1 \cdot 11 + 5\\
    11 &= 2 \cdot 5 + 1\\
    \label{eqn:euqulid_beweis}
  \end{split}
\end{equation*}
%
Der grösste gemeinsame Teiler von 839 und 199 ist 1. Sie sind daher teilerfremd. \\[2ex]
Um dieses Verfahren zu beweisen, gehen wir davon aus, dass $\varphi(N)$, e, r und q alles natürliche nicht negative Zahlen sind. Da wir im zweiten Schritt den grössten gemeinsamen Teiler von $e$ und r errechnet haben, kann man basierend auf Formel \ref{eqn:euklidischer_algo_RSA} folgende Behauptung aufstellen:
%
\begin{equation}
  ggT(\varphi(N),e) = ggT(e,r)
  \label{eqn:ggT}
\end{equation}
%
Nur der grösste gemeinsame Teiler von $\varphi(N)$ und $e$ teilt sowohl $\varphi(N)$ als auch e und r ohne Rest. Auf unser Beispiel angewendet ergibt das $ggT(839,199) \leq ggT(199,43)$.\\
Und es gilt: Nur der grösste gemeinsame Teiler von e und r teilt auch $\varphi(N)$ ohne Rest.\\
Das heisst: $ggT(199,43) \leq ggT(839,199)$.
Somit stimmen die beiden grössten gemeinsamen Teiler überein\cite{zahlentheorie_fuer_einsteiger}.\\[2ex]
Gehen wir zurück zu unserem Beispiel:
%
\begin{equation*}
 ggT(839,199) = ggT(199,43) = ggT(43,27) = ggT(27,16) = ggT(16,11) = ggT(11,5) = 1
\end{equation*}
%
%******************************************************************************
% Erweiterter Euklidischer Algorithmus
%******************************************************************************
\subsubsection{Erweiterter Euklidischer Algorithmus}
Der erweiterte Euklidische Algorithmus berechnet neben dem grössten gemeinsamen Teiler $r$ zweier natürlichen Zahlen auch noch die dazugehörigen natürlichen Zahlen d und k, dass folgende Gleichung stimmt:
%
\begin{equation}
  ggT(a,b) = s \cdot a + t \cdot b
  \label{eqn:erw_euklid_algo}
\end{equation}
%
Auf den RSA-Algorithmus angewendet sieht die Formel so aus:
%
\begin{equation}
  r = d \cdot e + k \cdot \varphi(N) 
  \label{eqn:erw_euklid_algo_RSA}
\end{equation}
%
Mit seiner Hilfe lässt sich der Entschlüsselungs-Exponent d berechnen, welchen wir für die Berechnung des privaten Schlüssels brauchen.\\
Dieser Algorithmus wird auch \textit{Vielfachsummendarstellung} genannt, denn um auf die Zahlen d und k zu kommen, muss man die Vielfachsummendarstellung anwenden.\\
Es ist eine erweiterte Form des Euklidischen Algorithmus.\\
Wir schreiben jetzt die Rechnung \ref{eqn:euqulid_beweis} so um, dass wir den Rest auf einer Seite isolieren.\\ 
\begin{equation}
  r = \varphi(N) - a \cdot e
  \label{eqn:form_erw_euklid}
\end{equation}
%
\begin{flalign*}
  43 &= 839 - 4 \cdot 199\\
  27 &= 199 - 4 \cdot 43\\
  16 &= 43 - 1 \cdot 27\\
  11 &= 27 - 1 \cdot 16\\
  5 &= 16 - 1 \cdot 11\\
  1 &= 11 - 2 \cdot 5
  \label{eqn:erw_euklid_10}
\end{flalign*}
%
% TODO: AUSSCHREIBEN
Man startet bei der untersten Zeile und ersetzt die Zahl, die den nächsten oberen Rest beschreibt, durch die Gleichung. Das heisst man ersetzt in der ersten Gleichung die Zahl $5$ mit dem Term $16 - 1 \cdot 11$. Das wiederholt man so lange, bis man die letzte Gleichung eingesetzt hat. 
Es ist zu beachten, dass auf der rechten Seite eine Summe aus zwei Produkten stehen bleibt.
%
\begin{flalign*}
  1  &= 11 - 2 \cdot 5 = 11 - 2 \cdot 16 + 2 \cdot 11 = 3 \cdot 11 - 2 \cdot 16 \\
  &= 3 \cdot 11 - 2 \cdot 16 = 3 \cdot 27 - 3 \cdot 1 \cdot 16 - 2 \cdot 16 = 3 \cdot 27 - 5 \cdot 16\\
  &= 3 \cdot 27 - 5 \cdot 16 = 3 \cdot 27 - 5 \cdot 43 + 5 \cdot 1 \cdot 27 = 8 \cdot 27 - 5 \cdot 43\\
  &= 8 \cdot 27 - 5 \cdot 43 = 8 \cdot 199 - 8 \cdot 4 \cdot 43 - 5 \cdot 43 = 8 \cdot 199 - 37 \cdot 43\\
  &= 8 \cdot 199 - 37 \cdot 43 = 8 \cdot 199 - 37 \cdot 839 + 37 \cdot 4 \cdot 199 = 156 \cdot 199 - 37 \cdot 839
\end{flalign*}
%
Am Schluss bekommen wir die Formel
%
\begin{equation*}
 1 = 156 \cdot 199 + (-37) \cdot 839
 \label{eqn:erw_euklid_end}
\end{equation*}
%
Damit wir jetzt den Entschlüsselungs-Exponenten d bekommen, müssen wir die Gleichung in die Modulare Inverse umschreiben. Wir berechnen somit die Modulare Inverse von $\varphi(N)$ und e.\\
Die Formel der modularen Inverse auf den RSA-Algorithmus angewendet lautet:
\begin{equation}
  e \cdot d \bmod(\varphi(N)) \equiv 1
\end{equation}
%
Wir gehen von Formel \ref{eqn:erw_euklid_end} aus und setzen den Term $(-37) \cdot 839$ auf die linke Seite.
%
\begin{equation*}
 1 - 37 \cdot 839 = 156 \cdot 199
\end{equation*}
Damit man den nächsten Schritt versteht, muss man wissen, dass man eine Gleichung
\begin{equation*}
  1 + k \cdot \varphi(N) = d \cdot e
\end{equation*}
mit Modulo ausdrücken kann.
\begin{equation*}
  1 \equiv d \cdot e \bmod(\varphi(N))
  \label{eqn:erw_eukl_fertig}
\end{equation*}
%
Aufgrund dessen können wir die Modulare Inverse von $\varphi(N)$ und e so ausdrücken:
%
\begin{equation*}
 1 \equiv 156 \cdot 199 \bmod(839)
\end{equation*}
Wir erhalten den Entschlüsselungs-Exponent $d = 156$.
%
%Um das zu beweisen schauen wir uns nochmal die Formel \ref{eqn:euklidischer_algo} an. Aus einer Vielfachsummendarstellung von $q$ und $r$, lässt sich dei Vielfachsummendarstellung von $p$ und $q$ ableiten. Man kann die Formel
%{eqn:ggT}
%\begin{equation*}
%  2 = x \cdot p + y \cdot q
%  \label{eqn:erw_eukl_fertig}
%\end{equation*}
%
%auch mit r und a darstellen.
%
%\begin{equation*}
%  2 = x \cdot r + y \cdot a
%\end{equation*}
%
%Jetzt setzen wir die Formel \ref{eqn:form_erw_euklid} anstatt r ein und erhalten diese Gleichung
%\begin{equation*}
%  2 = x \cdot (p - a \cdot q)  + y \cdot a = x \cdot p + (y - xa) \cdot q
%\end{equation*}
%
%Somit haben wie die Vielfachsummendarstellung für $p$ und $q$.
%
%\subsubsection{Modulare Inverse}
%Haben wir zwei natürliche, teilerfremde Zahlen, gibt es eine Zahl x, sodass die Formel
%
%\begin{equation*}
%  p \cdot x \bmod(q) \equiv 1
%\end{equation*}
%
%Es gibt zwei ganze Zahlen $x$ und $y$ sodass die Gleichung \ref{eqn:erw_eukl_fertig} entsteht.
%Weil $y cdot q$ durch $q$ teilbar ist, gibt es bei der Divison von $x \cdot p$ durch $q$ immer 1. Somit kann man sagen
%
%\begin{equation}
%  p \cdot x \bmod q = 1
%  \label{eqn:mod_inverse}
%\end{equation}
%
%Die Modulareinverse von zwei teilerfremden natürlichen Zahlen $p$ und $q$, errechnet man mit dem erweiterten Euklidischen Algorithmus. Dann ist $x$ die modulare Inverse von $p$ und $q$.

%
\section{Der Schlüssel}
Für eine sichere Verschlüsslung kommt es auf den Schlüssel an. Denn nur mit diesem Schlüssel, kann man die verschlüsselte Nachricht enschlüsseln. Der Schlüssel spielt also eine zentrale Rolle und muss sicher sein.

\subsection{Sicherer Schlüssel generieren}
Für einen sicheren Schlüssel sollte man immer Zufallszahlen verwenden. Zufallszahlen sind Zahlenfolgen bei denen man die nächste Zahl nicht durch mathematische Berechunungen vorhersagen kann - also Zufällige Zahlen. 
\subsubsection{Zufallszahlen generieren}
Um solche Zufallszahlen zu generieren hat man Zufallszahlengeneratoren (engl random number generator = RNG) gebaut. Diese messten den Radioaktiven Zerfall oder beobachteten die atmosphärischen Bedingungen in der Umgebung. Die Zahlen die der RNG zurückgibt, kann man dann Verwenden um eine Zufallszahl zu generieren, da der Output auf einem Input basiert, der sich permanent ändert und nicht wiederholbar ist.

Was, wenn man keine solchen Messgeräte hat, die solche Zahlen generieren können. 
Für das gibt es Pseudozufallszahlengeneratoren (engl pseudo random number generator = PRNG)

\paragraph{Entropie}
\paragraph{Zahlentheoretische Funktionen}
\subsection{Schlüssel austauschen}

%
\subsection{Formel Verschlüsselung}
Sind die Schlüssel erstellt, kann ein Text m einfach verschlüsselt werden.\\
Dazu dient folgende Formel:
%
\begin{equation}
  c \equiv m^e  \bmod(N)
  \label{eqn:rsa_encription}
\end{equation}
%
%\subsection{RSA Entschlüsselung}
%Die RSA Entschlüsselung ist ohne den Geheimschlüssel zu wissen nicht möglich. Da die Sicherheit von diesem Geheimschlüssel abhängt, gilt es diesen möglichst lang zu wählen und geheim zu halten. Aktuell gilt eine Schlüssellänge von 2048 Bit als sicher. Als Dezimalzahl ausgedrückt, ist dies eine Zahl von $ 3,2 \cdot 10^{616} $
%Die Angriffe auf die RSA Verschlüsselung zielen darauf ab, den Geheimschlüssel zu ermitteln. Das Problem dabei ist die Primfaktoren Zerlegung von grossen Zahlen. Mehr dazu im Kapitel Angriffe %Verweis%

\subsection{Formel Entschlüsselung}
Die Entschlüsselung der Nachricht wird mit folgender Formel realisiert:
%
\begin{equation}
  m \equiv c^d \bmod(N)
  \label{eqn:rsa_decription}
\end{equation}
%
Der Klartext m hängt in diesem Fall vom verschlüsselten Text c, dem privaten Schlüssel d und dem RSA-Modul N ab. 
%
%
\newpage
%******************************************************************************
% Mathematischer Beweis
%******************************************************************************
\section{Mathematischer Beweis der Funktionsweise}
Bisher nahmen wir an, dass der RSA-Algorithmus korrekt arbeitet. Diese Annahme möchten wir nun beweisen. Der Algorithmus arbeitet korrekt, wenn die Verschlüsselung und nachherige Entschlüsselung wieder das gleiche ergibt. Dies darf nur mit dem zugehörigen privaten und öffentlichen Schlüssel möglich sein.
%
\subsection{Verschlüsselung und Entschlüsselung gleichsetzen}
Für den Beweis benötigen wir die ursprünglichen Formeln zur Verschlüsselung und Entschlüsselung. Diese lauten:
\begin{flalign*}
  c & = m^e mod N \\
  m & = c^d mod N
\end{flalign*}
Da in beiden Formeln die Ursprungsnachricht m und die verschlüsselte Nachricht c vorkommen, können wir diese Formeln gleichsetzen. Dazu lösen wir die Entschlüsselungsformel nach c auf:
\begin{flalign*}
  m &= c^d mod N \\
  m &= c^d - k * N \\
  m + k \cdot N & = c^d \\
  \sqrt[d]{m + k \cdot N} & = c
\end{flalign*}
Modulo ergibt jeweils den Rest einer ganzzahligen Division. Durch Umformung kann diese jeweils als $ k \cdot N $ dargestellt werden, wobei k eine ganze Zahl sein muss.\\
Nun können wir c anders ausdrücken und in der Verschlüsselungsformel einsetzen.
%Wir setzen nun die beiden Formeln gleich, so das nur noch die Ursprungsnachricht m in der Gleichung vorhanden ist:
\begin{flalign*}
  m^e mod N & = \sqrt[d]{m + k \cdot N}\\
  m^{e \cdot d} mod N & = m + k \cdot N\\
  m^{e \cdot d} mod N & = m 
  %{m^e}^d mod N & = m \\
  %{m^d}^e mod N & = m
\end{flalign*}
Die Formel $ m^{e \cdot d} mod N = m + k \cdot N $ kann gekürzt werden, da k in jedem Fall 0 sein muss. Dies liegt daran, dass wir auf der rechten Seite mit Modulo N den Rest ausgeben. Der Rest kann 0 bis N-1 gross sein. Da m einen Wert hat, muss k in diesem Fall 0 sein. \\
Wir möchten beweisen, dass die Verschlüsselung (hoch e) und nachherige Entschlüsselung (hoch d) wieder die ursprüngliche Nachricht ergibt:
\begin{equation*}
 m^{e \cdot d} mod N = m 
\end{equation*}
%
%\subsection{Grundlagen zur Erklärung}
%Für den Beweis benötigen wir vorherige Kenntnisse und bestimmte Sätze. Diese möchten wir hier nochmals kurz in Erinnerung rufen.\\ 
%Das RSA-Modul N wird aus den ausgewählten Primzahlen p und q erstellt.
%\begin{equation*}
%  N = p \cdot q
%\end{equation*}
%
%e wurde teilerfremd zu $ \varphi(n) $ gewählt. 
%$ \varphi(n) = (p-1) \cdot (q-1) $
%
%Zusätzlich wurde d so gewählt, dass folgendes zählt:
%\begin{equation*}
% e \cdot d + k \cdot \varphi(N) = 1 = ggT(e,\varphi(N))
%\end{equation*}
%
%Für den Beweis müssen wir ausschweifen auf den Satz von Euler Fermat. Dieser bildet die Grundlage zur RSA-Verschlüsselung. Er lautet wie folgt:
%\begin{equation*}
%	a^{\varphi(n)} \equiv 1\,(\mathrm{mod}\,n)
%\end{equation*}
%Da wir mit Primzahlen arbeiten, kann dieser Satz auch anders ausgedrückt werden. $ \varphi(n) $ gibt alle teilerfremden Zahlen zu n an. Da n durch zwei Primzahlen gebildet wurde, gibt es $ (p-1) \cdot (q-1) $ teilerfremde Zahlen. 
%
\subsection{Beweis der Funktionsweise}
Die Gleichsetzung der Entschlüsselung und Verschlüsselung dient uns als Grundlage des Beweises:
\begin{equation*}   
 m^{e \cdot d} mod N = m
\end{equation*}
%
Durch die Anwendung des erweiterten Euklidischen Algorithmus \ref{eqn:erw_eukl_fertig} bei der Erstellung des Schlüssels \ref{eqn:rsa_private_key_erstellen} wurde folgendes bestimmt:
%Wir könnnen durch die modulare Inverse $ e \cdot d $ anders ausdrücken, siehe [\ref{eqn:mod_inverse}]. Diese lösen wir nach $ e \cdot d $ auf:\\
\begin{flalign*}
 e \cdot d &= 1 \bmod{\varphi(N)}  \\
 % e \cdot d + k \cdot \varphi(N) &= 1  \\
 e \cdot d &= 1 + k \cdot \varphi(N) \\
 e \cdot d &= k \cdot \varphi(N) + 1
\end{flalign*}
Wenn die Formel Modulo beinhaltet, kann diese jeweils auch mit k Mal den Rest ausgedrückt werden. \\
%
Nun ersetzen wir in unserer Ausgangslage $ e \cdot d $ durch $ k \cdot \varphi(N)+1 $ und erhalten die Gleichung wie in Formel [\ref{eqn:kleiner_satz_fermat}].
\begin{flalign*}
 m^{e \cdot d} \bmod(N) &= m \\
 m^{ k \cdot \varphi(N) + 1} \bmod(N) &= m  \\
 m^{k \cdot \varphi(N)} \cdot m \bmod(N) &= m  \\
 { m^{ \varphi(N) }} ^k \cdot m \bmod(N) &= m \\
 { m^{ \varphi((p-1)\cdot(q-1)) }} ^k \cdot m \bmod((p-1)\cdot(q-1)) &= m
\end{flalign*}
%
%Durch den kleinen Satz von Fermat wissen wir dass $ \varphi(N) $ 1 sein muss. 
Mit dem kleinen fermatschen Satz können wir nun nach $ m^{\varphi(N)} $ auflösen.\ref{eqn:kleiner_satz_fermat}
%TODO: Kleiner Satz von Fermat erklären
\begin{flalign*}
  m^{(p-1)} &= 1 \bmod p \\
  m^{(p-1) \cdot (q-1)} &= 1 \bmod(p \cdot q) \\
  m^{\varphi(N)} &= 1 \bmod N
\end{flalign*}
1. Die Nachricht hoch die Anzahl der Teilerfremden Zahlen bei einer Primzahl ergibt 1 Modulo die Zahl \\
2. Bei der RSA-Verschlüsselung haben wir die Kombination aus zwei Primzahlen.
%nicht hoch eine Primzahl gerechnet, sondern hoch $ (p-1) \cdot (q-1) $. 
Daher können wir laut dem kleinen fermatschen Satz die Gleichung so darstellen.\\
3. $ p \cdot q $ ist das RSA-Modul N, siehe \ref{eqn:rsa_modul}. $ (p-1) \cdot (q-1) $ sind die Teilerfremden Zahlen von N, da es sich bei p und q um Primzahlen handelt, siehe Eulersche Funktion \ref{eqn:eulersche_func}. \\
%
Daraus resultiert das $ 1 \bmod N $ das gleiche ist wie $ m^\varphi(N) $ und entsprechend eingesetzt werden kann. Danach lösen wir die Formel auf. 
\begin{flalign*}
 { m^{ \varphi(N) }} ^k \cdot m \bmod(N) &= m  \\
 {1 \bmod N }^k \cdot m \bmod(N) &= m  \\
 1^k \cdot m \bmod(N) &= m \\
 1 \cdot m \bmod(N) &= m \\
 m + x \cdot N &= m \\
 m + 0 \cdot N &= m \\
 m &= m 
\end{flalign*}
$ 1 \bmod N $ muss 1 ergeben da der Rest höchstens 1 sein kann und das RSA-Modul N grösser als 1 ist. \\
$ 1^k $ ergibt immer 1, da die Basis 1 hoch eines beliebigen Exponenten 1 ergibt. \\
Wir wissen das $ x = 0 $ sein muss, da m immer kleiner als N gewählt wird. Falls m grösser gleich N ist, würde der RSA-Algorithmus nicht funktionieren. In diesem Fall wird die Nachricht aufgeteilt und später wieder zusammengesetzt.\\
Schlussendlich stellen wir fest, dass m = m ist und beweisen somit die korrekte Funktionsweise des RSA-Algorithmus. Die Verschlüsselung und darauffolgende Entschlüsselung ergibt die Ursprungsnachricht.\\
Die Entschlüsselung der Nachricht funktioniert nur mit dem zugehörigen privaten Schlüssel, da mit dem euklidischen Algorithmus, d so bestimmt wurde, das folgendes zählt:
\begin{flalign*}
 e \cdot d + k \cdot \varphi(N) &= 1	
\end{flalign*}
Mit der modularen Inverse wurde d so bestimmt, das keine negative Zahl entsteht. Aus der vorherigen Formel ist ersichtlich, dass nur ein ganz bestimmtes d zu einem e und N gehört. Jeder andere Entschlüsselungexponent würde die Formel nicht korrekt erfüllen und zu einem anderen Ergebnis führen. 
%
%
%***************************************
% BEISPIEL
%***************************************
\section{Beispiel}
Nachdem wir nun das ganze Wissen für den RSA-Algorithmus haben, ist es an der Zeit ein kleines Beispiel durchzugehen.
\subsection{Einfaches Zahlenbeispiel}
Wir starten mit einem einfachen Zahlenbeispiel. Auf die mathematischen Methoden wird nur noch verwiesen. Wir erklären sie hier nicht mehr.\\
Für dieses Beispiel nehmen wir die Zahlen
%
\begin{flalign*}
  p = 13 \\
  q = 19
\end{flalign*}
%
\paragraph{Schlüssel-Paar erstellen}
Wir berechnen nun das RSA-Modul N [siehe Formel \ref{eqn:rsa_modul}] und $\varphi(N) $ [siehe Formel \ref{eqn:eulersche_func}] aus p und q.
\begin{equation*}
  \tag{RSA-Modul}
  247 = 13 \cdot 19
\end{equation*}
%
\begin{equation*}
  \tag{$\varphi(N)$}
  216 = (13 - 1) \cdot (19 - 1)
\end{equation*}
%
Zu $ \varphi(N) $ suchen wir uns eine zweite Zahl e, die teilerfremd zu $ \varphi(N) $ ist, sprich den $ggT(\varphi(N),e) = 1$ hat. Am Besten man verwendet für e eine weitere Primzahl.
%
\begin{equation*}
    e = 23
\end{equation*}
%
Der öffentliche Schlüssel $N = 247$ und $e = 23$ ist nun berechnet [siehe Section \ref{sec:public_key}]\\
Um den Entschlüsselungs-Exponenten zu berechnen, müssen wir zuerst den ggT(216,23) [siehe Formel \ref{eqn:euklidischer_algo}] ausrechnen.
\begin{flalign*}
  216 & = 9 \cdot 23 + 9 \\
  23 & = 2 \cdot  9 + 5 \\
  9 & = 1 \cdot  5 + 4 \\
  5 & = 1 \cdot  4 + 1
\end{flalign*}
%
Jetzt weneden wir den Erweiterter Euklidischer Algorithmus [siehe Formel \ref{eqn:erw_euklid_algo}] an.
\begin{flalign*}
  1 &= 5 - 1 \cdot 4 = 5 - 1 \cdot(9 - 1 \cdot 5) = 5 - 1 \cdot 9 + 1 \cdot 5 = 2 \cdot 5 - 1 \cdot 9\\
  &= 2 \cdot 5 - 1 \cdot 9 = 2 \cdot (23 - 2 \cdot 9) - 1 \cdot 9 = 2 \cdot 23 - 4 \cdot 9 - 1 \cdot 9 = 2 \cdot 23 - 5 \cdot 9\\
  &= 2 \cdot 23 - 5 \cdot 9 = 2 \cdot 23 - 5 \cdot (216 - 9 \cdot 23) = 2 \cdot 23 - 5 \cdot 216 + 45 \cdot 23 = \textbf{47} \cdot 23 \textbf{- 5} \cdot 216
\end{flalign*}
Modulare Inverse auf letzte Gleichung anwenden und wir erhalten:
\begin{equation*}
  1 \equiv 47 \cdot 23 \bmod(216)
\end{equation*}
Und dadurch unseren privatern Schlüssel $N = 216$ und $d = 47$\\
%
Jetzt haben wir unseren privaten als auch öffentlichen Schlüssel die wir für eine Ver- Entschlüsselung benötigen.
\paragraph{Verschlüsseln}
Zur Einfachheit verschlüsseln wir die Zahl 15.\\
Verschlüsselungsfunktion anwenden:
\begin{equation*}
  15^{23} \bmod 247 = 59
\end{equation*}
Die Zahl 15 verschlüsselt mit unserem öffentlichen Schlüssel (23,247) ergibt die Zahl 59.
%
\paragraph{Entschlüsseln}
Um zu überprüfen ob unserer privater Schlüssel auch wirklich funktioniert, entschlüsseln wir die Zahl 59 mit unserem privaten Schlüssel.\\
\begin{equation*}
  59^{47} \bmod 247 = 15
\end{equation*}
Als Resultat erhalten wir 15 und können dadurch sehen, dass unser kleines Beispiel funktioniert hat.
%
% Beispiel TEXT
%
\subsection{Beispiel an einem Text}
Nun möchten wir ein Beispiel an einem Text zeigen. Da man Buchstaben nicht verschlüsseln, denn mit ihnen lässt sich nicht rechnen, müssen wir den Text in Zahlen umzuwandeln. Dafür gibt es viele verschiedene Möglichkeiten. Wir verwenden in diesem Beispiel die \textit{ASCII-Tabelle}\footnote{Die ASCII-Tabelle ordnet jedem Buchstaben eine eindeutige Zahl zu.}.\\
Weiter gilt zu beachten, dass man bei Text-Verschlüsselung grössere Schlüssel braucht. Deshalb nehmen wir für das nächste Beispiel neue Zahlen:
\begin{flalign*}
  p &= 101\\
  q &= 349\\
  N &= 35'249\\
  \varphi(N) &= 34'800\\
  e &= 509\\
  d &= 7'589
\end{flalign*}
Wir verschlüsseln das Wort \textit{GIBM MUTTENZ}. Als erstes müssen wir unseren Text in Zahlen umwandeln.\\
\textit{GIBM MUTTENZ} würde in der ASCII-Schreibweise \textit{717366773277858484697890} heissen.\\
Da $m < N$ sein muss, müssen wir unsere grosse Zahl in Blöcke aufteillen. Wir machen uns immer 4er Blöcke und verschlüsseln sie einzeln mit unserem öffentlichen Schlüssel (35'249,509).
\begin{flalign*}
  7173^{509} \bmod(35'249) &= 2330\\
  6677^{509} \bmod(35'249) &= 21891\\
  3277^{509} \bmod(35'249) &= 10955\\
  8584^{509} \bmod(35'249) &= 403\\
  8469^{509} \bmod(35'249) &= 32164\\
  7890^{509} \bmod(35'249) &= 34762
\end{flalign*}
Der verschlüsselte Text \textit{233021891109554033216434762} kann somit zum Empfänger gesendet werden.\\
Leider gibt es nicht immer gleich lange Zahlen. Das wird dann zum Problem, wenn wir den verschlüsselten Text zurück entschlüsseln. Dieses Problem wird bei einer richtigen Verschlüsselung in der Implementierung gelöst. Entweder werden die Zahlen aufgefüllt, oder man verwendet ein eindeutiges Trennzeichen.\\
Der Empfänger entschlüsselt jetzt mit seinem private Key (35'240,7589) die Nachricht. Damit er die Originalnachricht erhaltet, müssen die gleichen Blöcke genommen werden, wie bei der Verschlüsselung.
%
\begin{flalign*}
  2330^{7589} \bmod(35'249) &= 7173\\
  21891^{7589} \bmod(35'249) &= 6677\\
  10955^{7589} \bmod(35'249) &= 3277\\
  403^{7589} \bmod(35'249) &= 8584\\
  32164^{7589} \bmod(35'249) &= 8469\\
  34762^{7589} \bmod(35'249) &= 7890
\end{flalign*}
%
Der Empfänger hat seine geheime Nachricht erfolgreich entschlüsseln können. 
\section{Anwendung}
Asymmetrische Verschlüsselungen werden häufig verwendet.  Obwohl verschiedene neue asymmetrische Verschlüsselungen erfunden wurden, deckt der RSA-Algorithmus weiterhin eine grosse Menge der asymmetrischen Verschlüsselung ab.\\
Wir möchten hier einige Beispiele aufzeigen, in denen der RSA-Algorithmus zu tragen kommt. Prinzipiell kommt ein asymmetrisches Verfahren dann zum Einsatz, wenn sich zwei Parteien ohne vorherigen Kontakt eine sichere Kommunikation aufbauen möchten. Es ist naheliegend, dass im IT-Bereich, über ein Netzwerk bzw. das Internet, solche Anforderungen bestehen.

\subsection{SSH - Secure Shell}
Die secure shell dient zum Aufbau einer Verbindung auf ein Gerät. Dies sind meistens Netzwerkkomponente oder Server. Die shell ist eine Konsole mit der Befehle an das Gerät gesendet werden kann.\\
Ohne RSA-Verschlüsselung müsste die Wartung vor Ort mit einem Kabel gemacht werden oder ein gleichbleibender Schlüssel vergeben werden. Aus Sicherheits- und Zeitgründen wären beide Möglichkeiten ziemlich schlecht.

\subsection{RFID}
RFID ist eine Technologie für den Kontaktlosen Austausch von Informationen. Dies funktioniert über elektromagnetische Wellen. Der Einsatz ist sehr unterschiedlich. In der Logistik wird es gebraucht um Waren schneller zu finden und direkt mit Listen abzugleichen, ohne jedes Produkt einzeln über einen Strichcode zu scannen. Das RSA-Kryptosystem wird zum Austausch des Schlüssels bei RFID basierten Zugangssystemen verwendet. Ohne ein solches asymmetrisches Verfahren, könnte jemand die Verbindung abhören und die Karte somit kopieren bzw. den Schlüssel der Karte herausfinden.

\subsection{PGP}
PGP ist ein Programm zur Verschlüsselung von Daten mit verschiedenen Methoden. Seit der ersten Version 1991 kann RSA verwendet werden. Das Programm wurde ausserhalb der USA weiterentwickelt und liegt seit 1998 auch Quelloffen(open source) bereit. Da in den USA Exportbeschränkungen auf Kryptografische Software vorhanden sind, durfte die Software nicht in der üblichen Form vertrieben werden. Um dieses Exportverbot zu Umgehen, wurde der Quellcode der Software in Buchform vertrieben und exportiert. In Europa wurde er dann wieder abgeschrieben und kompiliert(ausführbar gemacht). 

