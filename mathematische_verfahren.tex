\section{Mathematisches Verfahren}
Dieses Kapitel wurde mit Hilfe von dem Buch \textit{Moderne Verfahren der Kryptographie} und \textit{Kryptologie} erarbeitet\\[2ex]
%
Der RSA-Algorithmus baut grundsätzlich auf Primzahlen, Modulo, Eulersche-Funktion, den Satz von Euler und den erweiterten euklidischen Algorithmus auf. Versteht man diese mathematischen Verfahren nicht, kann man ebensowenig den RSA-Algorithmus verstehen oder verwenden. \\
Im nächsten Abschnitt werden die einzelnen mathematischen Verfahren kurz in ihren Funktionen erläutert.\\
Wir werden durchgehend für die beiden Primzahlen die Variabeln $p$ und $q$ verwenden. $N$ steht immer für das Produkt von $p \cdot q$. Die Variable $r$ wird für den Rest bei Modulorechungen verwendet.
\subsection{Modulo}
Modulo oder auch \textit{Division mit Rest} gibt den Rest von einer ganzzahligen Division an. Modulo erklärt sich am einfachsten an einem Beispiel.
%
\begin{equation*}
  21 \bmod(5) = 1
\end{equation*}
%
Man sucht die nächst kleinere Zahl die durch 5 eine natürliche Zahl ergibt. Der Rest, der bis zu 21 fehlt, ist dann das Resultat.
%
\begin{equation*}
  4 * 5 = 20
\end{equation*}
%
Bis auf 21 fehlt genau 1. Somit ist 
%
\begin{equation*}
  21 \bmod(5) = 1
\end{equation*}
%
\paragraph{Kongruent Modulo}
Kongruent Modulo bedeutet nichts anders, als dass zwei verschiedene Zahlen modulo einer gleichen Zahl den selben Rest haben. Man sagt auch, dass sie in der gleichen Restklasse sind.\\
Als Beispiel nehmen wir die Zahlen 9 und 7 die modulo 2 gerechnet werden.
%
\begin{flalign*}
  9 \bmod(2) &= 1 \\
  7 \bmod(2) &= 1  \\
  9 \bmod(2) & \equiv 7 \bmod(2)
\end{flalign*}
%
%******************************************************************************
% Eulersche Funktion
%******************************************************************************
%Verbessern die Zahlen sind kleiner als die Zahl -> HAHA
\subsection{Eulersche Funktion}
Die Eulersche Funktion sagt aus, wieviele teilfremde \footnote{Zahlen sind sich teilfremd, wenn ihr grösster gemeinsamer Teiler 1 ist.} natürliche Zahlen zu einer Zahl es gibt.(Wikipedia)\\
Für den RSA-Algorithmus verwenden wir die Eulersche Funktion nur bei Primzahlen. Deshalb schauen wir uns nur die Spezialformel bei Primzahlen an.\\
Primzahlen sind durch sich selbst und durch 1 dividierbar und ihr grösster gemeinsamer Teiler ist immer 1. Somit ist die Eulersche Funktion einer Primzahl p immer $p - 1$.
%
\begin{equation*}
  \varphi(p) = p - 1
\end{equation*}
%
Wenn wir jetzt also zwei verschiedene Primzahlen haben, gilt
\begin{equation}
  \varphi(pq) = (p - 1) \cdot (q - 1)
  \label{eqn:eulersche_func}
\end{equation}
Der Beweis dafür liefert folgende Überlegung. Es gibt insgesamt $p \cdot q -1$ ganze Zahlen die kleiner sind als $p \cdot q$.\\
Es ist bei Primzahlen einfacher, die \textbf{nicht teilfremden} Zahlen zu zählen und diese dann von allen möglichen Zahlen abzuziehen, als alle teilfremden Zahlen zu zählen. Nicht teilfremd zu p sind $(q - 1) \cdot p$ und für q gilt $ (p - 1) \cdot q$.\\
Als Formel geschrieben:
%
\begin{equation*}
  \begin{split}
    \varphi(pq) & = p \cdot q -1 - (p - 1) - (q - 1)  \\
    \varphi(pq) & = p \cdot q -1 - p + 1 - q + 1  \\
    \varphi(pq) & = p \cdot q -q - p + 1  \\
    \varphi(pq) & = (p -1) \cdot (q - 1)
    \label{eqn:herleitung_eulersche_func}
  \end{split}
\end{equation*}
%
% Satz von Euler
%=====================
\subsection{Satz von Euler}
Der Satz von Euler sagt über zwei natürliche, teilfremde Zahlen, hier $n$ und $m$, Folgendes aus:
%
\begin{equation}
  m^{\varphi(n)} \bmod(n) = 1
  \label{eqn:satz_von_euler}
\end{equation}
%
Nehmen wir für $n = 15$ und als m wählen wir 4 und setzten das ganze in die obige Formel ein.
%
\begin{flalign*}
  4^{\varphi(15)} \bmod(15) = 1  \\
  4^8 \bmod(15) = 1
\end{flalign*}
%
%
% Kleiner Satz von Verma
%
%Wenn wir jetzt eine natürliche Zahl k nehmen, gilt für diese folgende Gleichung (Erleuterung siehe Buch \textit{Kryptologie}):
%
%\begin{equation}
%  \begin{split}
%    m^{1 + k \cdot \varphi(N)} \bmod(N)  &= m \cdot 1 \\
%    m \cdot m^{k \cdot \varphi(N)} \bmod(N) &= m \cdot 1
%    \label{eqn:satz_von_euler_erweitert}
%  \end{split}
%\end{equation}
%
%
%Hat man zwei verschiedene Primzahlen p und q und eine natürliche Zahl m, die kleiner ist als $p \cdot q$, dann gilt für jede natürliche Zahl k:
%
%\begin{equation}
%  m^{k \cdot (p - 1) \cdot (q - 1) +1} \bmod(p \cdot q) = m
%  \label{eqn:kleiner_satz_fermat}
%\end{equation}
%
%Dieser Satz wird in der Mathematik auch \textit{kleiner Satz von Fermat} genannt.
% Buch S 109 / 110
%
%******************************************************************************
% Euklidischer Algorithmus
%******************************************************************************
\subsection{Euklidischer Algorithmus}
Der Euklidische Algorithmus errechnet den grössten gemeinsamen Teiler (ggT) von zwei natürlichen Zahlen. Um den grössten gemeinsamen Teiler herauszufinden gibt es zwei Methoden: dass etwas ältere Subtraktionsverfahren und das moderne Modulo-Verfahren. Da das Subtraktions-Verfahren nicht die gleiche Effizenz hat, konzentrieren wir uns auf das Modulo-Verfahren.\\

Die Formel für das Modulo-Verfahren lautet:
%
\begin{equation}
  \label{eqn:euklidischer_algo}
  p = a \cdot q + r
\end{equation}
%
und ist, wie es der Name schon sagt, eine modulo Rechnung. Der Rest r wird zum neuen q, und q wird zum neuen p. \\
Zur Veranschaulichung nehmen wir zwei Zahlen $p = 302$ und $q = 146$ und setzen diese in usere Gleichung ein.
%
\begin{equation}
  \begin{split}
    302 & = 2 \cdot 146 + 10 \\
    146 & = 14 \cdot 10 + 6  \\
    10 & = 1 \cdot 6 + 4  \\
    6 & = 1 \cdot 4 + 2  \\
    4 & = 2 \cdot 2 + 0
    \label{eqn:euqulid_beweis}
  \end{split}
\end{equation}
%
Der grösste gemeinsame Teiler von 302 und 146 ist somit 2. \\[2ex]
Um dieses Verfahren zu beweisen, gehen wir davon aus, dass q, p, r und a alles natürliche Zahlen sind. Da wir im zweiten Schritt den grössten gemeinsamen Teiler von q und r errechnet haben, kann man folgende Gleichung aufstellen:
%
\begin{equation}
  ggT(p,q) = ggT(q,r)
  \label{eqn:ggT}
\end{equation}
%
Wenn also eine Zahl q und p teilt, muss diese Zahl auch r teilen. Ansonsten würde die Gleichung \ref{eqn:euqulid_beweis} nicht aufgehen. Umgekehrt muss eine Zahl die q und r teilt auch p teilen. 
% TODO: Besser schreiben
Gehen wir zurück zu unserem Beispiel. Da diese Gleichung stimmt, stimmt auch unseres Verfahren.
%
\begin{equation*}
 ggT(5984,302) = ggT(302,146) = ggT(146,10) = ggT(10,6) = ggT(6,4) = 2 
\end{equation*}
%
%******************************************************************************
% Erweiterter Euklidischer Algorithmus
%******************************************************************************
\subsubsection{Erweiterter Euklidischer Algorithmus}
Der erweiterte Euklidische Algorithmus, berechnet den grössten gemeinsamen Teiler $d$ aus zwei natürlichen Zahlen und die dazu gehörigen natürlichen Zahlen x und y, dass folgende Gleichung stimmt.
%
\begin{equation}
  d = x \cdot q + y \cdot p 
  \label{eqn:erw_euklid_algo}
\end{equation}
%
Dieser Algorithmus wird auch \textit{Vielfachsummendarstellung} gennant, denn um auf die Zahlen x und y zu kommen, muss man die Vielfachsummendarstellung anwenden.\\
Es ist eine erweiterte Form des Euklidischen Algorithmus folglich benötigt man den Euklidischen Algorithmus auch. Wir schreiben jetzt die Rechnung \ref{eqn:euqulid_beweis} so um, dass wir den Rest allein auf einer Seite haben.\\ 
\begin{equation}
  r = p - a \cdot q
  \label{eqn:form_erw_euklid}
\end{equation}
%
%
\begin{flalign*}
  10 & = 302 - 2 \cdot 146 \\
  6 & = 146 - 14 \cdot 10 \\
  4 & = 10 - 1 \cdot 6  \\
  2 & = 6 - 1 \cdot 4
  \label{eqn:erw_euklid_10}
\end{flalign*}
%
% TODO: AUSSCHREIBEN
Man startet bei der Untersten Rechnung und ersetzt die Zahl, die den nächsten oberen Rest beschreibt, durch die Gleichung. Das heisst man ersetzt in der ersten Gleichung die Zahl $4$ mit der Gleichung $10 - 1 \cdot 6$. Das macht man immer so weiter, bis man die letze Gleichun eingesetz hat. Man darf nicht zuviel vereinfachen, denn sonst kommt die Gleichung $2 = 2$ heraus.\\
%
\begin{flalign*}
  2  &= 6 - 1 \cdot 4  =  6 - 1 \cdot (10 - 1 \cdot 6) =  6 - 1 \cdot 10 + 1 \cdot 6 = 2 \cdot 6 - 1 \cdot 10 \\
%  &= 2 \cdot 6 - 1 \cdot 10  = 2 \cdot (146 - 14 \cdot 10) - 1 \cdot 10 = 2 \cdot 146 - 2 \cdot 14 \cdot 10 - 1 \cdot 10 = 2 \cdot 146 - 29 \cdot 10   \\
  &= 2 \cdot 6 - 1 \cdot 10  = 2 \cdot 146 - 2 \cdot 14 \cdot 10 - 1 \cdot 10 = 2 \cdot 146 - 29 \cdot 10   \\
  &= 2 \cdot 146 - 29 \cdot 10 = 2 \cdot 146 - 29 \cdot 302 + 58 \cdot 146 = 60 \cdot 146 - 29 \cdot 302
\end{flalign*}
%
Jetzt haben wir unsere Formel
%
\begin{equation}
 2 = 60 \cdot 146 + (-29) \cdot 302
\end{equation}
%
Um das zu beweisen schauen wir uns nochmal die Formel \ref{eqn:euklidischer_algo} an. Aus einer Vielfachsummendarstellung von $q$ und $r$, lässt sich dei Vielfachsummendarstellung von $p$ und $q$ ableiten. Man kann die Formel
%{eqn:ggT}
\begin{equation*}
  2 = x \cdot p + y \cdot q
  \label{eqn:erw_eukl_fertig}
\end{equation*}
%
auch mit r und a darstellen.
%
\begin{equation*}
  2 = x \cdot r + y \cdot a
\end{equation*}
%
Jetzt setzen wir die Formel \ref{eqn:form_erw_euklid} anstatt r ein und erhalten diese Gleichung
\begin{equation*}
  2 = x \cdot (p - a \cdot q)  + y \cdot a = x \cdot p + (y - xa) \cdot q
\end{equation*}
%
Somit haben wie die Vielfachsummendarstellung für $p$ und $q$.
%
\subsubsection{Modulare Inverse}
Haben wir zwei natürliche, teilfremde Zahlen, gibt es eine Zahl x, sodass die Formel
%
\begin{equation*}
  p \cdot x \bmod(q) \equiv 1
\end{equation*}
%
Es gibt zwei ganze Zahlen $x$ und $y$ sodass die Gleichung \ref{eqn:erw_eukl_fertig} entsteht.
Weil $y cdot q$ durch $q$ teilbar ist, gibt es bei der Divison von $x \cdot p$ durch $q$ immer 1. Somit kann man sagen
%
\begin{equation}
  p \cdot x \bmod q = 1
  \label{eqn:mod_inverse}
\end{equation}
%
Die Modulareinverse von zwei teilfremden natürlichen Zahlen $p$ und $q$, errechnet man mit dem erweiterten Euklidischen Algorithmus. Dann ist $x$ die modulare Inverse von $p$ und $q$.
